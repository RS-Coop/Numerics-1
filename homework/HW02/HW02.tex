\documentclass[11pt]{article}

    \usepackage[breakable]{tcolorbox}
    \usepackage{parskip} % Stop auto-indenting (to mimic markdown behaviour)
    
    \usepackage{iftex}
    \ifPDFTeX
    	\usepackage[T1]{fontenc}
    	\usepackage{mathpazo}
    \else
    	\usepackage{fontspec}
    \fi

    % Basic figure setup, for now with no caption control since it's done
    % automatically by Pandoc (which extracts ![](path) syntax from Markdown).
    \usepackage{graphicx}
    % Maintain compatibility with old templates. Remove in nbconvert 6.0
    \let\Oldincludegraphics\includegraphics
    % Ensure that by default, figures have no caption (until we provide a
    % proper Figure object with a Caption API and a way to capture that
    % in the conversion process - todo).
    \usepackage{caption}
    \DeclareCaptionFormat{nocaption}{}
    \captionsetup{format=nocaption,aboveskip=0pt,belowskip=0pt}

    \usepackage[Export]{adjustbox} % Used to constrain images to a maximum size
    \adjustboxset{max size={0.9\linewidth}{0.9\paperheight}}
    \usepackage{float}
    \floatplacement{figure}{H} % forces figures to be placed at the correct location
    \usepackage{xcolor} % Allow colors to be defined
    \usepackage{enumerate} % Needed for markdown enumerations to work
    \usepackage{geometry} % Used to adjust the document margins
    \usepackage{amsmath} % Equations
    \usepackage{amssymb} % Equations
    \usepackage{textcomp} % defines textquotesingle
    % Hack from http://tex.stackexchange.com/a/47451/13684:
    \AtBeginDocument{%
        \def\PYZsq{\textquotesingle}% Upright quotes in Pygmentized code
    }
    \usepackage{upquote} % Upright quotes for verbatim code
    \usepackage{eurosym} % defines \euro
    \usepackage[mathletters]{ucs} % Extended unicode (utf-8) support
    \usepackage{fancyvrb} % verbatim replacement that allows latex
    \usepackage{grffile} % extends the file name processing of package graphics 
                         % to support a larger range
    \makeatletter % fix for grffile with XeLaTeX
    \def\Gread@@xetex#1{%
      \IfFileExists{"\Gin@base".bb}%
      {\Gread@eps{\Gin@base.bb}}%
      {\Gread@@xetex@aux#1}%
    }
    \makeatother

    % The hyperref package gives us a pdf with properly built
    % internal navigation ('pdf bookmarks' for the table of contents,
    % internal cross-reference links, web links for URLs, etc.)
    \usepackage{hyperref}
    % The default LaTeX title has an obnoxious amount of whitespace. By default,
    % titling removes some of it. It also provides customization options.
    \usepackage{titling}
    \usepackage{longtable} % longtable support required by pandoc >1.10
    \usepackage{booktabs}  % table support for pandoc > 1.12.2
    \usepackage[inline]{enumitem} % IRkernel/repr support (it uses the enumerate* environment)
    \usepackage[normalem]{ulem} % ulem is needed to support strikethroughs (\sout)
                                % normalem makes italics be italics, not underlines
    \usepackage{mathrsfs}
    

    
    % Colors for the hyperref package
    \definecolor{urlcolor}{rgb}{0,.145,.698}
    \definecolor{linkcolor}{rgb}{.71,0.21,0.01}
    \definecolor{citecolor}{rgb}{.12,.54,.11}

    % ANSI colors
    \definecolor{ansi-black}{HTML}{3E424D}
    \definecolor{ansi-black-intense}{HTML}{282C36}
    \definecolor{ansi-red}{HTML}{E75C58}
    \definecolor{ansi-red-intense}{HTML}{B22B31}
    \definecolor{ansi-green}{HTML}{00A250}
    \definecolor{ansi-green-intense}{HTML}{007427}
    \definecolor{ansi-yellow}{HTML}{DDB62B}
    \definecolor{ansi-yellow-intense}{HTML}{B27D12}
    \definecolor{ansi-blue}{HTML}{208FFB}
    \definecolor{ansi-blue-intense}{HTML}{0065CA}
    \definecolor{ansi-magenta}{HTML}{D160C4}
    \definecolor{ansi-magenta-intense}{HTML}{A03196}
    \definecolor{ansi-cyan}{HTML}{60C6C8}
    \definecolor{ansi-cyan-intense}{HTML}{258F8F}
    \definecolor{ansi-white}{HTML}{C5C1B4}
    \definecolor{ansi-white-intense}{HTML}{A1A6B2}
    \definecolor{ansi-default-inverse-fg}{HTML}{FFFFFF}
    \definecolor{ansi-default-inverse-bg}{HTML}{000000}

    % commands and environments needed by pandoc snippets
    % extracted from the output of `pandoc -s`
    \providecommand{\tightlist}{%
      \setlength{\itemsep}{0pt}\setlength{\parskip}{0pt}}
    \DefineVerbatimEnvironment{Highlighting}{Verbatim}{commandchars=\\\{\}}
    % Add ',fontsize=\small' for more characters per line
    \newenvironment{Shaded}{}{}
    \newcommand{\KeywordTok}[1]{\textcolor[rgb]{0.00,0.44,0.13}{\textbf{{#1}}}}
    \newcommand{\DataTypeTok}[1]{\textcolor[rgb]{0.56,0.13,0.00}{{#1}}}
    \newcommand{\DecValTok}[1]{\textcolor[rgb]{0.25,0.63,0.44}{{#1}}}
    \newcommand{\BaseNTok}[1]{\textcolor[rgb]{0.25,0.63,0.44}{{#1}}}
    \newcommand{\FloatTok}[1]{\textcolor[rgb]{0.25,0.63,0.44}{{#1}}}
    \newcommand{\CharTok}[1]{\textcolor[rgb]{0.25,0.44,0.63}{{#1}}}
    \newcommand{\StringTok}[1]{\textcolor[rgb]{0.25,0.44,0.63}{{#1}}}
    \newcommand{\CommentTok}[1]{\textcolor[rgb]{0.38,0.63,0.69}{\textit{{#1}}}}
    \newcommand{\OtherTok}[1]{\textcolor[rgb]{0.00,0.44,0.13}{{#1}}}
    \newcommand{\AlertTok}[1]{\textcolor[rgb]{1.00,0.00,0.00}{\textbf{{#1}}}}
    \newcommand{\FunctionTok}[1]{\textcolor[rgb]{0.02,0.16,0.49}{{#1}}}
    \newcommand{\RegionMarkerTok}[1]{{#1}}
    \newcommand{\ErrorTok}[1]{\textcolor[rgb]{1.00,0.00,0.00}{\textbf{{#1}}}}
    \newcommand{\NormalTok}[1]{{#1}}
    
    % Additional commands for more recent versions of Pandoc
    \newcommand{\ConstantTok}[1]{\textcolor[rgb]{0.53,0.00,0.00}{{#1}}}
    \newcommand{\SpecialCharTok}[1]{\textcolor[rgb]{0.25,0.44,0.63}{{#1}}}
    \newcommand{\VerbatimStringTok}[1]{\textcolor[rgb]{0.25,0.44,0.63}{{#1}}}
    \newcommand{\SpecialStringTok}[1]{\textcolor[rgb]{0.73,0.40,0.53}{{#1}}}
    \newcommand{\ImportTok}[1]{{#1}}
    \newcommand{\DocumentationTok}[1]{\textcolor[rgb]{0.73,0.13,0.13}{\textit{{#1}}}}
    \newcommand{\AnnotationTok}[1]{\textcolor[rgb]{0.38,0.63,0.69}{\textbf{\textit{{#1}}}}}
    \newcommand{\CommentVarTok}[1]{\textcolor[rgb]{0.38,0.63,0.69}{\textbf{\textit{{#1}}}}}
    \newcommand{\VariableTok}[1]{\textcolor[rgb]{0.10,0.09,0.49}{{#1}}}
    \newcommand{\ControlFlowTok}[1]{\textcolor[rgb]{0.00,0.44,0.13}{\textbf{{#1}}}}
    \newcommand{\OperatorTok}[1]{\textcolor[rgb]{0.40,0.40,0.40}{{#1}}}
    \newcommand{\BuiltInTok}[1]{{#1}}
    \newcommand{\ExtensionTok}[1]{{#1}}
    \newcommand{\PreprocessorTok}[1]{\textcolor[rgb]{0.74,0.48,0.00}{{#1}}}
    \newcommand{\AttributeTok}[1]{\textcolor[rgb]{0.49,0.56,0.16}{{#1}}}
    \newcommand{\InformationTok}[1]{\textcolor[rgb]{0.38,0.63,0.69}{\textbf{\textit{{#1}}}}}
    \newcommand{\WarningTok}[1]{\textcolor[rgb]{0.38,0.63,0.69}{\textbf{\textit{{#1}}}}}
    
    
    % Define a nice break command that doesn't care if a line doesn't already
    % exist.
    \def\br{\hspace*{\fill} \\* }
    % Math Jax compatibility definitions
    \def\gt{>}
    \def\lt{<}
    \let\Oldtex\TeX
    \let\Oldlatex\LaTeX
    \renewcommand{\TeX}{\textrm{\Oldtex}}
    \renewcommand{\LaTeX}{\textrm{\Oldlatex}}
    % Document parameters
    % Document title
    \title{Numerics 1: HW 2}
    \author{Cooper Simpson}
    
    
    
    
    
% Pygments definitions
\makeatletter
\def\PY@reset{\let\PY@it=\relax \let\PY@bf=\relax%
    \let\PY@ul=\relax \let\PY@tc=\relax%
    \let\PY@bc=\relax \let\PY@ff=\relax}
\def\PY@tok#1{\csname PY@tok@#1\endcsname}
\def\PY@toks#1+{\ifx\relax#1\empty\else%
    \PY@tok{#1}\expandafter\PY@toks\fi}
\def\PY@do#1{\PY@bc{\PY@tc{\PY@ul{%
    \PY@it{\PY@bf{\PY@ff{#1}}}}}}}
\def\PY#1#2{\PY@reset\PY@toks#1+\relax+\PY@do{#2}}

\expandafter\def\csname PY@tok@w\endcsname{\def\PY@tc##1{\textcolor[rgb]{0.73,0.73,0.73}{##1}}}
\expandafter\def\csname PY@tok@c\endcsname{\let\PY@it=\textit\def\PY@tc##1{\textcolor[rgb]{0.25,0.50,0.50}{##1}}}
\expandafter\def\csname PY@tok@cp\endcsname{\def\PY@tc##1{\textcolor[rgb]{0.74,0.48,0.00}{##1}}}
\expandafter\def\csname PY@tok@k\endcsname{\let\PY@bf=\textbf\def\PY@tc##1{\textcolor[rgb]{0.00,0.50,0.00}{##1}}}
\expandafter\def\csname PY@tok@kp\endcsname{\def\PY@tc##1{\textcolor[rgb]{0.00,0.50,0.00}{##1}}}
\expandafter\def\csname PY@tok@kt\endcsname{\def\PY@tc##1{\textcolor[rgb]{0.69,0.00,0.25}{##1}}}
\expandafter\def\csname PY@tok@o\endcsname{\def\PY@tc##1{\textcolor[rgb]{0.40,0.40,0.40}{##1}}}
\expandafter\def\csname PY@tok@ow\endcsname{\let\PY@bf=\textbf\def\PY@tc##1{\textcolor[rgb]{0.67,0.13,1.00}{##1}}}
\expandafter\def\csname PY@tok@nb\endcsname{\def\PY@tc##1{\textcolor[rgb]{0.00,0.50,0.00}{##1}}}
\expandafter\def\csname PY@tok@nf\endcsname{\def\PY@tc##1{\textcolor[rgb]{0.00,0.00,1.00}{##1}}}
\expandafter\def\csname PY@tok@nc\endcsname{\let\PY@bf=\textbf\def\PY@tc##1{\textcolor[rgb]{0.00,0.00,1.00}{##1}}}
\expandafter\def\csname PY@tok@nn\endcsname{\let\PY@bf=\textbf\def\PY@tc##1{\textcolor[rgb]{0.00,0.00,1.00}{##1}}}
\expandafter\def\csname PY@tok@ne\endcsname{\let\PY@bf=\textbf\def\PY@tc##1{\textcolor[rgb]{0.82,0.25,0.23}{##1}}}
\expandafter\def\csname PY@tok@nv\endcsname{\def\PY@tc##1{\textcolor[rgb]{0.10,0.09,0.49}{##1}}}
\expandafter\def\csname PY@tok@no\endcsname{\def\PY@tc##1{\textcolor[rgb]{0.53,0.00,0.00}{##1}}}
\expandafter\def\csname PY@tok@nl\endcsname{\def\PY@tc##1{\textcolor[rgb]{0.63,0.63,0.00}{##1}}}
\expandafter\def\csname PY@tok@ni\endcsname{\let\PY@bf=\textbf\def\PY@tc##1{\textcolor[rgb]{0.60,0.60,0.60}{##1}}}
\expandafter\def\csname PY@tok@na\endcsname{\def\PY@tc##1{\textcolor[rgb]{0.49,0.56,0.16}{##1}}}
\expandafter\def\csname PY@tok@nt\endcsname{\let\PY@bf=\textbf\def\PY@tc##1{\textcolor[rgb]{0.00,0.50,0.00}{##1}}}
\expandafter\def\csname PY@tok@nd\endcsname{\def\PY@tc##1{\textcolor[rgb]{0.67,0.13,1.00}{##1}}}
\expandafter\def\csname PY@tok@s\endcsname{\def\PY@tc##1{\textcolor[rgb]{0.73,0.13,0.13}{##1}}}
\expandafter\def\csname PY@tok@sd\endcsname{\let\PY@it=\textit\def\PY@tc##1{\textcolor[rgb]{0.73,0.13,0.13}{##1}}}
\expandafter\def\csname PY@tok@si\endcsname{\let\PY@bf=\textbf\def\PY@tc##1{\textcolor[rgb]{0.73,0.40,0.53}{##1}}}
\expandafter\def\csname PY@tok@se\endcsname{\let\PY@bf=\textbf\def\PY@tc##1{\textcolor[rgb]{0.73,0.40,0.13}{##1}}}
\expandafter\def\csname PY@tok@sr\endcsname{\def\PY@tc##1{\textcolor[rgb]{0.73,0.40,0.53}{##1}}}
\expandafter\def\csname PY@tok@ss\endcsname{\def\PY@tc##1{\textcolor[rgb]{0.10,0.09,0.49}{##1}}}
\expandafter\def\csname PY@tok@sx\endcsname{\def\PY@tc##1{\textcolor[rgb]{0.00,0.50,0.00}{##1}}}
\expandafter\def\csname PY@tok@m\endcsname{\def\PY@tc##1{\textcolor[rgb]{0.40,0.40,0.40}{##1}}}
\expandafter\def\csname PY@tok@gh\endcsname{\let\PY@bf=\textbf\def\PY@tc##1{\textcolor[rgb]{0.00,0.00,0.50}{##1}}}
\expandafter\def\csname PY@tok@gu\endcsname{\let\PY@bf=\textbf\def\PY@tc##1{\textcolor[rgb]{0.50,0.00,0.50}{##1}}}
\expandafter\def\csname PY@tok@gd\endcsname{\def\PY@tc##1{\textcolor[rgb]{0.63,0.00,0.00}{##1}}}
\expandafter\def\csname PY@tok@gi\endcsname{\def\PY@tc##1{\textcolor[rgb]{0.00,0.63,0.00}{##1}}}
\expandafter\def\csname PY@tok@gr\endcsname{\def\PY@tc##1{\textcolor[rgb]{1.00,0.00,0.00}{##1}}}
\expandafter\def\csname PY@tok@ge\endcsname{\let\PY@it=\textit}
\expandafter\def\csname PY@tok@gs\endcsname{\let\PY@bf=\textbf}
\expandafter\def\csname PY@tok@gp\endcsname{\let\PY@bf=\textbf\def\PY@tc##1{\textcolor[rgb]{0.00,0.00,0.50}{##1}}}
\expandafter\def\csname PY@tok@go\endcsname{\def\PY@tc##1{\textcolor[rgb]{0.53,0.53,0.53}{##1}}}
\expandafter\def\csname PY@tok@gt\endcsname{\def\PY@tc##1{\textcolor[rgb]{0.00,0.27,0.87}{##1}}}
\expandafter\def\csname PY@tok@err\endcsname{\def\PY@bc##1{\setlength{\fboxsep}{0pt}\fcolorbox[rgb]{1.00,0.00,0.00}{1,1,1}{\strut ##1}}}
\expandafter\def\csname PY@tok@kc\endcsname{\let\PY@bf=\textbf\def\PY@tc##1{\textcolor[rgb]{0.00,0.50,0.00}{##1}}}
\expandafter\def\csname PY@tok@kd\endcsname{\let\PY@bf=\textbf\def\PY@tc##1{\textcolor[rgb]{0.00,0.50,0.00}{##1}}}
\expandafter\def\csname PY@tok@kn\endcsname{\let\PY@bf=\textbf\def\PY@tc##1{\textcolor[rgb]{0.00,0.50,0.00}{##1}}}
\expandafter\def\csname PY@tok@kr\endcsname{\let\PY@bf=\textbf\def\PY@tc##1{\textcolor[rgb]{0.00,0.50,0.00}{##1}}}
\expandafter\def\csname PY@tok@bp\endcsname{\def\PY@tc##1{\textcolor[rgb]{0.00,0.50,0.00}{##1}}}
\expandafter\def\csname PY@tok@fm\endcsname{\def\PY@tc##1{\textcolor[rgb]{0.00,0.00,1.00}{##1}}}
\expandafter\def\csname PY@tok@vc\endcsname{\def\PY@tc##1{\textcolor[rgb]{0.10,0.09,0.49}{##1}}}
\expandafter\def\csname PY@tok@vg\endcsname{\def\PY@tc##1{\textcolor[rgb]{0.10,0.09,0.49}{##1}}}
\expandafter\def\csname PY@tok@vi\endcsname{\def\PY@tc##1{\textcolor[rgb]{0.10,0.09,0.49}{##1}}}
\expandafter\def\csname PY@tok@vm\endcsname{\def\PY@tc##1{\textcolor[rgb]{0.10,0.09,0.49}{##1}}}
\expandafter\def\csname PY@tok@sa\endcsname{\def\PY@tc##1{\textcolor[rgb]{0.73,0.13,0.13}{##1}}}
\expandafter\def\csname PY@tok@sb\endcsname{\def\PY@tc##1{\textcolor[rgb]{0.73,0.13,0.13}{##1}}}
\expandafter\def\csname PY@tok@sc\endcsname{\def\PY@tc##1{\textcolor[rgb]{0.73,0.13,0.13}{##1}}}
\expandafter\def\csname PY@tok@dl\endcsname{\def\PY@tc##1{\textcolor[rgb]{0.73,0.13,0.13}{##1}}}
\expandafter\def\csname PY@tok@s2\endcsname{\def\PY@tc##1{\textcolor[rgb]{0.73,0.13,0.13}{##1}}}
\expandafter\def\csname PY@tok@sh\endcsname{\def\PY@tc##1{\textcolor[rgb]{0.73,0.13,0.13}{##1}}}
\expandafter\def\csname PY@tok@s1\endcsname{\def\PY@tc##1{\textcolor[rgb]{0.73,0.13,0.13}{##1}}}
\expandafter\def\csname PY@tok@mb\endcsname{\def\PY@tc##1{\textcolor[rgb]{0.40,0.40,0.40}{##1}}}
\expandafter\def\csname PY@tok@mf\endcsname{\def\PY@tc##1{\textcolor[rgb]{0.40,0.40,0.40}{##1}}}
\expandafter\def\csname PY@tok@mh\endcsname{\def\PY@tc##1{\textcolor[rgb]{0.40,0.40,0.40}{##1}}}
\expandafter\def\csname PY@tok@mi\endcsname{\def\PY@tc##1{\textcolor[rgb]{0.40,0.40,0.40}{##1}}}
\expandafter\def\csname PY@tok@il\endcsname{\def\PY@tc##1{\textcolor[rgb]{0.40,0.40,0.40}{##1}}}
\expandafter\def\csname PY@tok@mo\endcsname{\def\PY@tc##1{\textcolor[rgb]{0.40,0.40,0.40}{##1}}}
\expandafter\def\csname PY@tok@ch\endcsname{\let\PY@it=\textit\def\PY@tc##1{\textcolor[rgb]{0.25,0.50,0.50}{##1}}}
\expandafter\def\csname PY@tok@cm\endcsname{\let\PY@it=\textit\def\PY@tc##1{\textcolor[rgb]{0.25,0.50,0.50}{##1}}}
\expandafter\def\csname PY@tok@cpf\endcsname{\let\PY@it=\textit\def\PY@tc##1{\textcolor[rgb]{0.25,0.50,0.50}{##1}}}
\expandafter\def\csname PY@tok@c1\endcsname{\let\PY@it=\textit\def\PY@tc##1{\textcolor[rgb]{0.25,0.50,0.50}{##1}}}
\expandafter\def\csname PY@tok@cs\endcsname{\let\PY@it=\textit\def\PY@tc##1{\textcolor[rgb]{0.25,0.50,0.50}{##1}}}

\def\PYZbs{\char`\\}
\def\PYZus{\char`\_}
\def\PYZob{\char`\{}
\def\PYZcb{\char`\}}
\def\PYZca{\char`\^}
\def\PYZam{\char`\&}
\def\PYZlt{\char`\<}
\def\PYZgt{\char`\>}
\def\PYZsh{\char`\#}
\def\PYZpc{\char`\%}
\def\PYZdl{\char`\$}
\def\PYZhy{\char`\-}
\def\PYZsq{\char`\'}
\def\PYZdq{\char`\"}
\def\PYZti{\char`\~}
% for compatibility with earlier versions
\def\PYZat{@}
\def\PYZlb{[}
\def\PYZrb{]}
\makeatother


    % For linebreaks inside Verbatim environment from package fancyvrb. 
    \makeatletter
        \newbox\Wrappedcontinuationbox 
        \newbox\Wrappedvisiblespacebox 
        \newcommand*\Wrappedvisiblespace {\textcolor{red}{\textvisiblespace}} 
        \newcommand*\Wrappedcontinuationsymbol {\textcolor{red}{\llap{\tiny$\m@th\hookrightarrow$}}} 
        \newcommand*\Wrappedcontinuationindent {3ex } 
        \newcommand*\Wrappedafterbreak {\kern\Wrappedcontinuationindent\copy\Wrappedcontinuationbox} 
        % Take advantage of the already applied Pygments mark-up to insert 
        % potential linebreaks for TeX processing. 
        %        {, <, #, %, $, ' and ": go to next line. 
        %        _, }, ^, &, >, - and ~: stay at end of broken line. 
        % Use of \textquotesingle for straight quote. 
        \newcommand*\Wrappedbreaksatspecials {% 
            \def\PYGZus{\discretionary{\char`\_}{\Wrappedafterbreak}{\char`\_}}% 
            \def\PYGZob{\discretionary{}{\Wrappedafterbreak\char`\{}{\char`\{}}% 
            \def\PYGZcb{\discretionary{\char`\}}{\Wrappedafterbreak}{\char`\}}}% 
            \def\PYGZca{\discretionary{\char`\^}{\Wrappedafterbreak}{\char`\^}}% 
            \def\PYGZam{\discretionary{\char`\&}{\Wrappedafterbreak}{\char`\&}}% 
            \def\PYGZlt{\discretionary{}{\Wrappedafterbreak\char`\<}{\char`\<}}% 
            \def\PYGZgt{\discretionary{\char`\>}{\Wrappedafterbreak}{\char`\>}}% 
            \def\PYGZsh{\discretionary{}{\Wrappedafterbreak\char`\#}{\char`\#}}% 
            \def\PYGZpc{\discretionary{}{\Wrappedafterbreak\char`\%}{\char`\%}}% 
            \def\PYGZdl{\discretionary{}{\Wrappedafterbreak\char`\$}{\char`\$}}% 
            \def\PYGZhy{\discretionary{\char`\-}{\Wrappedafterbreak}{\char`\-}}% 
            \def\PYGZsq{\discretionary{}{\Wrappedafterbreak\textquotesingle}{\textquotesingle}}% 
            \def\PYGZdq{\discretionary{}{\Wrappedafterbreak\char`\"}{\char`\"}}% 
            \def\PYGZti{\discretionary{\char`\~}{\Wrappedafterbreak}{\char`\~}}% 
        } 
        % Some characters . , ; ? ! / are not pygmentized. 
        % This macro makes them "active" and they will insert potential linebreaks 
        \newcommand*\Wrappedbreaksatpunct {% 
            \lccode`\~`\.\lowercase{\def~}{\discretionary{\hbox{\char`\.}}{\Wrappedafterbreak}{\hbox{\char`\.}}}% 
            \lccode`\~`\,\lowercase{\def~}{\discretionary{\hbox{\char`\,}}{\Wrappedafterbreak}{\hbox{\char`\,}}}% 
            \lccode`\~`\;\lowercase{\def~}{\discretionary{\hbox{\char`\;}}{\Wrappedafterbreak}{\hbox{\char`\;}}}% 
            \lccode`\~`\:\lowercase{\def~}{\discretionary{\hbox{\char`\:}}{\Wrappedafterbreak}{\hbox{\char`\:}}}% 
            \lccode`\~`\?\lowercase{\def~}{\discretionary{\hbox{\char`\?}}{\Wrappedafterbreak}{\hbox{\char`\?}}}% 
            \lccode`\~`\!\lowercase{\def~}{\discretionary{\hbox{\char`\!}}{\Wrappedafterbreak}{\hbox{\char`\!}}}% 
            \lccode`\~`\/\lowercase{\def~}{\discretionary{\hbox{\char`\/}}{\Wrappedafterbreak}{\hbox{\char`\/}}}% 
            \catcode`\.\active
            \catcode`\,\active 
            \catcode`\;\active
            \catcode`\:\active
            \catcode`\?\active
            \catcode`\!\active
            \catcode`\/\active 
            \lccode`\~`\~ 	
        }
    \makeatother

    \let\OriginalVerbatim=\Verbatim
    \makeatletter
    \renewcommand{\Verbatim}[1][1]{%
        %\parskip\z@skip
        \sbox\Wrappedcontinuationbox {\Wrappedcontinuationsymbol}%
        \sbox\Wrappedvisiblespacebox {\FV@SetupFont\Wrappedvisiblespace}%
        \def\FancyVerbFormatLine ##1{\hsize\linewidth
            \vtop{\raggedright\hyphenpenalty\z@\exhyphenpenalty\z@
                \doublehyphendemerits\z@\finalhyphendemerits\z@
                \strut ##1\strut}%
        }%
        % If the linebreak is at a space, the latter will be displayed as visible
        % space at end of first line, and a continuation symbol starts next line.
        % Stretch/shrink are however usually zero for typewriter font.
        \def\FV@Space {%
            \nobreak\hskip\z@ plus\fontdimen3\font minus\fontdimen4\font
            \discretionary{\copy\Wrappedvisiblespacebox}{\Wrappedafterbreak}
            {\kern\fontdimen2\font}%
        }%
        
        % Allow breaks at special characters using \PYG... macros.
        \Wrappedbreaksatspecials
        % Breaks at punctuation characters . , ; ? ! and / need catcode=\active 	
        \OriginalVerbatim[#1,codes*=\Wrappedbreaksatpunct]%
    }
    \makeatother

    % Exact colors from NB
    \definecolor{incolor}{HTML}{303F9F}
    \definecolor{outcolor}{HTML}{D84315}
    \definecolor{cellborder}{HTML}{CFCFCF}
    \definecolor{cellbackground}{HTML}{F7F7F7}
    
    % prompt
    \makeatletter
    \newcommand{\boxspacing}{\kern\kvtcb@left@rule\kern\kvtcb@boxsep}
    \makeatother
    \newcommand{\prompt}[4]{
        \ttfamily\llap{{\color{#2}[#3]:\hspace{3pt}#4}}\vspace{-\baselineskip}
    }
    

    
    % Prevent overflowing lines due to hard-to-break entities
    \sloppy 
    % Setup hyperref package
    \hypersetup{
      breaklinks=true,  % so long urls are correctly broken across lines
      colorlinks=true,
      urlcolor=urlcolor,
      linkcolor=linkcolor,
      citecolor=citecolor,
      }
    % Slightly bigger margins than the latex defaults
    
    \geometry{verbose,tmargin=1in,bmargin=1in,lmargin=1in,rmargin=1in}
    
    

\begin{document}
    
    \maketitle
    
    

    

\hypertarget{cooper-simpson}{%
\subsection*{Cooper Simpson}\label{cooper-simpson}}

    \hypertarget{setup}{%
\subsection*{Setup}\label{setup}}

    \begin{tcolorbox}[breakable, size=fbox, boxrule=1pt, pad at break*=1mm,colback=cellbackground, colframe=cellborder]
\prompt{In}{incolor}{2}{\boxspacing}
\begin{Verbatim}[commandchars=\\\{\}]
\PY{k+kn}{import} \PY{n+nn}{numpy} \PY{k}{as} \PY{n+nn}{np}
\PY{k+kn}{from} \PY{n+nn}{decimal} \PY{k+kn}{import} \PY{o}{*}
\PY{k+kn}{from} \PY{n+nn}{scipy}\PY{n+nn}{.}\PY{n+nn}{special} \PY{k+kn}{import} \PY{n}{erf}
\PY{k+kn}{import} \PY{n+nn}{pandas} \PY{k}{as} \PY{n+nn}{pd}
\PY{k+kn}{import} \PY{n+nn}{matplotlib}\PY{n+nn}{.}\PY{n+nn}{pyplot} \PY{k}{as} \PY{n+nn}{plt}
\PY{k+kn}{from} \PY{n+nn}{IPython}\PY{n+nn}{.}\PY{n+nn}{display} \PY{k+kn}{import} \PY{n}{display}
\PY{k+kn}{import} \PY{n+nn}{seaborn} \PY{k}{as} \PY{n+nn}{sns}
\PY{n}{sns}\PY{o}{.}\PY{n}{set}\PY{p}{(}\PY{p}{)}
\end{Verbatim}
\end{tcolorbox}

    \hypertarget{root-finding-algorithms}{%
\subsubsection*{Root Finding Algorithms}\label{root-finding-algorithms}}

The coded algorithms for the Bisection method and Newton's method are
given here as they are used throughout the problems.

    \begin{tcolorbox}[breakable, size=fbox, boxrule=1pt, pad at break*=1mm,colback=cellbackground, colframe=cellborder]
\prompt{In}{incolor}{3}{\boxspacing}
\begin{Verbatim}[commandchars=\\\{\}]
\PY{l+s+sd}{\PYZsq{}\PYZsq{}\PYZsq{}}
\PY{l+s+sd}{Function: Bisection}
\PY{l+s+sd}{Input:}
\PY{l+s+sd}{    a: left end point guess}
\PY{l+s+sd}{    b: right end point guess}
\PY{l+s+sd}{    f: a callable function to find the root}
\PY{l+s+sd}{    tol: tolerance for the root}
\PY{l+s+sd}{    imax: max iterations}
\PY{l+s+sd}{    info: whether to return number of iterations}
\PY{l+s+sd}{Output:}
\PY{l+s+sd}{    \PYZhy{}the root within the tolerance}
\PY{l+s+sd}{    \PYZhy{}number of iterations to identify root (optional)}
\PY{l+s+sd}{Errors:}
\PY{l+s+sd}{    \PYZhy{}if given function is not callable}
\PY{l+s+sd}{    \PYZhy{}if no identifiable root bounded by initial guesses}
\PY{l+s+sd}{\PYZsq{}\PYZsq{}\PYZsq{}}
\PY{k}{def} \PY{n+nf}{bisection}\PY{p}{(}\PY{n}{a}\PY{p}{,} \PY{n}{b}\PY{p}{,} \PY{n}{f}\PY{p}{,} \PY{n}{tol}\PY{o}{=}\PY{l+m+mf}{1E\PYZhy{}4}\PY{p}{,} \PY{n}{imax}\PY{o}{=}\PY{l+m+mi}{1000}\PY{p}{,} \PY{n}{tab}\PY{o}{=}\PY{k+kc}{False}\PY{p}{)}\PY{p}{:}
    \PY{c+c1}{\PYZsh{}Check if f is callable}
    \PY{k}{if} \PY{o+ow}{not} \PY{n}{callable}\PY{p}{(}\PY{n}{f}\PY{p}{)}\PY{p}{:}
        \PY{k}{raise} \PY{n+ne}{ValueError}\PY{p}{(}\PY{l+s+s1}{\PYZsq{}}\PY{l+s+s1}{Given function not callable}\PY{l+s+s1}{\PYZsq{}}\PY{p}{)}
    
    \PY{c+c1}{\PYZsh{}Check if there is a zero}
    \PY{k}{if} \PY{n}{f}\PY{p}{(}\PY{n}{a}\PY{p}{)}\PY{o}{*}\PY{n}{f}\PY{p}{(}\PY{n}{b}\PY{p}{)}\PY{o}{\PYZgt{}}\PY{l+m+mi}{0}\PY{p}{:}
        \PY{k}{raise} \PY{n+ne}{ValueError}\PY{p}{(}\PY{l+s+s1}{\PYZsq{}}\PY{l+s+s1}{Initial guesses do not bound a zero, }\PY{l+s+se}{\PYZbs{}}
\PY{l+s+s1}{                         or guesses are poor.}\PY{l+s+s1}{\PYZsq{}}\PY{p}{)}
    
    \PY{c+c1}{\PYZsh{}Save iterates?}
    \PY{k}{if} \PY{n}{tab}\PY{p}{:}
        \PY{n}{iterates} \PY{o}{=} \PY{p}{[}\PY{p}{]}
        
    \PY{c+c1}{\PYZsh{}Bisect}
    \PY{k}{for} \PY{n}{i} \PY{o+ow}{in} \PY{n+nb}{range}\PY{p}{(}\PY{n}{imax}\PY{p}{)}\PY{p}{:} \PY{c+c1}{\PYZsh{}Stay under max iterations}
        \PY{n}{c} \PY{o}{=} \PY{p}{(}\PY{n}{a}\PY{o}{+}\PY{n}{b}\PY{p}{)}\PY{o}{/}\PY{l+m+mi}{2} \PY{c+c1}{\PYZsh{}Take the midpoint}
        
        \PY{k}{if} \PY{n}{tab}\PY{p}{:}
            \PY{n}{iterates}\PY{o}{.}\PY{n}{append}\PY{p}{(}\PY{n}{x\PYZus{}new}\PY{p}{)}
        
        \PY{k}{if} \PY{n}{f}\PY{p}{(}\PY{n}{c}\PY{p}{)}\PY{o}{==}\PY{l+m+mi}{0} \PY{o+ow}{or} \PY{p}{(}\PY{n}{b}\PY{o}{\PYZhy{}}\PY{n}{a}\PY{p}{)}\PY{o}{/}\PY{l+m+mi}{2}\PY{o}{\PYZlt{}}\PY{n}{tol}\PY{p}{:} \PY{c+c1}{\PYZsh{}Check to see if root is found}
            \PY{k}{if} \PY{n}{tab}\PY{p}{:}
                \PY{k}{return} \PY{n}{c}\PY{p}{,} \PY{n}{iterates}
            \PY{k}{return} \PY{n}{c}

        \PY{k}{if} \PY{n}{f}\PY{p}{(}\PY{n}{a}\PY{p}{)}\PY{o}{*}\PY{n}{f}\PY{p}{(}\PY{n}{c}\PY{p}{)}\PY{o}{\PYZlt{}}\PY{l+m+mi}{0}\PY{p}{:} \PY{c+c1}{\PYZsh{}Reset the interval}
            \PY{n}{b} \PY{o}{=} \PY{n}{c}
        \PY{k}{else}\PY{p}{:}
            \PY{n}{a} \PY{o}{=} \PY{n}{c}
       
    \PY{c+c1}{\PYZsh{}Something failed along the way}
    \PY{k}{raise} \PY{n+ne}{ValueError}\PY{p}{(}\PY{l+s+s1}{\PYZsq{}}\PY{l+s+s1}{Maximum number of iterations exceeded.}\PY{l+s+s1}{\PYZsq{}}\PY{p}{)} 


\PY{l+s+sd}{\PYZsq{}\PYZsq{}\PYZsq{}}
\PY{l+s+sd}{Function: Newton}
\PY{l+s+sd}{Input:}
\PY{l+s+sd}{    x0: the initial root guess}
\PY{l+s+sd}{    f: a callable function for root finding}
\PY{l+s+sd}{    df: a callable derivative of the function}
\PY{l+s+sd}{    p: rate parameter for modified newtons method}
\PY{l+s+sd}{    tol: minimum acceptable tolerance}
\PY{l+s+sd}{    imax: maximum iterations}
\PY{l+s+sd}{    tab: boolean to keep track of iterates or not}
\PY{l+s+sd}{    error: (rel)ative or (abs)olute error quantification}
\PY{l+s+sd}{    decimal: use Python decimal representation}
\PY{l+s+sd}{Output:}
\PY{l+s+sd}{    \PYZhy{}The root estimate within the tolerance if successful}
\PY{l+s+sd}{    \PYZhy{}A tabulation of the root iterates}
\PY{l+s+sd}{Errors:}
\PY{l+s+sd}{    \PYZhy{}If functions are not callable}
\PY{l+s+sd}{    \PYZhy{}If maximum number of iterations are exceeded}
\PY{l+s+sd}{\PYZsq{}\PYZsq{}\PYZsq{}}
\PY{k}{def} \PY{n+nf}{newton}\PY{p}{(}\PY{n}{x0}\PY{p}{,} \PY{n}{f}\PY{p}{,} \PY{n}{df}\PY{p}{,} \PY{n}{p}\PY{o}{=}\PY{l+m+mi}{1}\PY{p}{,} \PY{n}{tol}\PY{o}{=}\PY{l+m+mf}{1E\PYZhy{}6}\PY{p}{,} \PY{n}{imax}\PY{o}{=}\PY{l+m+mi}{1000}\PY{p}{,} \PY{n}{tab}\PY{o}{=}\PY{k+kc}{False}\PY{p}{,} \PY{n}{error}\PY{o}{=}\PY{l+s+s1}{\PYZsq{}}\PY{l+s+s1}{rel}\PY{l+s+s1}{\PYZsq{}}\PY{p}{,} \PY{n}{decimal}\PY{o}{=}\PY{k+kc}{False}\PY{p}{)}\PY{p}{:}
    \PY{c+c1}{\PYZsh{}Check if functions are callable}
    \PY{k}{if} \PY{o+ow}{not} \PY{n}{callable}\PY{p}{(}\PY{n}{f}\PY{p}{)} \PY{o+ow}{or} \PY{o+ow}{not} \PY{n}{callable}\PY{p}{(}\PY{n}{df}\PY{p}{)}\PY{p}{:}
        \PY{k}{raise} \PY{n+ne}{ValueError}\PY{p}{(}\PY{l+s+s1}{\PYZsq{}}\PY{l+s+s1}{Function not callable}\PY{l+s+s1}{\PYZsq{}}\PY{p}{)}
    
    \PY{c+c1}{\PYZsh{}Save iterates?}
    \PY{k}{if} \PY{n}{tab}\PY{p}{:}
        \PY{n}{E} \PY{o}{=} \PY{p}{[}\PY{p}{]}
        \PY{n}{iterates} \PY{o}{=} \PY{p}{[}\PY{n}{x0}\PY{p}{]}
        
    \PY{c+c1}{\PYZsh{}Work with decimals?}
    \PY{k}{if} \PY{n}{decimal}\PY{p}{:}
        \PY{n}{x0} \PY{o}{=} \PY{n}{Decimal}\PY{p}{(}\PY{n}{x0}\PY{p}{)}
    
    \PY{c+c1}{\PYZsh{}Iterate through max iterations}
    \PY{k}{for} \PY{n}{i} \PY{o+ow}{in} \PY{n+nb}{range}\PY{p}{(}\PY{n}{imax}\PY{p}{)}\PY{p}{:}
        \PY{n}{x\PYZus{}new} \PY{o}{=} \PY{n}{x0} \PY{o}{\PYZhy{}} \PY{n}{p}\PY{o}{*}\PY{n}{f}\PY{p}{(}\PY{n}{x0}\PY{p}{)}\PY{o}{/}\PY{n}{df}\PY{p}{(}\PY{n}{x0}\PY{p}{)} \PY{c+c1}{\PYZsh{}New root estimate}
        
        \PY{c+c1}{\PYZsh{}Calculate an error}
        \PY{k}{if} \PY{n}{error}\PY{o}{==}\PY{l+s+s1}{\PYZsq{}}\PY{l+s+s1}{rel}\PY{l+s+s1}{\PYZsq{}}\PY{p}{:} \PY{n}{er} \PY{o}{=} \PY{n}{np}\PY{o}{.}\PY{n}{abs}\PY{p}{(}\PY{p}{(}\PY{n}{x\PYZus{}new}\PY{o}{\PYZhy{}}\PY{n}{x0}\PY{p}{)}\PY{o}{/}\PY{n}{x\PYZus{}new}\PY{p}{)}
        \PY{k}{elif} \PY{n}{error}\PY{o}{==}\PY{l+s+s1}{\PYZsq{}}\PY{l+s+s1}{abs}\PY{l+s+s1}{\PYZsq{}}\PY{p}{:} \PY{n}{er} \PY{o}{=} \PY{n}{np}\PY{o}{.}\PY{n}{abs}\PY{p}{(}\PY{n}{x\PYZus{}new}\PY{o}{\PYZhy{}}\PY{n}{x0}\PY{p}{)}
        \PY{k}{else}\PY{p}{:} \PY{n+ne}{ValueError}\PY{p}{(}\PY{l+s+s1}{\PYZsq{}}\PY{l+s+s1}{Invalid error type, }\PY{l+s+se}{\PYZbs{}}
\PY{l+s+s1}{                         must be (rel)ative or (abs)olute}\PY{l+s+s1}{\PYZsq{}}\PY{p}{)}
        
        \PY{k}{if} \PY{n}{tab}\PY{p}{:}
            \PY{n}{E}\PY{o}{.}\PY{n}{append}\PY{p}{(}\PY{n}{er}\PY{p}{)}
            \PY{n}{iterates}\PY{o}{.}\PY{n}{append}\PY{p}{(}\PY{n}{x\PYZus{}new}\PY{p}{)}
        
        \PY{c+c1}{\PYZsh{}If successful return}
        \PY{k}{if} \PY{n}{er} \PY{o}{\PYZlt{}} \PY{n}{tol}\PY{p}{:}
            \PY{k}{if} \PY{n}{tab}\PY{p}{:}
                \PY{k}{return} \PY{n}{x\PYZus{}new}\PY{p}{,} \PY{n}{E}\PY{p}{,} \PY{n}{iterates}
            \PY{k}{return} \PY{n}{x\PYZus{}new}
        
        \PY{n}{x0} \PY{o}{=} \PY{n}{x\PYZus{}new}

    \PY{k}{raise} \PY{n+ne}{ValueError}\PY{p}{(}\PY{l+s+s1}{\PYZsq{}}\PY{l+s+s1}{Max iterations exceeded}\PY{l+s+s1}{\PYZsq{}}\PY{p}{)}
\end{Verbatim}
\end{tcolorbox}
\newpage
    \hypertarget{problem-1}{%
\subsection*{Problem 1}\label{problem-1}}

Determine which of the following iterations will converge to the fixed
point \(x_*\) provided \(x_0\) is sufficiently close. If it does
converge give the order, and if linear give the rate.

When considering these fixed point iterations we have to consider not
only the fixed point method, but also Newton's method. This is because
Newton's method is just a very specific version of fixed point
iteration.

    \hypertarget{a.}{%
\subsubsection*{a).}\label{a.}}

\(x_{n+1} = -16 + 6x_n + \frac{12}{x_n},\:x_*=2\)

    Letting \(g(x) = -16+6x+\frac{12}{x}\) we can see that \(g\) is
continuous around \(x_*=2\). We will look at the value of the derivative
at the fixed point.

\[ g'(x) = 6 - \frac{12}{x^2} \implies |g'(x_*)| = 3 > 1 \]

    Thus we \(\boxed{\text{cannot say}}\) that the iteration will converge
to the fixed point.

    \hypertarget{b.}{%
\subsubsection*{b).}\label{b.}}

\(x_{n+1} = \frac{2}{3}x_n + \frac{1}{x_n^2},\:x_*=3^{1/3}\)

    Letting \(g(x) = \frac{2}{3}x+\frac{1}{x^2}\) we can see that \(g\) is
continuous around \(x_*=3^{1/3}\).

\[ g'(x) = \frac{2}{3} - \frac{2}{x^3} \implies |g'(x_*)| = 0 \]

Ah, so this requires more investigation. Let us see if we can write this
in the form of Newton's method.

\[ x_{n+1} = \frac{2}{3}x_n + \frac{1}{x_n^2} = x_n - \frac{x_n}{3} + \frac{1}{x_n^2} = x_n + \frac{\frac{-x_n^3}{3}+1}{x^2} = x_n - \frac{f(x_n)}{f'(x_n)} \]

Where \(f(x) = \frac{x^3}{3}-1\) and \(f'(x) = x^2\)

    So this iteration is in fact Newton's method. Thus, because \(f(x_*)=0\)
and \(f'(x_*)\neq0\) we will have at least
\(\boxed{\text{quadratic convergence}}\).
\newpage
    \hypertarget{c.}{%
\subsubsection*{c).}\label{c.}}

\(x_{n+1} = \frac{12}{1+x_n},\:x_*=3\)

    Letting \(g(x) = \frac{12}{1+x}\) we can see that \(g\) is continuous
around the fixed point.

\[ g'(x) = -12(1+x)^{-2} \implies |g'(x_*)|=\frac{3}{4} \]

Thus we are guranteed \(\boxed{\text{linear convergance}}\) for the
iteration. The rate of this convergence is then given as
\(\boxed{\frac{3}{4}}\).
\newpage
    \hypertarget{problem-2}{%
\subsection*{Problem 2}\label{problem-2}}

Making some assumptions, we can model the temperature in the soil,
\(T(x,t)\), a distance \(x\) meters below the surface and \(t\) seconds
after a cold snap with the following equation:

\[ \frac{T(x,t)-T_s}{T_i-T_s}=\text{erf}\Big (\frac{x}{2\sqrt{\alpha t}}\Big) \]

\(T_s\) is the constant cold temperature after the snap, \(T_i\) is the
initial soil temperature, and \(\alpha\) is thermal conductivity
(m\^{}2/s). The error function (erf) is given as\ldots{}

\[ \text{erf}(t) = \frac{2}{\sqrt{\pi}}\int_0^te^{-s^2}ds \]

We will assume that \(T_i=20\), \(T_s=-15\), and
\(\alpha = 0.138\cdot10^{-16}\) -- with the temperatures all in degrees
celsius.

    \begin{tcolorbox}[breakable, size=fbox, boxrule=1pt, pad at break*=1mm,colback=cellbackground, colframe=cellborder]
\prompt{In}{incolor}{4}{\boxspacing}
\begin{Verbatim}[commandchars=\\\{\}]
\PY{n}{T\PYZus{}i} \PY{o}{=} \PY{l+m+mi}{20}
\PY{n}{T\PYZus{}s} \PY{o}{=} \PY{o}{\PYZhy{}}\PY{l+m+mi}{15}
\PY{n}{alpha} \PY{o}{=} \PY{l+m+mf}{0.138E\PYZhy{}6}

\PY{n}{eps} \PY{o}{=} \PY{l+m+mf}{1E\PYZhy{}13} \PY{c+c1}{\PYZsh{}Tolerance}
\end{Verbatim}
\end{tcolorbox}

    \hypertarget{a.}{%
\subsubsection*{a).}\label{a.}}

We will determine the depth of a water main so that it only freezes
after 60 days of exposure. To do this we recast as a root finding
problem in the following manner:

As we are interested in the x dimension of the temperature at a certain
time we will begin by isolating \(T(x,t)\), and we will also plug in our
temperature values.

\[ \implies T(x,t)=35\cdot\text{erf}\Big(\frac{x}{2\sqrt{\alpha t}}\Big)-15 \]

    We can see that whether this function is positive or negative, and thus
whether the ground is frozen or not, depends on the value of the error
function. The error function is non-negative and saturates to one for
non-negative arguments (i.e. \(\text{erf}(0)=0\) and
\(\text{erf}(\infty)=1\)). As \(x\) increases the function will move
towards one and as \(t\) increases it will move towards zero. Plugging
in our specified time (60 days) we can look for the root of \(T(x)=0\)
and be sure that for \(x\) greater than the root the ground will not be
frozen.

Converting our days to seconds we have\ldots{}

\[ 60d \cdot \frac{24h}{1d} \cdot \frac{60m}{1h} \cdot \frac{60s}{1m} = 60^3\cdot24 = 5.184\cdot10^6 = \bar{t} \]

Thus our root finding problem is for \(f(x)=T(x,\bar{t})=0\) giving us
our function and its derivative as follows.

\[ f(x)=T(x,\bar{t})=35\cdot\text{erf}\Big(\frac{x}{2\sqrt{\alpha\bar{t}}}\Big)-15 \]

\[ f'(x)=T'(x,\bar{t})=\frac{35}{\sqrt{\alpha\bar{t}\pi}}\text{exp}(-\Big(\frac{x}{2\sqrt{\alpha\bar{t}}}\Big)^2) \]

Which follows from the chain rule on the derivative of the error
function and the Fundamental Theorem of Calculus. Because the function
saturates quite fast we will look at our function on the interval
\([0,5]\) to ensure that \(f(5)>0\).

    \begin{tcolorbox}[breakable, size=fbox, boxrule=1pt, pad at break*=1mm,colback=cellbackground, colframe=cellborder]
\prompt{In}{incolor}{5}{\boxspacing}
\begin{Verbatim}[commandchars=\\\{\}]
\PY{n}{t\PYZus{}bar} \PY{o}{=} \PY{l+m+mf}{5.148E6}

\PY{k}{def} \PY{n+nf}{f}\PY{p}{(}\PY{n}{x}\PY{p}{,} \PY{n}{t}\PY{o}{=}\PY{n}{t\PYZus{}bar}\PY{p}{,} \PY{n}{a}\PY{o}{=}\PY{n}{alpha}\PY{p}{)}\PY{p}{:}
    \PY{k}{return} \PY{l+m+mi}{35}\PY{o}{*}\PY{n}{erf}\PY{p}{(}\PY{n}{x}\PY{o}{/}\PY{p}{(}\PY{l+m+mi}{2}\PY{o}{*}\PY{n}{np}\PY{o}{.}\PY{n}{sqrt}\PY{p}{(}\PY{n}{alpha}\PY{o}{*}\PY{n}{t}\PY{p}{)}\PY{p}{)}\PY{p}{)}\PY{o}{\PYZhy{}}\PY{l+m+mi}{15}

\PY{k}{def} \PY{n+nf}{df}\PY{p}{(}\PY{n}{x}\PY{p}{,} \PY{n}{t}\PY{o}{=}\PY{n}{t\PYZus{}bar}\PY{p}{,} \PY{n}{a}\PY{o}{=}\PY{n}{alpha}\PY{p}{)}\PY{p}{:}
    \PY{k}{return} \PY{p}{(}\PY{l+m+mi}{35}\PY{o}{/}\PY{p}{(}\PY{n}{np}\PY{o}{.}\PY{n}{sqrt}\PY{p}{(}\PY{n}{alpha}\PY{o}{*}\PY{n}{t}\PY{o}{*}\PY{n}{np}\PY{o}{.}\PY{n}{pi}\PY{p}{)}\PY{p}{)}\PY{p}{)} \PY{o}{*} \PYZbs{}
            \PY{n}{np}\PY{o}{.}\PY{n}{exp}\PY{p}{(}\PY{o}{\PYZhy{}}\PY{p}{(}\PY{n}{x}\PY{o}{/}\PY{p}{(}\PY{l+m+mi}{2}\PY{o}{*}\PY{n}{np}\PY{o}{.}\PY{n}{sqrt}\PY{p}{(}\PY{n}{alpha}\PY{o}{*}\PY{n}{t}\PY{p}{)}\PY{p}{)}\PY{p}{)}\PY{o}{*}\PY{o}{*}\PY{l+m+mi}{2}\PY{p}{)}
\end{Verbatim}
\end{tcolorbox}

    \begin{tcolorbox}[breakable, size=fbox, boxrule=1pt, pad at break*=1mm,colback=cellbackground, colframe=cellborder]
\prompt{In}{incolor}{6}{\boxspacing}
\begin{Verbatim}[commandchars=\\\{\}]
\PY{n}{x\PYZus{}bar} \PY{o}{=} \PY{l+m+mi}{5}
\PY{n}{dx} \PY{o}{=} \PY{l+m+mf}{0.01}
\PY{n}{x} \PY{o}{=} \PY{n}{np}\PY{o}{.}\PY{n}{arange}\PY{p}{(}\PY{l+m+mi}{0}\PY{p}{,}\PY{n}{x\PYZus{}bar}\PY{o}{+}\PY{n}{dx}\PY{p}{,}\PY{n}{dx}\PY{p}{)}

\PY{n}{fig}\PY{p}{,} \PY{n}{ax} \PY{o}{=} \PY{n}{plt}\PY{o}{.}\PY{n}{subplots}\PY{p}{(}\PY{l+m+mi}{1}\PY{p}{,}\PY{l+m+mi}{1}\PY{p}{,}\PY{n}{figsize}\PY{o}{=}\PY{p}{(}\PY{l+m+mi}{10}\PY{p}{,}\PY{l+m+mi}{10}\PY{p}{)}\PY{p}{)}

\PY{n}{ax}\PY{o}{.}\PY{n}{plot}\PY{p}{(}\PY{n}{x}\PY{p}{,} \PY{n}{f}\PY{p}{(}\PY{n}{x}\PY{p}{)}\PY{p}{)}
\PY{n}{ax}\PY{o}{.}\PY{n}{set\PYZus{}title}\PY{p}{(}\PY{l+s+s1}{\PYZsq{}}\PY{l+s+s1}{Temperature Function @ 60 Days}\PY{l+s+s1}{\PYZsq{}}\PY{p}{)}
\PY{n}{ax}\PY{o}{.}\PY{n}{set\PYZus{}xlabel}\PY{p}{(}\PY{l+s+s1}{\PYZsq{}}\PY{l+s+s1}{x}\PY{l+s+s1}{\PYZsq{}}\PY{p}{)}
\PY{n}{ax}\PY{o}{.}\PY{n}{set\PYZus{}ylabel}\PY{p}{(}\PY{l+s+sa}{r}\PY{l+s+s1}{\PYZsq{}}\PY{l+s+s1}{\PYZdl{}T(x,}\PY{l+s+s1}{\PYZbs{}}\PY{l+s+s1}{bar}\PY{l+s+si}{\PYZob{}t\PYZcb{}}\PY{l+s+s1}{)\PYZdl{}}\PY{l+s+s1}{\PYZsq{}}\PY{p}{)}\PY{p}{;}
\end{Verbatim}
\end{tcolorbox}

    \begin{center}
    \adjustimage{max size={0.9\linewidth}{0.9\paperheight}}{output_19_0.png}
    \end{center}
    { \hspace*{\fill} \\}
    
    \hypertarget{b.}{%
\subsubsection*{b).}\label{b.}}

Using the Bisection method we will solve our root finding problem
described above.

    \begin{tcolorbox}[breakable, size=fbox, boxrule=1pt, pad at break*=1mm,colback=cellbackground, colframe=cellborder]
\prompt{In}{incolor}{7}{\boxspacing}
\begin{Verbatim}[commandchars=\\\{\}]
\PY{n}{a0} \PY{o}{=} \PY{l+m+mi}{0}
\PY{n}{b0} \PY{o}{=} \PY{l+m+mi}{5}

\PY{n}{p\PYZus{}b} \PY{o}{=} \PY{n}{bisection}\PY{p}{(}\PY{n}{a0}\PY{p}{,} \PY{n}{b0}\PY{p}{,} \PY{n}{f}\PY{p}{,} \PY{n}{tol}\PY{o}{=}\PY{n}{eps}\PY{p}{)}
\PY{n+nb}{print}\PY{p}{(}\PY{l+s+s1}{\PYZsq{}}\PY{l+s+s1}{Root @ x=}\PY{l+s+si}{\PYZpc{}.4f}\PY{l+s+s1}{ meters}\PY{l+s+s1}{\PYZsq{}} \PY{o}{\PYZpc{}} \PY{n}{p\PYZus{}b}\PY{p}{)}
\end{Verbatim}
\end{tcolorbox}

    \begin{Verbatim}[commandchars=\\\{\}]
Root @ x=0.6746 meters
    \end{Verbatim}

    \hypertarget{c.}{%
\subsubsection*{c).}\label{c.}}

We will solve our root finding problem using Newton's method.

First, with our initial guess as \(x_0=0.01\)

    \begin{tcolorbox}[breakable, size=fbox, boxrule=1pt, pad at break*=1mm,colback=cellbackground, colframe=cellborder]
\prompt{In}{incolor}{8}{\boxspacing}
\begin{Verbatim}[commandchars=\\\{\}]
\PY{n}{x0} \PY{o}{=} \PY{l+m+mf}{0.01}

\PY{n}{p\PYZus{}n} \PY{o}{=} \PY{n}{newton}\PY{p}{(}\PY{n}{x0}\PY{p}{,} \PY{n}{f}\PY{p}{,} \PY{n}{df}\PY{p}{,} \PY{n}{tol}\PY{o}{=}\PY{n}{eps}\PY{p}{)}
\PY{n+nb}{print}\PY{p}{(}\PY{l+s+s1}{\PYZsq{}}\PY{l+s+s1}{Root @ x=}\PY{l+s+si}{\PYZpc{}.4f}\PY{l+s+s1}{ meters}\PY{l+s+s1}{\PYZsq{}} \PY{o}{\PYZpc{}} \PY{n}{p\PYZus{}n}\PY{p}{)}
\end{Verbatim}
\end{tcolorbox}

    \begin{Verbatim}[commandchars=\\\{\}]
Root @ x=0.6746 meters
    \end{Verbatim}

    Next, with our initial guess as \(x_0=\bar{x}=5\)

    \begin{tcolorbox}[breakable, size=fbox, boxrule=1pt, pad at break*=1mm,colback=cellbackground, colframe=cellborder]
\prompt{In}{incolor}{9}{\boxspacing}
\begin{Verbatim}[commandchars=\\\{\}]
\PY{n}{p\PYZus{}n} \PY{o}{=} \PY{n}{newton}\PY{p}{(}\PY{l+m+mi}{5}\PY{p}{,} \PY{n}{f}\PY{p}{,} \PY{n}{df}\PY{p}{,} \PY{n}{tol}\PY{o}{=}\PY{n}{eps}\PY{p}{)}
\PY{n+nb}{print}\PY{p}{(}\PY{l+s+s1}{\PYZsq{}}\PY{l+s+s1}{Root @ x=}\PY{l+s+si}{\PYZpc{}.4f}\PY{l+s+s1}{ meters}\PY{l+s+s1}{\PYZsq{}} \PY{o}{\PYZpc{}} \PY{n}{p\PYZus{}n}\PY{p}{)}
\end{Verbatim}
\end{tcolorbox}

    \begin{Verbatim}[commandchars=\\\{\}]
/home/rs-coop/anaconda3/lib/python3.7/site-packages/ipykernel\_launcher.py:87:
RuntimeWarning: divide by zero encountered in double\_scalars
/home/rs-coop/anaconda3/lib/python3.7/site-packages/ipykernel\_launcher.py:90:
RuntimeWarning: invalid value encountered in double\_scalars
/home/rs-coop/anaconda3/lib/python3.7/site-packages/ipykernel\_launcher.py:87:
RuntimeWarning: invalid value encountered in double\_scalars
    \end{Verbatim}

    \begin{Verbatim}[commandchars=\\\{\}]

        ---------------------------------------------------------------------------

        ValueError                                Traceback (most recent call last)

        <ipython-input-9-57d0cafbd60c> in <module>
    ----> 1 p\_n = newton(5, f, df, tol=eps)
          2 print('Root @ x=\%.4f meters' \% p\_n)


        <ipython-input-3-375968645729> in newton(x0, f, df, p, tol, imax, tab, error, decimal)
        105         x0 = x\_new
        106 
    --> 107     raise ValueError('Max iterations exceeded')
    

        ValueError: Max iterations exceeded

    \end{Verbatim}

    Using our initial guess as \(x_0=\bar{x}=5\) we can see in the above
error output that the maximum number of iterations are exceeded, and
that there are some issues with the numerical values. This is because
the derivative at the starting location is essentially zero. Even if the
guess had been slightly less than 5, if the derivative is too small then
the iterates explode and break the method.

Using both Bisection and Newton's method we determined that the water
main should be buried at least \(\boxed{0.6746\;\text{meters}}\) beneath
the surface of the ground.

My preferred method is Bisection. This is because it works just as well
as Newtons in terms of speed and accuracy (in practice here), doesn't
require computing a derivative, and won't break as long as you provide
mildly resonable guesses.
\newpage
    \hypertarget{problem-3}{%
\subsection*{Problem 3}\label{problem-3}}

Here we will consider applying Newton's method to a real cubic
polynomial.

    \hypertarget{a.}{%
\subsubsection*{a).}\label{a.}}

We will assume that the polynomial has three distinct roots
(\(x=\alpha,\; x=\beta,\;x=\gamma\)) and show the guess
\(x_0=\frac{1}{2}(\alpha+\beta)\) will converge to \(x=\gamma\) in a
single step.

    Because our function is assumed to have these three roots we can write
\(f(x) = (x-\alpha)(x-\beta)(x-\gamma)\). The Newton iteration equation
requires us to find the derivative of the function, so we will start
there.

\begin{align}
    &f'(x) = [(x-\alpha)+(x-\beta)](x-\gamma)+(x-\alpha)(x-\beta)\\
    &=(2x-\alpha-\beta)(x-\gamma)+(x-\alpha)(x-\beta)
\end{align}

    We are trying to show that \(\gamma = x_0 - \frac{f(x_0)}{f'(x_0)}\), so
we will plug in and see what happens.

\begin{align}
    &f(x_0) = \frac{1}{2}(\beta-\alpha)\frac{1}{2}(\alpha-\beta)\frac{1}{2}(\alpha+\beta-\gamma)\\
    &=\frac{-1}{8}(\alpha-\beta)^2(\alpha+\beta-2\gamma)\\
    &~\\
    &f'(x_0) = 0 - \frac{1}{4}(\alpha-\beta)^2
\end{align}

    Putting this all together:

\[ \frac{1}{2}(\alpha+\beta)-\frac{\frac{-1}{8}(\alpha-\beta)^2(\alpha+\beta-2\gamma)}{- \frac{1}{4}(\alpha-\beta)^2} = \frac{1}{2}(\alpha+\beta) - \frac{1}{2}(\alpha+\beta-2\gamma) = \gamma \]

    Thus we see that this initial guess will yield the root \(\gamma\) in a
single step.

    \hypertarget{b.}{%
\subsubsection*{b).}\label{b.}}

If we assume that two roots coincide (i.e.~are equal) with the third
distinct, then we can show that there is exactly one starting guess that
is not the root iteself for which Newton's method will fail.

    Let us plot an example of this just to get an idea of what we are
looking at. We will use the function \(f(x)=(x-2)^2(x-5)\).

    \begin{tcolorbox}[breakable, size=fbox, boxrule=1pt, pad at break*=1mm,colback=cellbackground, colframe=cellborder]
\prompt{In}{incolor}{10}{\boxspacing}
\begin{Verbatim}[commandchars=\\\{\}]
\PY{n}{x} \PY{o}{=} \PY{n}{np}\PY{o}{.}\PY{n}{arange}\PY{p}{(}\PY{o}{\PYZhy{}}\PY{l+m+mi}{1}\PY{p}{,}\PY{l+m+mi}{6}\PY{o}{+}\PY{l+m+mf}{0.001}\PY{p}{,}\PY{l+m+mf}{0.001}\PY{p}{)}
\PY{n}{fig}\PY{p}{,} \PY{n}{ax} \PY{o}{=} \PY{n}{plt}\PY{o}{.}\PY{n}{subplots}\PY{p}{(}\PY{l+m+mi}{1}\PY{p}{,}\PY{l+m+mi}{1}\PY{p}{,}\PY{n}{figsize}\PY{o}{=}\PY{p}{(}\PY{l+m+mi}{10}\PY{p}{,}\PY{l+m+mi}{10}\PY{p}{)}\PY{p}{)}
\PY{n}{ax}\PY{o}{.}\PY{n}{plot}\PY{p}{(}\PY{n}{x}\PY{p}{,} \PY{p}{(}\PY{n}{x}\PY{o}{\PYZhy{}}\PY{l+m+mi}{5}\PY{p}{)}\PY{o}{*}\PY{p}{(}\PY{n}{x}\PY{o}{\PYZhy{}}\PY{l+m+mi}{2}\PY{p}{)}\PY{o}{*}\PY{o}{*}\PY{l+m+mi}{2}\PY{p}{)}\PY{p}{;}
\end{Verbatim}
\end{tcolorbox}

    \begin{center}
    \adjustimage{max size={0.9\linewidth}{0.9\paperheight}}{output_35_0.png}
    \end{center}
    { \hspace*{\fill} \\}
    
    Any cubic polynomial will have a shape roughly similar to this. Maybe
the function will be odd in the opposite manner (i.e.~positive at
\(-\infty\)), or have higher peaks and steeper valleys, but the general
shape remains. This eliminates any risk of the function shooting off to
infinity. It may take a while for some guesses to converge, but in the
end they will. Our only problems will occur when the derivative is zero
at an iterate. This can happen at the double root and at one other
place. Specifically, if the double root is \(x=r\) and the other root is
\(x=\alpha\), then \(f'(\frac{2\alpha+r}{3})=0\).

    The important thing to note is that the only way to get to this point
(which we will call \(x_0\)) is by having it as your initial guess. If
the initial guess is anywhere else then no iteration will ever result in
\(x_0\) where the derivative is zero, and thus the method will converge
to a root. Geometrically this makes sense because the only place on a
cubic polynomial where the tangent line would intersect the x-axis at
\(x_0\) is near the other peak. In this case the other peak is also a
root and so this cannot happen because the peak is not above the x-axis.

Looking at our example above, we can see that no matter where we draw a
tangent line, it will not intersect the x-axis where the derivative is
zero.

Thus we can conclude that the only other initial guess that results in
Newton's method failing is the point exactly where the derivative is
zero (and not a root).

    \hypertarget{c.}{%
\subsubsection*{c).}\label{c.}}

Extending part b). we can show why there are infinitely many starting
guess for which Newton's method will fail when all three roots are
distinct.

Again we will plot an example function (\(f(x)=(x-2)(x-5)(x+1)\)) to
help aid our understanding.

    \begin{tcolorbox}[breakable, size=fbox, boxrule=1pt, pad at break*=1mm,colback=cellbackground, colframe=cellborder]
\prompt{In}{incolor}{17}{\boxspacing}
\begin{Verbatim}[commandchars=\\\{\}]
\PY{n}{x} \PY{o}{=} \PY{n}{np}\PY{o}{.}\PY{n}{arange}\PY{p}{(}\PY{o}{\PYZhy{}}\PY{l+m+mi}{1}\PY{p}{,}\PY{l+m+mi}{6}\PY{o}{+}\PY{l+m+mf}{0.001}\PY{p}{,}\PY{l+m+mf}{0.001}\PY{p}{)}
\PY{n}{fig}\PY{p}{,} \PY{n}{ax} \PY{o}{=} \PY{n}{plt}\PY{o}{.}\PY{n}{subplots}\PY{p}{(}\PY{l+m+mi}{1}\PY{p}{,}\PY{l+m+mi}{1}\PY{p}{,}\PY{n}{figsize}\PY{o}{=}\PY{p}{(}\PY{l+m+mi}{10}\PY{p}{,}\PY{l+m+mi}{10}\PY{p}{)}\PY{p}{)}
\PY{n}{ax}\PY{o}{.}\PY{n}{plot}\PY{p}{(}\PY{n}{x}\PY{p}{,} \PY{p}{(}\PY{n}{x}\PY{o}{\PYZhy{}}\PY{l+m+mi}{5}\PY{p}{)}\PY{o}{*}\PY{p}{(}\PY{n}{x}\PY{o}{\PYZhy{}}\PY{l+m+mi}{3}\PY{p}{)}\PY{o}{*}\PY{p}{(}\PY{n}{x}\PY{o}{+}\PY{l+m+mi}{1}\PY{p}{)}\PY{p}{)}\PY{p}{;}
\end{Verbatim}
\end{tcolorbox}

    \begin{center}
    \adjustimage{max size={0.9\linewidth}{0.9\paperheight}}{output_39_0.png}
    \end{center}
    { \hspace*{\fill} \\}
    
    In the previous part where one of the roots had multiplicity of two we
concluded that there was no way for Newton to fail unless you started at
a point where the derivative was zero. However, in this case where all
of the roots are distinct, it is now possible to reach a point where the
derivative is zero from a perfectly acceptable previous point. This is
because, geometrically, a tangent line from one of the peaks (around
where the derivative is zero) can now intersect the x-axis where the
other peak is exactly. Essentially, in one iteration you can go from
near one peak to the x location where the derivative is zero on the
adjacent peak. Furthermore, there are infinitely many ways to get here
even if some of the iterations take longer. Because one of the peaks is
no longer also a root it allows for iterations to reach these places of
a zero derivative and thus Newton can fail for infinitely many starting
points.
\newpage
    \hypertarget{problem-4}{%
\subsection*{Problem 4}\label{problem-4}}

We will assume \(f(x)=(x-x_*)^pq(x)\) with \(p\) a positive integer,
\(q\) twice continuously differentiable, and \(q(x_*)\neq0\). We note
that \(f'(x_*)=0\) and we will use the following notation:
\(x_k,\;f_k=f(x_k),\;e_k=|x_*-x_k|\).

    \hypertarget{a.}{%
\subsubsection*{a).}\label{a.}}

We will show that Newton's method converges linearly for \(f(x)\).

    We will begin by finding the derivative of our function \(f(x)\).

\[ f'(x) = p(x-x_*)^{p-1}q(x) + (x-x_*)^pq'(x) \]

Plugging this and the function into the Newton iteration equation we
have the following:

\begin{align}
    &x_{n+1} = x_n - \frac{(x_n-x_*)^pq(x_n)}{p(x_n-x_*)^{p-1}q(x_n) + (x_n-x_*)^pq'(x_n)}\\
    &~\\
    &\text{Subtracting the root }\implies x_{n+1}-x_* = x_n-x_* - \frac{(x_n-x_*)^pq(x_n)}{p(x_n-x_*)^{p-1}q(x_n) + (x_n-x_*)^pq'(x_n)}\\
    &~\\
    &\implies x_{n+1}-x_* = x_n-x_*(1-\frac{(x_n-x_*)^{p-1}q(x_n)}{p(x_n-x_*)^{p-1}q(x_n) + (x_n-x_*)^pq'(x_n)})\\
    &~\\
    &\implies \frac{x_{n+1}-x_*}{x_n-x_*} = \frac{(p-1)q(x_n)+(x_n-x_*)q'(x_n)}{pq(x_n)+(x_n-x_*)q'(x_n)}\\
    &~\\
    &\text{Apply absolute values, take the limit, and assume the sequence converges...}\\
    &\lim_{n\to\infty} \frac{|x_{n+1}-x_*|}{|x_n-x_*|} = \boxed{\lim_{n\to\infty} \frac{e_{k+1}}{e_k} = \frac{p-1}{p}}
\end{align}

Where the last lines follows because \(x_n\to x_*\) by assumption. We
can see that we have linear convergence in the case where we have a
higher multiplicity of roots,

    \hypertarget{b.}{%
\subsubsection*{b).}\label{b.}}

The modified Newton's method with a rate parameter is given as follows:

\[ x_{k+1} = x_k - p\frac{f_k}{f'_k} \]

We will show that if \(x_k\to x_*\) then the rate of convergence is
quadratic (i.e. \(|e_{k+1}|\leq C|e_k|^2\) where \(C\) is constant).

    The process is quite similar to that shown above just with an extra
multiple of p somewhere inside. Specifically, instead of the \((p-1)\)
term we see in the line right before we take the limit, we will have a
\((p-p)=0\) term. This gives us the following:

\begin{align}
    &\frac{x_{n+1}-x_*}{x_n-x_*} = \frac{(x_n-x_*)q'(x_n)}{pq(x_n)+(x_n-x_*)q'(x_n)}\\
    &~\\
    &\implies \frac{x_{n+1}-x_*}{(x_n-x_*)^2} = \frac{q'(x_n)}{pq(x_n)+(x_n-x_*)q'(x_n)}\\
    &\text{Apply absolute values, take the limit, and assume the sequence converges...}\\
    &\lim_{n\to\infty} \frac{|x_{n+1}-x_*|}{|x_n-x_*|^2} = \boxed{\lim_{n\to\infty} \frac{e_{k+1}}{e_k^2} = \frac{q'(x_*)}{q(x_*)p}}
\end{align}

    Again, the last line follows because we are assuming that
\(x_n\to x_*\). Further, we may note that by assumption \(q(x_*)\neq0\)
and \(q'(x_*)\) exists. Thus we can let \(C=\frac{q'(x_*)}{q(x_*)p}\),
and we have shown that the rate of convergence is quadratic.

    \hypertarget{c.}{%
\subsubsection*{c).}\label{c.}}

Note the code for Newton's and Modified Newton's method are given at the
beginning of this work.

We will apply these methods to the following function:

\[ f(x) = (x-1)^5e^x \]

    \begin{tcolorbox}[breakable, size=fbox, boxrule=1pt, pad at break*=1mm,colback=cellbackground, colframe=cellborder]
\prompt{In}{incolor}{10}{\boxspacing}
\begin{Verbatim}[commandchars=\\\{\}]
\PY{n}{x0} \PY{o}{=} \PY{l+m+mi}{0} \PY{c+c1}{\PYZsh{}Initial guess}
\PY{n}{tol} \PY{o}{=} \PY{l+m+mf}{1E\PYZhy{}15} \PY{c+c1}{\PYZsh{}Relative error tolerance}

\PY{k}{def} \PY{n+nf}{f}\PY{p}{(}\PY{n}{x}\PY{p}{)}\PY{p}{:}
    \PY{k}{return} \PY{p}{(}\PY{p}{(}\PY{n}{x}\PY{o}{\PYZhy{}}\PY{l+m+mi}{1}\PY{p}{)}\PY{o}{*}\PY{o}{*}\PY{l+m+mi}{5}\PY{p}{)}\PY{o}{*}\PY{n}{np}\PY{o}{.}\PY{n}{exp}\PY{p}{(}\PY{n}{x}\PY{p}{)}

\PY{k}{def} \PY{n+nf}{df}\PY{p}{(}\PY{n}{x}\PY{p}{)}\PY{p}{:}
    \PY{k}{return} \PY{p}{(}\PY{n}{x}\PY{o}{+}\PY{l+m+mi}{4}\PY{p}{)}\PY{o}{*}\PY{n}{np}\PY{o}{.}\PY{n}{exp}\PY{p}{(}\PY{n}{x}\PY{p}{)}\PY{o}{*}\PY{p}{(}\PY{n}{x}\PY{o}{\PYZhy{}}\PY{l+m+mi}{1}\PY{p}{)}\PY{o}{*}\PY{o}{*}\PY{l+m+mi}{4}
\end{Verbatim}
\end{tcolorbox}

    \begin{tcolorbox}[breakable, size=fbox, boxrule=1pt, pad at break*=1mm,colback=cellbackground, colframe=cellborder]
\prompt{In}{incolor}{11}{\boxspacing}
\begin{Verbatim}[commandchars=\\\{\}]
\PY{n}{r1}\PY{p}{,} \PY{n}{\PYZus{}}\PY{p}{,} \PY{n}{i1} \PY{o}{=} \PY{n}{newton}\PY{p}{(}\PY{n}{x0}\PY{p}{,} \PY{n}{f}\PY{p}{,} \PY{n}{df}\PY{p}{,} \PY{n}{tol}\PY{o}{=}\PY{n}{tol}\PY{p}{,} \PY{n}{tab}\PY{o}{=}\PY{k+kc}{True}\PY{p}{,} \PY{n}{decimal}\PY{o}{=}\PY{k+kc}{True}\PY{p}{)}
\PY{n}{i1} \PY{o}{=} \PY{n}{np}\PY{o}{.}\PY{n}{array}\PY{p}{(}\PY{n}{i1}\PY{p}{,} \PY{n}{dtype}\PY{o}{=}\PY{l+s+s1}{\PYZsq{}}\PY{l+s+s1}{float64}\PY{l+s+s1}{\PYZsq{}}\PY{p}{)}

\PY{n}{r2}\PY{p}{,} \PY{n}{\PYZus{}}\PY{p}{,} \PY{n}{i2} \PY{o}{=} \PY{n}{newton}\PY{p}{(}\PY{n}{x0}\PY{p}{,} \PY{n}{f}\PY{p}{,} \PY{n}{df}\PY{p}{,} \PY{n}{p}\PY{o}{=}\PY{l+m+mi}{5}\PY{p}{,} \PY{n}{tol}\PY{o}{=}\PY{n}{tol}\PY{p}{,} \PY{n}{tab}\PY{o}{=}\PY{k+kc}{True}\PY{p}{,} \PY{n}{decimal}\PY{o}{=}\PY{k+kc}{True}\PY{p}{)}
\PY{n}{i2} \PY{o}{=} \PY{n}{np}\PY{o}{.}\PY{n}{array}\PY{p}{(}\PY{n}{i2}\PY{p}{,} \PY{n}{dtype}\PY{o}{=}\PY{l+s+s1}{\PYZsq{}}\PY{l+s+s1}{float64}\PY{l+s+s1}{\PYZsq{}}\PY{p}{)}
\end{Verbatim}
\end{tcolorbox}

    \begin{tcolorbox}[breakable, size=fbox, boxrule=1pt, pad at break*=1mm,colback=cellbackground, colframe=cellborder]
\prompt{In}{incolor}{12}{\boxspacing}
\begin{Verbatim}[commandchars=\\\{\}]
\PY{n}{pd}\PY{o}{.}\PY{n}{options}\PY{o}{.}\PY{n}{display}\PY{o}{.}\PY{n}{float\PYZus{}format} \PY{o}{=} \PY{l+s+s1}{\PYZsq{}}\PY{l+s+si}{\PYZob{}:.15E\PYZcb{}}\PY{l+s+s1}{\PYZsq{}}\PY{o}{.}\PY{n}{format}

\PY{n}{df\PYZus{}1} \PY{o}{=} \PY{n}{pd}\PY{o}{.}\PY{n}{DataFrame}\PY{p}{(}\PY{p}{\PYZob{}}\PY{l+s+s1}{\PYZsq{}}\PY{l+s+s1}{Iteration}\PY{l+s+s1}{\PYZsq{}}\PY{p}{:} \PY{p}{[}\PY{n}{i} \PY{k}{for} \PY{n}{i} \PY{o+ow}{in} \PY{n+nb}{range}\PY{p}{(}\PY{n+nb}{len}\PY{p}{(}\PY{n}{i1}\PY{p}{)}\PY{p}{)}\PY{p}{]}\PY{p}{,}
                     \PY{l+s+s1}{\PYZsq{}}\PY{l+s+s1}{Iterate}\PY{l+s+s1}{\PYZsq{}}\PY{p}{:} \PY{n}{i1}\PY{p}{,}
                     \PY{l+s+s1}{\PYZsq{}}\PY{l+s+s1}{Error}\PY{l+s+s1}{\PYZsq{}}\PY{p}{:} \PY{n}{np}\PY{o}{.}\PY{n}{abs}\PY{p}{(}\PY{n}{i1}\PY{o}{\PYZhy{}}\PY{l+m+mi}{1}\PY{p}{)}\PY{p}{\PYZcb{}}\PY{p}{)}\PY{o}{.}\PY{n}{set\PYZus{}index}\PY{p}{(}\PY{l+s+s1}{\PYZsq{}}\PY{l+s+s1}{Iteration}\PY{l+s+s1}{\PYZsq{}}\PY{p}{)}

\PY{n}{df\PYZus{}2} \PY{o}{=} \PY{n}{pd}\PY{o}{.}\PY{n}{DataFrame}\PY{p}{(}\PY{p}{\PYZob{}}\PY{l+s+s1}{\PYZsq{}}\PY{l+s+s1}{Iteration}\PY{l+s+s1}{\PYZsq{}}\PY{p}{:} \PY{p}{[}\PY{n}{i} \PY{k}{for} \PY{n}{i} \PY{o+ow}{in} \PY{n+nb}{range}\PY{p}{(}\PY{n+nb}{len}\PY{p}{(}\PY{n}{i2}\PY{p}{)}\PY{p}{)}\PY{p}{]}\PY{p}{,}
                     \PY{l+s+s1}{\PYZsq{}}\PY{l+s+s1}{Iterate}\PY{l+s+s1}{\PYZsq{}}\PY{p}{:} \PY{n}{i2}\PY{p}{,}
                     \PY{l+s+s1}{\PYZsq{}}\PY{l+s+s1}{Error}\PY{l+s+s1}{\PYZsq{}}\PY{p}{:} \PY{n}{np}\PY{o}{.}\PY{n}{abs}\PY{p}{(}\PY{n}{i2}\PY{o}{\PYZhy{}}\PY{l+m+mi}{1}\PY{p}{)}\PY{p}{\PYZcb{}}\PY{p}{)}\PY{o}{.}\PY{n}{set\PYZus{}index}\PY{p}{(}\PY{l+s+s1}{\PYZsq{}}\PY{l+s+s1}{Iteration}\PY{l+s+s1}{\PYZsq{}}\PY{p}{)}
\end{Verbatim}
\end{tcolorbox}

    With traditional Newton's method.

    \begin{tcolorbox}[breakable, size=fbox, boxrule=1pt, pad at break*=1mm,colback=cellbackground, colframe=cellborder]
\prompt{In}{incolor}{13}{\boxspacing}
\begin{Verbatim}[commandchars=\\\{\}]
\PY{n}{display}\PY{p}{(}\PY{n}{pd}\PY{o}{.}\PY{n}{concat}\PY{p}{(}\PY{p}{[}\PY{n}{df\PYZus{}1}\PY{o}{.}\PY{n}{iloc}\PY{p}{[}\PY{p}{:}\PY{p}{:}\PY{l+m+mi}{5}\PY{p}{,}\PY{p}{:}\PY{p}{]}\PY{p}{,} \PY{n}{df\PYZus{}1}\PY{o}{.}\PY{n}{iloc}\PY{p}{[}\PY{o}{\PYZhy{}}\PY{l+m+mi}{2}\PY{p}{:}\PY{p}{,}\PY{p}{:}\PY{p}{]}\PY{p}{]}\PY{p}{)}\PY{p}{)}
\end{Verbatim}
\end{tcolorbox}

    
    \begin{verbatim}
                        Iterate                 Error
Iteration                                            
0         0.000000000000000E+00 1.000000000000000E+00
5         7.279472249514199E-01 2.720527750485801E-01
10        9.149462812517818E-01 8.505371874821821E-02
15        9.725285627331187E-01 2.747143726688128E-02
20        9.910397429641807E-01 8.960257035819263E-03
25        9.970683254984478E-01 2.931674501552184E-03
30        9.990398222884037E-01 9.601777115962884E-04
35        9.996854197455141E-01 3.145802544859411E-04
40        9.998969237926445E-01 1.030762073555014E-04
45        9.999662245735458E-01 3.377542645421894E-05
50        9.999889325310896E-01 1.106746891044175E-05
55        9.999963734185336E-01 3.626581466353862E-06
60        9.999988116425095E-01 1.188357490522485E-06
65        9.999996105990953E-01 3.894009047433755E-07
70        9.999998724011199E-01 1.275988801285877E-07
75        9.999999581883998E-01 4.181160018212893E-08
80        9.999999862991750E-01 1.370082503004966E-08
85        9.999999955105137E-01 4.489486338243864E-09
90        9.999999985288851E-01 1.471114896567371E-09
95        9.999999995179450E-01 4.820549515116568E-10
100       9.999999998420402E-01 1.579597563861057E-10
105       9.999999999482397E-01 5.176026274256174E-11
110       9.999999999830392E-01 1.696076612489605E-11
115       9.999999999944423E-01 5.557665438971071E-12
120       9.999999999981789E-01 1.821098827292644E-12
125       9.999999999994033E-01 5.967448757360216E-13
130       9.999999999998045E-01 1.955102746364901E-13
135       9.999999999999359E-01 6.405986852087153E-14
140       9.999999999999790E-01 2.098321516541546E-14
145       9.999999999999931E-01 6.883382752675971E-15
147       9.999999999999956E-01 4.440892098500626E-15
148       9.999999999999964E-01 3.552713678800501E-15
    \end{verbatim}

    
    With Modified Newtons

    \begin{tcolorbox}[breakable, size=fbox, boxrule=1pt, pad at break*=1mm,colback=cellbackground, colframe=cellborder]
\prompt{In}{incolor}{14}{\boxspacing}
\begin{Verbatim}[commandchars=\\\{\}]
\PY{n}{display}\PY{p}{(}\PY{n}{df\PYZus{}2}\PY{p}{)}
\end{Verbatim}
\end{tcolorbox}

    
    \begin{verbatim}
                        Iterate                 Error
Iteration                                            
0         0.000000000000000E+00 1.000000000000000E+00
1         1.250000000000000E+00 2.500000000000000E-01
2         1.011904761904762E+00 1.190476190476186E-02
3         1.000028277344192E+00 2.827734419175165E-05
4         1.000000000159921E+00 1.599207433145011E-10
5         1.000000000000000E+00 0.000000000000000E+00
6         1.000000000000000E+00 0.000000000000000E+00
    \end{verbatim}

    
    \begin{tcolorbox}[breakable, size=fbox, boxrule=1pt, pad at break*=1mm,colback=cellbackground, colframe=cellborder]
\prompt{In}{incolor}{18}{\boxspacing}
\begin{Verbatim}[commandchars=\\\{\}]
\PY{n}{fig}\PY{p}{,} \PY{n}{ax} \PY{o}{=} \PY{n}{plt}\PY{o}{.}\PY{n}{subplots}\PY{p}{(}\PY{l+m+mi}{1}\PY{p}{,}\PY{l+m+mi}{1}\PY{p}{,} \PY{n}{figsize}\PY{o}{=}\PY{p}{(}\PY{l+m+mi}{10}\PY{p}{,}\PY{l+m+mi}{10}\PY{p}{)}\PY{p}{)}

\PY{n}{ax}\PY{o}{.}\PY{n}{scatter}\PY{p}{(}\PY{n}{df\PYZus{}1}\PY{o}{.}\PY{n}{index}\PY{p}{,} \PY{n}{df\PYZus{}1}\PY{o}{.}\PY{n}{Error}\PY{p}{)}
\PY{n}{ax}\PY{o}{.}\PY{n}{scatter}\PY{p}{(}\PY{n}{df\PYZus{}2}\PY{o}{.}\PY{n}{index}\PY{p}{,} \PY{n}{df\PYZus{}2}\PY{o}{.}\PY{n}{Error}\PY{p}{)}
\PY{n}{ax}\PY{o}{.}\PY{n}{set\PYZus{}title}\PY{p}{(}\PY{l+s+s1}{\PYZsq{}}\PY{l+s+s1}{Error Per Iteration}\PY{l+s+s1}{\PYZsq{}}\PY{p}{)}
\PY{n}{ax}\PY{o}{.}\PY{n}{set\PYZus{}xlabel}\PY{p}{(}\PY{l+s+s1}{\PYZsq{}}\PY{l+s+s1}{Iteration \PYZsh{}}\PY{l+s+s1}{\PYZsq{}}\PY{p}{)}
\PY{n}{ax}\PY{o}{.}\PY{n}{set\PYZus{}ylabel}\PY{p}{(}\PY{l+s+s1}{\PYZsq{}}\PY{l+s+s1}{Error}\PY{l+s+s1}{\PYZsq{}}\PY{p}{)}
\PY{n}{ax}\PY{o}{.}\PY{n}{legend}\PY{p}{(}\PY{p}{[}\PY{l+s+s1}{\PYZsq{}}\PY{l+s+s1}{Newton}\PY{l+s+s1}{\PYZsq{}}\PY{p}{,} \PY{l+s+s1}{\PYZsq{}}\PY{l+s+s1}{Modified Newton}\PY{l+s+s1}{\PYZsq{}}\PY{p}{]}\PY{p}{)}\PY{p}{;}
\end{Verbatim}
\end{tcolorbox}

    \begin{center}
    \adjustimage{max size={0.9\linewidth}{0.9\paperheight}}{output_55_0.png}
    \end{center}
    { \hspace*{\fill} \\}
    
    In the previous part of this problem we showed that without the rate
parameter in the Modfied Newton's method the convergence would be linear
due to the multiplicity of the root. We can clearly see that in action
here, both in the plot above and in the tables. For the standard
Newton's method it took almost 150 iterations to converge on the root to
our desired accuracy. The modified method achieved this same goal in
only 6 iterations. This is what we would expect given our theory
development (that the modified would be faster), although we do see some
variation in the specifics. In particular we see that the standard
Newton's method is performing potentially better than expected, but
still poorly.
\newpage
    \hypertarget{problem-5}{%
\subsection*{Problem 5}\label{problem-5}}

Letting \(x_0,\;x_1\) be two succesive points from a secant iteration
scheme applied to the equation \(f(x)=0\) with
\(f_0=f(x_0),\;f_1=f(x_1)\). We will show that the order of the points
(i.e.~which one is the most ``recent'') has no effect on value of the
next point.

    In the Secant method the recurrence equation is as follows:

\[ x_{n+1} = x_n - \frac{f(x_n)(x_n-x_{n-1})}{f(x_n)-f(x_{n-1})} \]

    We will show that swapping \(x_n\) and \(x_{n-1}\) has no effect on
\(x_{n+1}\).

\begin{align}
    &\text{Let } y_1 = x_1 - \frac{f_1(x_1-x_0)}{f_1-f_0}\\
    &\text{Let } y_0 = x_0 - \frac{f_0(x_0-x_1)}{f_0-f_1}\\
    &y_1-y_0 = x_1-x_0 - \frac{f_1(x_1-x_0)}{f_1-f_0} + \frac{f_0(x_0-x_1)}{f_0-f_1}\\
    &\implies y_1-y_0 = x_1-x_0 + \frac{(f_0-f_1)(x_0-x_1)}{f_0-f_1} = 0
\end{align}

    Thus we conclude that \(y_0=y_1\) and are the same point no matter which
of the previous points we consider the most recent.


    % Add a bibliography block to the postdoc
    
    
    
\end{document}
