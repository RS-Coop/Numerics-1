\documentclass[11pt]{article}

    \usepackage[breakable]{tcolorbox}
    \usepackage{parskip} % Stop auto-indenting (to mimic markdown behaviour)
    
    \usepackage{iftex}
    \ifPDFTeX
    	\usepackage[T1]{fontenc}
    	\usepackage{mathpazo}
    \else
    	\usepackage{fontspec}
    \fi

    % Basic figure setup, for now with no caption control since it's done
    % automatically by Pandoc (which extracts ![](path) syntax from Markdown).
    \usepackage{graphicx}
    % Maintain compatibility with old templates. Remove in nbconvert 6.0
    \let\Oldincludegraphics\includegraphics
    % Ensure that by default, figures have no caption (until we provide a
    % proper Figure object with a Caption API and a way to capture that
    % in the conversion process - todo).
    \usepackage{caption}
    \DeclareCaptionFormat{nocaption}{}
    \captionsetup{format=nocaption,aboveskip=0pt,belowskip=0pt}

    \usepackage[Export]{adjustbox} % Used to constrain images to a maximum size
    \adjustboxset{max size={0.9\linewidth}{0.9\paperheight}}
    \usepackage{float}
    \floatplacement{figure}{H} % forces figures to be placed at the correct location
    \usepackage{xcolor} % Allow colors to be defined
    \usepackage{enumerate} % Needed for markdown enumerations to work
    \usepackage{geometry} % Used to adjust the document margins
    \usepackage{amsmath} % Equations
    \usepackage{amssymb} % Equations
    \usepackage{textcomp} % defines textquotesingle
    % Hack from http://tex.stackexchange.com/a/47451/13684:
    \AtBeginDocument{%
        \def\PYZsq{\textquotesingle}% Upright quotes in Pygmentized code
    }
    \usepackage{upquote} % Upright quotes for verbatim code
    \usepackage{eurosym} % defines \euro
    \usepackage[mathletters]{ucs} % Extended unicode (utf-8) support
    \usepackage{fancyvrb} % verbatim replacement that allows latex
    \usepackage{grffile} % extends the file name processing of package graphics 
                         % to support a larger range
    \makeatletter % fix for grffile with XeLaTeX
    \def\Gread@@xetex#1{%
      \IfFileExists{"\Gin@base".bb}%
      {\Gread@eps{\Gin@base.bb}}%
      {\Gread@@xetex@aux#1}%
    }
    \makeatother

    % The hyperref package gives us a pdf with properly built
    % internal navigation ('pdf bookmarks' for the table of contents,
    % internal cross-reference links, web links for URLs, etc.)
    \usepackage{hyperref}
    % The default LaTeX title has an obnoxious amount of whitespace. By default,
    % titling removes some of it. It also provides customization options.
    \usepackage{titling}
    \usepackage{longtable} % longtable support required by pandoc >1.10
    \usepackage{booktabs}  % table support for pandoc > 1.12.2
    \usepackage[inline]{enumitem} % IRkernel/repr support (it uses the enumerate* environment)
    \usepackage[normalem]{ulem} % ulem is needed to support strikethroughs (\sout)
                                % normalem makes italics be italics, not underlines
    \usepackage{mathrsfs}
    

    
    % Colors for the hyperref package
    \definecolor{urlcolor}{rgb}{0,.145,.698}
    \definecolor{linkcolor}{rgb}{.71,0.21,0.01}
    \definecolor{citecolor}{rgb}{.12,.54,.11}

    % ANSI colors
    \definecolor{ansi-black}{HTML}{3E424D}
    \definecolor{ansi-black-intense}{HTML}{282C36}
    \definecolor{ansi-red}{HTML}{E75C58}
    \definecolor{ansi-red-intense}{HTML}{B22B31}
    \definecolor{ansi-green}{HTML}{00A250}
    \definecolor{ansi-green-intense}{HTML}{007427}
    \definecolor{ansi-yellow}{HTML}{DDB62B}
    \definecolor{ansi-yellow-intense}{HTML}{B27D12}
    \definecolor{ansi-blue}{HTML}{208FFB}
    \definecolor{ansi-blue-intense}{HTML}{0065CA}
    \definecolor{ansi-magenta}{HTML}{D160C4}
    \definecolor{ansi-magenta-intense}{HTML}{A03196}
    \definecolor{ansi-cyan}{HTML}{60C6C8}
    \definecolor{ansi-cyan-intense}{HTML}{258F8F}
    \definecolor{ansi-white}{HTML}{C5C1B4}
    \definecolor{ansi-white-intense}{HTML}{A1A6B2}
    \definecolor{ansi-default-inverse-fg}{HTML}{FFFFFF}
    \definecolor{ansi-default-inverse-bg}{HTML}{000000}

    % commands and environments needed by pandoc snippets
    % extracted from the output of `pandoc -s`
    \providecommand{\tightlist}{%
      \setlength{\itemsep}{0pt}\setlength{\parskip}{0pt}}
    \DefineVerbatimEnvironment{Highlighting}{Verbatim}{commandchars=\\\{\}}
    % Add ',fontsize=\small' for more characters per line
    \newenvironment{Shaded}{}{}
    \newcommand{\KeywordTok}[1]{\textcolor[rgb]{0.00,0.44,0.13}{\textbf{{#1}}}}
    \newcommand{\DataTypeTok}[1]{\textcolor[rgb]{0.56,0.13,0.00}{{#1}}}
    \newcommand{\DecValTok}[1]{\textcolor[rgb]{0.25,0.63,0.44}{{#1}}}
    \newcommand{\BaseNTok}[1]{\textcolor[rgb]{0.25,0.63,0.44}{{#1}}}
    \newcommand{\FloatTok}[1]{\textcolor[rgb]{0.25,0.63,0.44}{{#1}}}
    \newcommand{\CharTok}[1]{\textcolor[rgb]{0.25,0.44,0.63}{{#1}}}
    \newcommand{\StringTok}[1]{\textcolor[rgb]{0.25,0.44,0.63}{{#1}}}
    \newcommand{\CommentTok}[1]{\textcolor[rgb]{0.38,0.63,0.69}{\textit{{#1}}}}
    \newcommand{\OtherTok}[1]{\textcolor[rgb]{0.00,0.44,0.13}{{#1}}}
    \newcommand{\AlertTok}[1]{\textcolor[rgb]{1.00,0.00,0.00}{\textbf{{#1}}}}
    \newcommand{\FunctionTok}[1]{\textcolor[rgb]{0.02,0.16,0.49}{{#1}}}
    \newcommand{\RegionMarkerTok}[1]{{#1}}
    \newcommand{\ErrorTok}[1]{\textcolor[rgb]{1.00,0.00,0.00}{\textbf{{#1}}}}
    \newcommand{\NormalTok}[1]{{#1}}
    
    % Additional commands for more recent versions of Pandoc
    \newcommand{\ConstantTok}[1]{\textcolor[rgb]{0.53,0.00,0.00}{{#1}}}
    \newcommand{\SpecialCharTok}[1]{\textcolor[rgb]{0.25,0.44,0.63}{{#1}}}
    \newcommand{\VerbatimStringTok}[1]{\textcolor[rgb]{0.25,0.44,0.63}{{#1}}}
    \newcommand{\SpecialStringTok}[1]{\textcolor[rgb]{0.73,0.40,0.53}{{#1}}}
    \newcommand{\ImportTok}[1]{{#1}}
    \newcommand{\DocumentationTok}[1]{\textcolor[rgb]{0.73,0.13,0.13}{\textit{{#1}}}}
    \newcommand{\AnnotationTok}[1]{\textcolor[rgb]{0.38,0.63,0.69}{\textbf{\textit{{#1}}}}}
    \newcommand{\CommentVarTok}[1]{\textcolor[rgb]{0.38,0.63,0.69}{\textbf{\textit{{#1}}}}}
    \newcommand{\VariableTok}[1]{\textcolor[rgb]{0.10,0.09,0.49}{{#1}}}
    \newcommand{\ControlFlowTok}[1]{\textcolor[rgb]{0.00,0.44,0.13}{\textbf{{#1}}}}
    \newcommand{\OperatorTok}[1]{\textcolor[rgb]{0.40,0.40,0.40}{{#1}}}
    \newcommand{\BuiltInTok}[1]{{#1}}
    \newcommand{\ExtensionTok}[1]{{#1}}
    \newcommand{\PreprocessorTok}[1]{\textcolor[rgb]{0.74,0.48,0.00}{{#1}}}
    \newcommand{\AttributeTok}[1]{\textcolor[rgb]{0.49,0.56,0.16}{{#1}}}
    \newcommand{\InformationTok}[1]{\textcolor[rgb]{0.38,0.63,0.69}{\textbf{\textit{{#1}}}}}
    \newcommand{\WarningTok}[1]{\textcolor[rgb]{0.38,0.63,0.69}{\textbf{\textit{{#1}}}}}
    
    
    % Define a nice break command that doesn't care if a line doesn't already
    % exist.
    \def\br{\hspace*{\fill} \\* }
    % Math Jax compatibility definitions
    \def\gt{>}
    \def\lt{<}
    \let\Oldtex\TeX
    \let\Oldlatex\LaTeX
    \renewcommand{\TeX}{\textrm{\Oldtex}}
    \renewcommand{\LaTeX}{\textrm{\Oldlatex}}
    % Document parameters
    % Document title
    \title{Numerics 1: HW 1}
    \author{Cooper Simpson}
    
    
    
    
    
% Pygments definitions
\makeatletter
\def\PY@reset{\let\PY@it=\relax \let\PY@bf=\relax%
    \let\PY@ul=\relax \let\PY@tc=\relax%
    \let\PY@bc=\relax \let\PY@ff=\relax}
\def\PY@tok#1{\csname PY@tok@#1\endcsname}
\def\PY@toks#1+{\ifx\relax#1\empty\else%
    \PY@tok{#1}\expandafter\PY@toks\fi}
\def\PY@do#1{\PY@bc{\PY@tc{\PY@ul{%
    \PY@it{\PY@bf{\PY@ff{#1}}}}}}}
\def\PY#1#2{\PY@reset\PY@toks#1+\relax+\PY@do{#2}}

\expandafter\def\csname PY@tok@w\endcsname{\def\PY@tc##1{\textcolor[rgb]{0.73,0.73,0.73}{##1}}}
\expandafter\def\csname PY@tok@c\endcsname{\let\PY@it=\textit\def\PY@tc##1{\textcolor[rgb]{0.25,0.50,0.50}{##1}}}
\expandafter\def\csname PY@tok@cp\endcsname{\def\PY@tc##1{\textcolor[rgb]{0.74,0.48,0.00}{##1}}}
\expandafter\def\csname PY@tok@k\endcsname{\let\PY@bf=\textbf\def\PY@tc##1{\textcolor[rgb]{0.00,0.50,0.00}{##1}}}
\expandafter\def\csname PY@tok@kp\endcsname{\def\PY@tc##1{\textcolor[rgb]{0.00,0.50,0.00}{##1}}}
\expandafter\def\csname PY@tok@kt\endcsname{\def\PY@tc##1{\textcolor[rgb]{0.69,0.00,0.25}{##1}}}
\expandafter\def\csname PY@tok@o\endcsname{\def\PY@tc##1{\textcolor[rgb]{0.40,0.40,0.40}{##1}}}
\expandafter\def\csname PY@tok@ow\endcsname{\let\PY@bf=\textbf\def\PY@tc##1{\textcolor[rgb]{0.67,0.13,1.00}{##1}}}
\expandafter\def\csname PY@tok@nb\endcsname{\def\PY@tc##1{\textcolor[rgb]{0.00,0.50,0.00}{##1}}}
\expandafter\def\csname PY@tok@nf\endcsname{\def\PY@tc##1{\textcolor[rgb]{0.00,0.00,1.00}{##1}}}
\expandafter\def\csname PY@tok@nc\endcsname{\let\PY@bf=\textbf\def\PY@tc##1{\textcolor[rgb]{0.00,0.00,1.00}{##1}}}
\expandafter\def\csname PY@tok@nn\endcsname{\let\PY@bf=\textbf\def\PY@tc##1{\textcolor[rgb]{0.00,0.00,1.00}{##1}}}
\expandafter\def\csname PY@tok@ne\endcsname{\let\PY@bf=\textbf\def\PY@tc##1{\textcolor[rgb]{0.82,0.25,0.23}{##1}}}
\expandafter\def\csname PY@tok@nv\endcsname{\def\PY@tc##1{\textcolor[rgb]{0.10,0.09,0.49}{##1}}}
\expandafter\def\csname PY@tok@no\endcsname{\def\PY@tc##1{\textcolor[rgb]{0.53,0.00,0.00}{##1}}}
\expandafter\def\csname PY@tok@nl\endcsname{\def\PY@tc##1{\textcolor[rgb]{0.63,0.63,0.00}{##1}}}
\expandafter\def\csname PY@tok@ni\endcsname{\let\PY@bf=\textbf\def\PY@tc##1{\textcolor[rgb]{0.60,0.60,0.60}{##1}}}
\expandafter\def\csname PY@tok@na\endcsname{\def\PY@tc##1{\textcolor[rgb]{0.49,0.56,0.16}{##1}}}
\expandafter\def\csname PY@tok@nt\endcsname{\let\PY@bf=\textbf\def\PY@tc##1{\textcolor[rgb]{0.00,0.50,0.00}{##1}}}
\expandafter\def\csname PY@tok@nd\endcsname{\def\PY@tc##1{\textcolor[rgb]{0.67,0.13,1.00}{##1}}}
\expandafter\def\csname PY@tok@s\endcsname{\def\PY@tc##1{\textcolor[rgb]{0.73,0.13,0.13}{##1}}}
\expandafter\def\csname PY@tok@sd\endcsname{\let\PY@it=\textit\def\PY@tc##1{\textcolor[rgb]{0.73,0.13,0.13}{##1}}}
\expandafter\def\csname PY@tok@si\endcsname{\let\PY@bf=\textbf\def\PY@tc##1{\textcolor[rgb]{0.73,0.40,0.53}{##1}}}
\expandafter\def\csname PY@tok@se\endcsname{\let\PY@bf=\textbf\def\PY@tc##1{\textcolor[rgb]{0.73,0.40,0.13}{##1}}}
\expandafter\def\csname PY@tok@sr\endcsname{\def\PY@tc##1{\textcolor[rgb]{0.73,0.40,0.53}{##1}}}
\expandafter\def\csname PY@tok@ss\endcsname{\def\PY@tc##1{\textcolor[rgb]{0.10,0.09,0.49}{##1}}}
\expandafter\def\csname PY@tok@sx\endcsname{\def\PY@tc##1{\textcolor[rgb]{0.00,0.50,0.00}{##1}}}
\expandafter\def\csname PY@tok@m\endcsname{\def\PY@tc##1{\textcolor[rgb]{0.40,0.40,0.40}{##1}}}
\expandafter\def\csname PY@tok@gh\endcsname{\let\PY@bf=\textbf\def\PY@tc##1{\textcolor[rgb]{0.00,0.00,0.50}{##1}}}
\expandafter\def\csname PY@tok@gu\endcsname{\let\PY@bf=\textbf\def\PY@tc##1{\textcolor[rgb]{0.50,0.00,0.50}{##1}}}
\expandafter\def\csname PY@tok@gd\endcsname{\def\PY@tc##1{\textcolor[rgb]{0.63,0.00,0.00}{##1}}}
\expandafter\def\csname PY@tok@gi\endcsname{\def\PY@tc##1{\textcolor[rgb]{0.00,0.63,0.00}{##1}}}
\expandafter\def\csname PY@tok@gr\endcsname{\def\PY@tc##1{\textcolor[rgb]{1.00,0.00,0.00}{##1}}}
\expandafter\def\csname PY@tok@ge\endcsname{\let\PY@it=\textit}
\expandafter\def\csname PY@tok@gs\endcsname{\let\PY@bf=\textbf}
\expandafter\def\csname PY@tok@gp\endcsname{\let\PY@bf=\textbf\def\PY@tc##1{\textcolor[rgb]{0.00,0.00,0.50}{##1}}}
\expandafter\def\csname PY@tok@go\endcsname{\def\PY@tc##1{\textcolor[rgb]{0.53,0.53,0.53}{##1}}}
\expandafter\def\csname PY@tok@gt\endcsname{\def\PY@tc##1{\textcolor[rgb]{0.00,0.27,0.87}{##1}}}
\expandafter\def\csname PY@tok@err\endcsname{\def\PY@bc##1{\setlength{\fboxsep}{0pt}\fcolorbox[rgb]{1.00,0.00,0.00}{1,1,1}{\strut ##1}}}
\expandafter\def\csname PY@tok@kc\endcsname{\let\PY@bf=\textbf\def\PY@tc##1{\textcolor[rgb]{0.00,0.50,0.00}{##1}}}
\expandafter\def\csname PY@tok@kd\endcsname{\let\PY@bf=\textbf\def\PY@tc##1{\textcolor[rgb]{0.00,0.50,0.00}{##1}}}
\expandafter\def\csname PY@tok@kn\endcsname{\let\PY@bf=\textbf\def\PY@tc##1{\textcolor[rgb]{0.00,0.50,0.00}{##1}}}
\expandafter\def\csname PY@tok@kr\endcsname{\let\PY@bf=\textbf\def\PY@tc##1{\textcolor[rgb]{0.00,0.50,0.00}{##1}}}
\expandafter\def\csname PY@tok@bp\endcsname{\def\PY@tc##1{\textcolor[rgb]{0.00,0.50,0.00}{##1}}}
\expandafter\def\csname PY@tok@fm\endcsname{\def\PY@tc##1{\textcolor[rgb]{0.00,0.00,1.00}{##1}}}
\expandafter\def\csname PY@tok@vc\endcsname{\def\PY@tc##1{\textcolor[rgb]{0.10,0.09,0.49}{##1}}}
\expandafter\def\csname PY@tok@vg\endcsname{\def\PY@tc##1{\textcolor[rgb]{0.10,0.09,0.49}{##1}}}
\expandafter\def\csname PY@tok@vi\endcsname{\def\PY@tc##1{\textcolor[rgb]{0.10,0.09,0.49}{##1}}}
\expandafter\def\csname PY@tok@vm\endcsname{\def\PY@tc##1{\textcolor[rgb]{0.10,0.09,0.49}{##1}}}
\expandafter\def\csname PY@tok@sa\endcsname{\def\PY@tc##1{\textcolor[rgb]{0.73,0.13,0.13}{##1}}}
\expandafter\def\csname PY@tok@sb\endcsname{\def\PY@tc##1{\textcolor[rgb]{0.73,0.13,0.13}{##1}}}
\expandafter\def\csname PY@tok@sc\endcsname{\def\PY@tc##1{\textcolor[rgb]{0.73,0.13,0.13}{##1}}}
\expandafter\def\csname PY@tok@dl\endcsname{\def\PY@tc##1{\textcolor[rgb]{0.73,0.13,0.13}{##1}}}
\expandafter\def\csname PY@tok@s2\endcsname{\def\PY@tc##1{\textcolor[rgb]{0.73,0.13,0.13}{##1}}}
\expandafter\def\csname PY@tok@sh\endcsname{\def\PY@tc##1{\textcolor[rgb]{0.73,0.13,0.13}{##1}}}
\expandafter\def\csname PY@tok@s1\endcsname{\def\PY@tc##1{\textcolor[rgb]{0.73,0.13,0.13}{##1}}}
\expandafter\def\csname PY@tok@mb\endcsname{\def\PY@tc##1{\textcolor[rgb]{0.40,0.40,0.40}{##1}}}
\expandafter\def\csname PY@tok@mf\endcsname{\def\PY@tc##1{\textcolor[rgb]{0.40,0.40,0.40}{##1}}}
\expandafter\def\csname PY@tok@mh\endcsname{\def\PY@tc##1{\textcolor[rgb]{0.40,0.40,0.40}{##1}}}
\expandafter\def\csname PY@tok@mi\endcsname{\def\PY@tc##1{\textcolor[rgb]{0.40,0.40,0.40}{##1}}}
\expandafter\def\csname PY@tok@il\endcsname{\def\PY@tc##1{\textcolor[rgb]{0.40,0.40,0.40}{##1}}}
\expandafter\def\csname PY@tok@mo\endcsname{\def\PY@tc##1{\textcolor[rgb]{0.40,0.40,0.40}{##1}}}
\expandafter\def\csname PY@tok@ch\endcsname{\let\PY@it=\textit\def\PY@tc##1{\textcolor[rgb]{0.25,0.50,0.50}{##1}}}
\expandafter\def\csname PY@tok@cm\endcsname{\let\PY@it=\textit\def\PY@tc##1{\textcolor[rgb]{0.25,0.50,0.50}{##1}}}
\expandafter\def\csname PY@tok@cpf\endcsname{\let\PY@it=\textit\def\PY@tc##1{\textcolor[rgb]{0.25,0.50,0.50}{##1}}}
\expandafter\def\csname PY@tok@c1\endcsname{\let\PY@it=\textit\def\PY@tc##1{\textcolor[rgb]{0.25,0.50,0.50}{##1}}}
\expandafter\def\csname PY@tok@cs\endcsname{\let\PY@it=\textit\def\PY@tc##1{\textcolor[rgb]{0.25,0.50,0.50}{##1}}}

\def\PYZbs{\char`\\}
\def\PYZus{\char`\_}
\def\PYZob{\char`\{}
\def\PYZcb{\char`\}}
\def\PYZca{\char`\^}
\def\PYZam{\char`\&}
\def\PYZlt{\char`\<}
\def\PYZgt{\char`\>}
\def\PYZsh{\char`\#}
\def\PYZpc{\char`\%}
\def\PYZdl{\char`\$}
\def\PYZhy{\char`\-}
\def\PYZsq{\char`\'}
\def\PYZdq{\char`\"}
\def\PYZti{\char`\~}
% for compatibility with earlier versions
\def\PYZat{@}
\def\PYZlb{[}
\def\PYZrb{]}
\makeatother


    % For linebreaks inside Verbatim environment from package fancyvrb. 
    \makeatletter
        \newbox\Wrappedcontinuationbox 
        \newbox\Wrappedvisiblespacebox 
        \newcommand*\Wrappedvisiblespace {\textcolor{red}{\textvisiblespace}} 
        \newcommand*\Wrappedcontinuationsymbol {\textcolor{red}{\llap{\tiny$\m@th\hookrightarrow$}}} 
        \newcommand*\Wrappedcontinuationindent {3ex } 
        \newcommand*\Wrappedafterbreak {\kern\Wrappedcontinuationindent\copy\Wrappedcontinuationbox} 
        % Take advantage of the already applied Pygments mark-up to insert 
        % potential linebreaks for TeX processing. 
        %        {, <, #, %, $, ' and ": go to next line. 
        %        _, }, ^, &, >, - and ~: stay at end of broken line. 
        % Use of \textquotesingle for straight quote. 
        \newcommand*\Wrappedbreaksatspecials {% 
            \def\PYGZus{\discretionary{\char`\_}{\Wrappedafterbreak}{\char`\_}}% 
            \def\PYGZob{\discretionary{}{\Wrappedafterbreak\char`\{}{\char`\{}}% 
            \def\PYGZcb{\discretionary{\char`\}}{\Wrappedafterbreak}{\char`\}}}% 
            \def\PYGZca{\discretionary{\char`\^}{\Wrappedafterbreak}{\char`\^}}% 
            \def\PYGZam{\discretionary{\char`\&}{\Wrappedafterbreak}{\char`\&}}% 
            \def\PYGZlt{\discretionary{}{\Wrappedafterbreak\char`\<}{\char`\<}}% 
            \def\PYGZgt{\discretionary{\char`\>}{\Wrappedafterbreak}{\char`\>}}% 
            \def\PYGZsh{\discretionary{}{\Wrappedafterbreak\char`\#}{\char`\#}}% 
            \def\PYGZpc{\discretionary{}{\Wrappedafterbreak\char`\%}{\char`\%}}% 
            \def\PYGZdl{\discretionary{}{\Wrappedafterbreak\char`\$}{\char`\$}}% 
            \def\PYGZhy{\discretionary{\char`\-}{\Wrappedafterbreak}{\char`\-}}% 
            \def\PYGZsq{\discretionary{}{\Wrappedafterbreak\textquotesingle}{\textquotesingle}}% 
            \def\PYGZdq{\discretionary{}{\Wrappedafterbreak\char`\"}{\char`\"}}% 
            \def\PYGZti{\discretionary{\char`\~}{\Wrappedafterbreak}{\char`\~}}% 
        } 
        % Some characters . , ; ? ! / are not pygmentized. 
        % This macro makes them "active" and they will insert potential linebreaks 
        \newcommand*\Wrappedbreaksatpunct {% 
            \lccode`\~`\.\lowercase{\def~}{\discretionary{\hbox{\char`\.}}{\Wrappedafterbreak}{\hbox{\char`\.}}}% 
            \lccode`\~`\,\lowercase{\def~}{\discretionary{\hbox{\char`\,}}{\Wrappedafterbreak}{\hbox{\char`\,}}}% 
            \lccode`\~`\;\lowercase{\def~}{\discretionary{\hbox{\char`\;}}{\Wrappedafterbreak}{\hbox{\char`\;}}}% 
            \lccode`\~`\:\lowercase{\def~}{\discretionary{\hbox{\char`\:}}{\Wrappedafterbreak}{\hbox{\char`\:}}}% 
            \lccode`\~`\?\lowercase{\def~}{\discretionary{\hbox{\char`\?}}{\Wrappedafterbreak}{\hbox{\char`\?}}}% 
            \lccode`\~`\!\lowercase{\def~}{\discretionary{\hbox{\char`\!}}{\Wrappedafterbreak}{\hbox{\char`\!}}}% 
            \lccode`\~`\/\lowercase{\def~}{\discretionary{\hbox{\char`\/}}{\Wrappedafterbreak}{\hbox{\char`\/}}}% 
            \catcode`\.\active
            \catcode`\,\active 
            \catcode`\;\active
            \catcode`\:\active
            \catcode`\?\active
            \catcode`\!\active
            \catcode`\/\active 
            \lccode`\~`\~ 	
        }
    \makeatother

    \let\OriginalVerbatim=\Verbatim
    \makeatletter
    \renewcommand{\Verbatim}[1][1]{%
        %\parskip\z@skip
        \sbox\Wrappedcontinuationbox {\Wrappedcontinuationsymbol}%
        \sbox\Wrappedvisiblespacebox {\FV@SetupFont\Wrappedvisiblespace}%
        \def\FancyVerbFormatLine ##1{\hsize\linewidth
            \vtop{\raggedright\hyphenpenalty\z@\exhyphenpenalty\z@
                \doublehyphendemerits\z@\finalhyphendemerits\z@
                \strut ##1\strut}%
        }%
        % If the linebreak is at a space, the latter will be displayed as visible
        % space at end of first line, and a continuation symbol starts next line.
        % Stretch/shrink are however usually zero for typewriter font.
        \def\FV@Space {%
            \nobreak\hskip\z@ plus\fontdimen3\font minus\fontdimen4\font
            \discretionary{\copy\Wrappedvisiblespacebox}{\Wrappedafterbreak}
            {\kern\fontdimen2\font}%
        }%
        
        % Allow breaks at special characters using \PYG... macros.
        \Wrappedbreaksatspecials
        % Breaks at punctuation characters . , ; ? ! and / need catcode=\active 	
        \OriginalVerbatim[#1,codes*=\Wrappedbreaksatpunct]%
    }
    \makeatother

    % Exact colors from NB
    \definecolor{incolor}{HTML}{303F9F}
    \definecolor{outcolor}{HTML}{D84315}
    \definecolor{cellborder}{HTML}{CFCFCF}
    \definecolor{cellbackground}{HTML}{F7F7F7}
    
    % prompt
    \makeatletter
    \newcommand{\boxspacing}{\kern\kvtcb@left@rule\kern\kvtcb@boxsep}
    \makeatother
    \newcommand{\prompt}[4]{
        \ttfamily\llap{{\color{#2}[#3]:\hspace{3pt}#4}}\vspace{-\baselineskip}
    }
    

    
    % Prevent overflowing lines due to hard-to-break entities
    \sloppy 
    % Setup hyperref package
    \hypersetup{
      breaklinks=true,  % so long urls are correctly broken across lines
      colorlinks=true,
      urlcolor=urlcolor,
      linkcolor=linkcolor,
      citecolor=citecolor,
      }
    % Slightly bigger margins than the latex defaults
    
    \geometry{verbose,tmargin=0.25in,bmargin=1in,lmargin=1in,rmargin=1in}
    
    

\begin{document}
    
    \maketitle
    %\hypertarget{numerics-1-homework-01}{%
%\section*{Numerics 1: Homework 01}\label{numerics-1-homework-01}}

    \hypertarget{setup}{%
\subsection*{Setup}\label{setup}}

    \begin{tcolorbox}[breakable, size=fbox, boxrule=1pt, pad at break*=1mm,colback=cellbackground, colframe=cellborder]
\prompt{In}{incolor}{1}{\boxspacing}
\begin{Verbatim}[commandchars=\\\{\}]
\PY{k+kn}{import} \PY{n+nn}{numpy} \PY{k}{as} \PY{n+nn}{np}
\PY{k+kn}{import} \PY{n+nn}{pandas} \PY{k}{as} \PY{n+nn}{pd}
\PY{k+kn}{import} \PY{n+nn}{matplotlib}\PY{n+nn}{.}\PY{n+nn}{pyplot} \PY{k}{as} \PY{n+nn}{plt}
\PY{k+kn}{from} \PY{n+nn}{IPython}\PY{n+nn}{.}\PY{n+nn}{display} \PY{k+kn}{import} \PY{n}{display}
\end{Verbatim}
\end{tcolorbox}

    \hypertarget{problem-1}{%
\subsection*{Problem 1}\label{problem-1}}

We consider the following calculations and how to avoid cancellation due
to numerical imprecision.

    \hypertarget{a.}{%
\subsubsection*{a).}\label{a.}}

\[ \sqrt{x+1}-1,\:x\approx 0 \]

    The problem here is that for small magnitude \(x\) we will be
subtracting 1 from something very close to 1. This follows not only
analytically but also because we can only approximate the square root up
to a certain precision. We can multiply by the conjugate to try and
eliminate this problem.

\[
\sqrt{x+1}-1\cdot\frac{\sqrt{x+1}+1}{\sqrt{x+1}+1} = \frac{x}{\sqrt{x+1}+1}
\]

Using the expression \(\boxed{\frac{x}{\sqrt{x+1}+1}}\) we can eliminate
numerical imprecision for \(x\approx0\) as we are no longer using any
subtraction.

    \hypertarget{b.}{%
\subsubsection*{b).}\label{b.}}

\[ \sin(x)-\sin(y),\: x\approx y \]

    In the above expression the two sine values will be almost identical. We
can use a trig sum identity to try and eliminate this subtraction.

\[
\sin(x) - \sin(y) = 2\cos(\frac{x+y}{2})\sin(\frac{x-y}{2})
\]

Using \(\boxed{2\cos(\frac{x+y}{2})\sin(\frac{x-y}{2})}\) to compute the
desired expression will be much more numerically stable. We can note
that we have not eliminated all subtraction as an \((x-y)\) term
remains. However, given that the numbers representable by a machine are
only a finite subset of the real numbers, we reduce error by doing the
subtraction before approximating the \(\sin\).

    \hypertarget{c.}{%
\subsubsection*{c).}\label{c.}}

\[ \frac{1-\cos(x)}{\sin(x)},\: x\approx 0 \]

    The cancellation problem in the above expression comes from the
\(1-\cos(x)\) term where \(\cos(x)\approx1\) for small magnitude x. We
can solve this problem by multiplying by the conjugate.

\[
\frac{1-\cos(x)}{\sin(x)}\cdot\frac{1+\cos(x)}{1+\cos(x)}=\frac{1-\cos^2(x)}{\sin(x)(1+\cos(x))} = \frac{\sin(x)}{1+\cos(x)}
\]

Using \(\boxed{\frac{\sin(x)}{1+\cos(x)}}\) we can avoid numerical
issues due to cancellcation.

    \hypertarget{problem-2}{%
\subsection*{Problem 2}\label{problem-2}}

Consider the following polynomial:

\[ p(x)=(x-2)^9=x^9-18x^8+144x^7-672x^6+2016x^5-4032x^4+5376x^3-4608x^2+2304x-512 \]

Taking \(x=[1.920:0.001:2.080]\) we will examine the variation in
evaluation depending on the form of the polynomial used.

    \begin{tcolorbox}[breakable, size=fbox, boxrule=1pt, pad at break*=1mm,colback=cellbackground, colframe=cellborder]
\prompt{In}{incolor}{11}{\boxspacing}
\begin{Verbatim}[commandchars=\\\{\}]
\PY{n}{step} \PY{o}{=} \PY{l+m+mf}{0.001}
\PY{n}{x} \PY{o}{=} \PY{n}{np}\PY{o}{.}\PY{n}{arange}\PY{p}{(}\PY{l+m+mf}{1.920}\PY{p}{,} \PY{l+m+mf}{2.080}\PY{o}{+}\PY{n}{step}\PY{p}{,} \PY{n}{step}\PY{p}{)}

\PY{k}{def} \PY{n+nf}{p\PYZus{}factored}\PY{p}{(}\PY{n}{x}\PY{p}{)}\PY{p}{:}
    \PY{k}{return} \PY{p}{(}\PY{n}{x}\PY{o}{\PYZhy{}}\PY{l+m+mi}{2}\PY{p}{)}\PY{o}{*}\PY{o}{*}\PY{l+m+mi}{9}

\PY{k}{def} \PY{n+nf}{p\PYZus{}expanded}\PY{p}{(}\PY{n}{x}\PY{p}{)}\PY{p}{:}
    \PY{k}{return} \PY{n}{x}\PY{o}{*}\PY{o}{*}\PY{l+m+mi}{9} \PY{o}{\PYZhy{}} \PY{l+m+mi}{18}\PY{o}{*}\PY{n}{x}\PY{o}{*}\PY{o}{*}\PY{l+m+mi}{8} \PY{o}{+} \PY{l+m+mi}{144}\PY{o}{*}\PY{n}{x}\PY{o}{*}\PY{o}{*}\PY{l+m+mi}{7} \PY{o}{\PYZhy{}} \PY{l+m+mi}{672}\PY{o}{*}\PY{n}{x}\PY{o}{*}\PY{o}{*}\PY{l+m+mi}{6} \PY{o}{+}\PYZbs{}
            \PY{l+m+mi}{2016}\PY{o}{*}\PY{n}{x}\PY{o}{*}\PY{o}{*}\PY{l+m+mi}{5} \PY{o}{\PYZhy{}} \PY{l+m+mi}{4032}\PY{o}{*}\PY{n}{x}\PY{o}{*}\PY{o}{*}\PY{l+m+mi}{4} \PY{o}{+} \PY{l+m+mi}{5376}\PY{o}{*}\PY{n}{x}\PY{o}{*}\PY{o}{*}\PY{l+m+mi}{3} \PY{o}{\PYZhy{}}\PYZbs{}
            \PY{l+m+mi}{4608}\PY{o}{*}\PY{n}{x}\PY{o}{*}\PY{o}{*}\PY{l+m+mi}{2} \PY{o}{+} \PY{l+m+mi}{2304}\PY{o}{*}\PY{n}{x} \PY{o}{\PYZhy{}} \PY{l+m+mi}{512}
\end{Verbatim}
\end{tcolorbox}

    \hypertarget{a.}{%
\subsubsection*{a).}\label{a.}}

We plot \(p(x)\) via the expanded form.

    \begin{tcolorbox}[breakable, size=fbox, boxrule=1pt, pad at break*=1mm,colback=cellbackground, colframe=cellborder]
\prompt{In}{incolor}{12}{\boxspacing}
\begin{Verbatim}[commandchars=\\\{\}]
\PY{n}{fig}\PY{p}{,} \PY{n}{ax} \PY{o}{=} \PY{n}{plt}\PY{o}{.}\PY{n}{subplots}\PY{p}{(}\PY{l+m+mi}{1}\PY{p}{,}\PY{l+m+mi}{1}\PY{p}{,}\PY{n}{figsize}\PY{o}{=}\PY{p}{(}\PY{l+m+mi}{10}\PY{p}{,}\PY{l+m+mi}{10}\PY{p}{)}\PY{p}{)}
\PY{n}{ax}\PY{o}{.}\PY{n}{set\PYZus{}title}\PY{p}{(}\PY{l+s+s1}{\PYZsq{}}\PY{l+s+s1}{p(x) via expanded form}\PY{l+s+s1}{\PYZsq{}}\PY{p}{)}
\PY{n}{ax}\PY{o}{.}\PY{n}{set\PYZus{}xlabel}\PY{p}{(}\PY{l+s+s1}{\PYZsq{}}\PY{l+s+s1}{x}\PY{l+s+s1}{\PYZsq{}}\PY{p}{)}
\PY{n}{ax}\PY{o}{.}\PY{n}{set\PYZus{}ylabel}\PY{p}{(}\PY{l+s+s1}{\PYZsq{}}\PY{l+s+s1}{p(x)}\PY{l+s+s1}{\PYZsq{}}\PY{p}{)}

\PY{n}{ax}\PY{o}{.}\PY{n}{plot}\PY{p}{(}\PY{n}{x}\PY{p}{,} \PY{n}{p\PYZus{}expanded}\PY{p}{(}\PY{n}{x}\PY{p}{)}\PY{p}{)}\PY{p}{;}
\end{Verbatim}
\end{tcolorbox}

    \begin{center}
    \adjustimage{max size={0.9\linewidth}{0.9\paperheight}}{output_13_0.png}
    \end{center}
    { \hspace*{\fill} \\}
    
    \hypertarget{b.}{%
\subsubsection*{b).}\label{b.}}

We plot \(p(x)\) using the factored form.

    \begin{tcolorbox}[breakable, size=fbox, boxrule=1pt, pad at break*=1mm,colback=cellbackground, colframe=cellborder]
\prompt{In}{incolor}{13}{\boxspacing}
\begin{Verbatim}[commandchars=\\\{\}]
\PY{n}{fig}\PY{p}{,} \PY{n}{ax} \PY{o}{=} \PY{n}{plt}\PY{o}{.}\PY{n}{subplots}\PY{p}{(}\PY{l+m+mi}{1}\PY{p}{,}\PY{l+m+mi}{1}\PY{p}{,}\PY{n}{figsize}\PY{o}{=}\PY{p}{(}\PY{l+m+mi}{10}\PY{p}{,}\PY{l+m+mi}{10}\PY{p}{)}\PY{p}{)}
\PY{n}{ax}\PY{o}{.}\PY{n}{set\PYZus{}title}\PY{p}{(}\PY{l+s+s1}{\PYZsq{}}\PY{l+s+s1}{p(x) via factored form}\PY{l+s+s1}{\PYZsq{}}\PY{p}{)}
\PY{n}{ax}\PY{o}{.}\PY{n}{set\PYZus{}xlabel}\PY{p}{(}\PY{l+s+s1}{\PYZsq{}}\PY{l+s+s1}{x}\PY{l+s+s1}{\PYZsq{}}\PY{p}{)}
\PY{n}{ax}\PY{o}{.}\PY{n}{set\PYZus{}ylabel}\PY{p}{(}\PY{l+s+s1}{\PYZsq{}}\PY{l+s+s1}{p(x)}\PY{l+s+s1}{\PYZsq{}}\PY{p}{)}

\PY{n}{ax}\PY{o}{.}\PY{n}{plot}\PY{p}{(}\PY{n}{x}\PY{p}{,} \PY{n}{p\PYZus{}factored}\PY{p}{(}\PY{n}{x}\PY{p}{)}\PY{p}{)}\PY{p}{;}
\end{Verbatim}
\end{tcolorbox}

    \begin{center}
    \adjustimage{max size={0.9\linewidth}{0.9\paperheight}}{output_15_0.png}
    \end{center}
    { \hspace*{\fill} \\}
    
    \hypertarget{c.}{%
\subsubsection*{c).}\label{c.}}

Thinking about our function \((x-2)^9\) we expect only one zero at
\(x=2\) (with higher multiplicity) and a curve that is somewhat similar
to a cubic, but with a wider range where the slope is almost zero around
\(x=2\). In the two plots above we can see wildly different outputs. In
the first of the plots -- where we have evaluated the polynomial using
its expanded form -- we see a jagged curve that in a general sense
resembles our expectation. In the second plot we have evaluated the
polynomial using the factored form, and we can see the curve is much
smoother and looks exactly as we would expect. Given this we can clearly
tell that the second plot (using the factored form) is the correct plot.
The discrepancy comes from the way the polynomial was evaluated. When
using the expanded form there are far more multiplies and adds
introducing more and more numerical imprecision. The general shape looks
correct, but there is significant numerical noise.

\newpage

    \hypertarget{problem-3}{%
\subsection*{Problem 3}\label{problem-3}}

We consider computing \(y=x_1-x_2\) where \(\bar{x}_1=x_1+\Delta x_1\)
and \(\bar{x}_2=x_2+\Delta x_2\) are approximations to the exact value.
This operation analytically results in
\(\bar{y}=y+(\Delta x_1 + \Delta x_2)\) where we denote
\(\Delta y=\Delta x_1 + \Delta x_2\).

    \hypertarget{a.}{%
\subsubsection*{a).}\label{a.}}

We find upper bounds on the absolute error \(|\Delta y|\) and the
relative error \(\frac{|\Delta y|}{|y|}\).

    We will use the Triangle Inequality (\(|a + b|\leq |a| + |b|\)) and the
Reverse Triangle Inequality (\(||a|-|b||\leq|a + b|\)) to bound our
expression.

For the absolute error a direct application of our inequality will give
us our bound.

\[
\boxed{|\Delta y| = |\Delta x_1 + \Delta x_2|\leq|\Delta x_1| + |\Delta x_2|}
\]

For the relative error we can do the same thing with an extra step using
the reverse inequality.

\begin{align}
    &\frac{|\Delta y|}{|y|}=\frac{|\Delta x_1 + \Delta x_2|}{|x_1 - x_2|}\\
    &~\\
    &|\Delta x_1 + \Delta x_2|\leq |\Delta x_1| + |\Delta x_2| \text{ and } |x_1 - x_2|\geq ||x_1| - |x_2||\\
    &~\\
    &\implies \boxed{\frac{|\Delta y|}{|y|}\leq \frac{|\Delta x_1| + |\Delta x_2|}{||x_1| - |x_2||}}
\end{align}

We used the Triangle Inequality to replace the numerator with something
larger and the Reverse Triangle Inequality to replace the denominator
with something smaller (making the whole expression larger).

    With this we can see that when \(x_1\) and \(x_2\) are close to each
other in magnitude the relative error is large. This follows because the
numerator is large and the denominator is small.

    \hypertarget{b.}{%
\subsubsection*{b).}\label{b.}}

We examine the expression \(\cos(x+\delta)-\cos(x)\) and the numerical
differences between it and an equivalent expression with no subtraction.

    We can use a trig sum identity to transform our original expression into
the following: \(-2\sin(\frac{2x+\delta}{2})\sin(\frac{\delta}{2})\). We
will then plot the difference between the two expressions below.

    \begin{tcolorbox}[breakable, size=fbox, boxrule=1pt, pad at break*=1mm,colback=cellbackground, colframe=cellborder]
\prompt{In}{incolor}{16}{\boxspacing}
\begin{Verbatim}[commandchars=\\\{\}]
\PY{n}{delta} \PY{o}{=} \PY{n}{np}\PY{o}{.}\PY{n}{logspace}\PY{p}{(}\PY{o}{\PYZhy{}}\PY{l+m+mi}{16}\PY{p}{,} \PY{l+m+mi}{0}\PY{p}{,} \PY{n}{base}\PY{o}{=}\PY{l+m+mi}{10}\PY{p}{)}
\PY{n}{x\PYZus{}vec} \PY{o}{=} \PY{p}{[}\PY{n}{np}\PY{o}{.}\PY{n}{pi}\PY{p}{,} \PY{l+m+mf}{1E3}\PY{p}{]} \PY{c+c1}{\PYZsh{}Arbitrary}

\PY{k}{def} \PY{n+nf}{f1}\PY{p}{(}\PY{n}{d}\PY{p}{,} \PY{n}{x}\PY{p}{)}\PY{p}{:}
    \PY{k}{return} \PY{n}{np}\PY{o}{.}\PY{n}{cos}\PY{p}{(}\PY{n}{x}\PY{o}{+}\PY{n}{d}\PY{p}{)}\PY{o}{\PYZhy{}}\PY{n}{np}\PY{o}{.}\PY{n}{cos}\PY{p}{(}\PY{n}{x}\PY{p}{)}

\PY{k}{def} \PY{n+nf}{f2}\PY{p}{(}\PY{n}{d}\PY{p}{,} \PY{n}{x}\PY{p}{)}\PY{p}{:}
    \PY{k}{return} \PY{o}{\PYZhy{}}\PY{l+m+mi}{2}\PY{o}{*}\PY{n}{np}\PY{o}{.}\PY{n}{sin}\PY{p}{(}\PY{p}{(}\PY{l+m+mi}{2}\PY{o}{*}\PY{n}{x}\PY{o}{+}\PY{n}{d}\PY{p}{)}\PY{o}{/}\PY{l+m+mi}{2}\PY{p}{)}\PY{o}{*}\PY{n}{np}\PY{o}{.}\PY{n}{sin}\PY{p}{(}\PY{n}{d}\PY{o}{/}\PY{l+m+mi}{2}\PY{p}{)}
\end{Verbatim}
\end{tcolorbox}

    \begin{tcolorbox}[breakable, size=fbox, boxrule=1pt, pad at break*=1mm,colback=cellbackground, colframe=cellborder]
\prompt{In}{incolor}{17}{\boxspacing}
\begin{Verbatim}[commandchars=\\\{\}]
\PY{n}{fig}\PY{p}{,} \PY{n}{ax} \PY{o}{=} \PY{n}{plt}\PY{o}{.}\PY{n}{subplots}\PY{p}{(}\PY{l+m+mi}{1}\PY{p}{,}\PY{l+m+mi}{2}\PY{p}{,}\PY{n}{figsize}\PY{o}{=}\PY{p}{(}\PY{l+m+mi}{15}\PY{p}{,}\PY{l+m+mi}{10}\PY{p}{)}\PY{p}{)}
\PY{n}{ax}\PY{p}{[}\PY{l+m+mi}{0}\PY{p}{]}\PY{o}{.}\PY{n}{set\PYZus{}title}\PY{p}{(}\PY{l+s+sa}{r}\PY{l+s+s1}{\PYZsq{}}\PY{l+s+s1}{Error for \PYZdl{}x=}\PY{l+s+s1}{\PYZbs{}}\PY{l+s+s1}{pi\PYZdl{}}\PY{l+s+s1}{\PYZsq{}}\PY{p}{)}
\PY{n}{ax}\PY{p}{[}\PY{l+m+mi}{1}\PY{p}{]}\PY{o}{.}\PY{n}{set\PYZus{}title}\PY{p}{(}\PY{l+s+sa}{r}\PY{l+s+s1}{\PYZsq{}}\PY{l+s+s1}{Error for \PYZdl{}x=1000\PYZdl{}}\PY{l+s+s1}{\PYZsq{}}\PY{p}{)}

\PY{k}{for} \PY{n}{i}\PY{p}{,}\PY{n}{x} \PY{o+ow}{in} \PY{n+nb}{enumerate}\PY{p}{(}\PY{n}{x\PYZus{}vec}\PY{p}{)}\PY{p}{:}
    \PY{n}{ax}\PY{p}{[}\PY{n}{i}\PY{p}{]}\PY{o}{.}\PY{n}{set\PYZus{}xlabel}\PY{p}{(}\PY{l+s+sa}{r}\PY{l+s+s1}{\PYZsq{}}\PY{l+s+s1}{\PYZdl{}}\PY{l+s+s1}{\PYZbs{}}\PY{l+s+s1}{delta\PYZdl{}}\PY{l+s+s1}{\PYZsq{}}\PY{p}{)}
    \PY{n}{ax}\PY{p}{[}\PY{n}{i}\PY{p}{]}\PY{o}{.}\PY{n}{set\PYZus{}ylabel}\PY{p}{(}\PY{l+s+s1}{\PYZsq{}}\PY{l+s+s1}{Error}\PY{l+s+s1}{\PYZsq{}}\PY{p}{)}

    \PY{n}{ax}\PY{p}{[}\PY{n}{i}\PY{p}{]}\PY{o}{.}\PY{n}{loglog}\PY{p}{(}\PY{n}{delta}\PY{p}{,} \PY{n}{np}\PY{o}{.}\PY{n}{abs}\PY{p}{(}\PY{n}{f1}\PY{p}{(}\PY{n}{delta}\PY{p}{,} \PY{n}{x}\PY{p}{)}\PY{o}{\PYZhy{}}\PY{n}{f2}\PY{p}{(}\PY{n}{delta}\PY{p}{,}\PY{n}{x}\PY{p}{)}\PY{p}{)}\PY{o}{/}\PY{n}{np}\PY{o}{.}\PY{n}{abs}\PY{p}{(}\PY{n}{f2}\PY{p}{(}\PY{n}{delta}\PY{p}{,}\PY{n}{x}\PY{p}{)}\PY{p}{)}\PY{p}{)}
    \PY{n}{ax}\PY{p}{[}\PY{n}{i}\PY{p}{]}\PY{o}{.}\PY{n}{loglog}\PY{p}{(}\PY{n}{delta}\PY{p}{,} \PY{n}{np}\PY{o}{.}\PY{n}{abs}\PY{p}{(}\PY{n}{f1}\PY{p}{(}\PY{n}{delta}\PY{p}{,}\PY{n}{x}\PY{p}{)}\PY{o}{\PYZhy{}}\PY{n}{f2}\PY{p}{(}\PY{n}{delta}\PY{p}{,} \PY{n}{x}\PY{p}{)}\PY{p}{)}\PY{p}{)}

    \PY{n}{ax}\PY{p}{[}\PY{n}{i}\PY{p}{]}\PY{o}{.}\PY{n}{legend}\PY{p}{(}\PY{p}{[}\PY{l+s+s1}{\PYZsq{}}\PY{l+s+s1}{Relative Error}\PY{l+s+s1}{\PYZsq{}}\PY{p}{,} \PY{l+s+s1}{\PYZsq{}}\PY{l+s+s1}{Absolute Error}\PY{l+s+s1}{\PYZsq{}}\PY{p}{]}\PY{p}{)}\PY{p}{;}
\end{Verbatim}
\end{tcolorbox}

    \begin{center}
    \adjustimage{max size={0.9\linewidth}{0.9\paperheight}}{output_24_0.png}
    \end{center}
    { \hspace*{\fill} \\}
    
    In the plots above we see the relative and absolute error computed for
\(x=\pi,1000\). We can see that in both cases the relative error starts
out large and will decrease as \(\delta\) increases while the absolute
error does the opposite.

    \hypertarget{c.}{%
\subsubsection*{c).}\label{c.}}

Talyor expansion on \(f(x+\delta)-f(x)\) results in,

\[ \delta f'(x)+\frac{\delta^2}{2}f''(\zeta),\:\zeta\in[x,x+\delta] \]

Applying this to \(cos(x+\delta)-cos(x)\) we have the following result:

\[ \delta(-sin(x)) + \frac{\delta^2}{2}(-cos(\zeta)) \]

    \begin{tcolorbox}[breakable, size=fbox, boxrule=1pt, pad at break*=1mm,colback=cellbackground, colframe=cellborder]
\prompt{In}{incolor}{18}{\boxspacing}
\begin{Verbatim}[commandchars=\\\{\}]
\PY{n}{zeta} \PY{o}{=} \PY{n}{x\PYZus{}vec} \PY{c+c1}{\PYZsh{}Maximize cos on [pi, pi+epsilon] and [pi/4,pi/4+epsilon]}

\PY{k}{def} \PY{n+nf}{f3}\PY{p}{(}\PY{n}{d}\PY{p}{,} \PY{n}{x}\PY{p}{,} \PY{n}{z}\PY{p}{)}\PY{p}{:}
    \PY{k}{return} \PY{o}{\PYZhy{}}\PY{n}{d}\PY{o}{*}\PY{n}{np}\PY{o}{.}\PY{n}{sin}\PY{p}{(}\PY{n}{x}\PY{p}{)}\PY{o}{\PYZhy{}}\PY{p}{(}\PY{n}{d}\PY{o}{*}\PY{o}{*}\PY{l+m+mi}{2}\PY{o}{/}\PY{l+m+mi}{2}\PY{p}{)}\PY{o}{*}\PY{n}{np}\PY{o}{.}\PY{n}{cos}\PY{p}{(}\PY{n}{z}\PY{p}{)}
\end{Verbatim}
\end{tcolorbox}

    \begin{tcolorbox}[breakable, size=fbox, boxrule=1pt, pad at break*=1mm,colback=cellbackground, colframe=cellborder]
\prompt{In}{incolor}{21}{\boxspacing}
\begin{Verbatim}[commandchars=\\\{\}]
\PY{n}{fig}\PY{p}{,} \PY{n}{ax} \PY{o}{=} \PY{n}{plt}\PY{o}{.}\PY{n}{subplots}\PY{p}{(}\PY{l+m+mi}{1}\PY{p}{,}\PY{l+m+mi}{2}\PY{p}{,}\PY{n}{figsize}\PY{o}{=}\PY{p}{(}\PY{l+m+mi}{15}\PY{p}{,}\PY{l+m+mi}{10}\PY{p}{)}\PY{p}{)}
\PY{n}{ax}\PY{p}{[}\PY{l+m+mi}{0}\PY{p}{]}\PY{o}{.}\PY{n}{set\PYZus{}title}\PY{p}{(}\PY{l+s+sa}{r}\PY{l+s+s1}{\PYZsq{}}\PY{l+s+s1}{Error for \PYZdl{}x=}\PY{l+s+s1}{\PYZbs{}}\PY{l+s+s1}{pi\PYZdl{}}\PY{l+s+s1}{\PYZsq{}}\PY{p}{)}
\PY{n}{ax}\PY{p}{[}\PY{l+m+mi}{1}\PY{p}{]}\PY{o}{.}\PY{n}{set\PYZus{}title}\PY{p}{(}\PY{l+s+sa}{r}\PY{l+s+s1}{\PYZsq{}}\PY{l+s+s1}{Error for \PYZdl{}x=1000\PYZdl{}}\PY{l+s+s1}{\PYZsq{}}\PY{p}{)}

\PY{k}{for} \PY{n}{i}\PY{p}{,}\PY{n}{x} \PY{o+ow}{in} \PY{n+nb}{enumerate}\PY{p}{(}\PY{n}{x\PYZus{}vec}\PY{p}{)}\PY{p}{:}
    \PY{n}{ax}\PY{p}{[}\PY{n}{i}\PY{p}{]}\PY{o}{.}\PY{n}{set\PYZus{}xlabel}\PY{p}{(}\PY{l+s+sa}{r}\PY{l+s+s1}{\PYZsq{}}\PY{l+s+s1}{\PYZdl{}}\PY{l+s+s1}{\PYZbs{}}\PY{l+s+s1}{delta\PYZdl{}}\PY{l+s+s1}{\PYZsq{}}\PY{p}{)}
    \PY{n}{ax}\PY{p}{[}\PY{n}{i}\PY{p}{]}\PY{o}{.}\PY{n}{set\PYZus{}ylabel}\PY{p}{(}\PY{l+s+s1}{\PYZsq{}}\PY{l+s+s1}{Error}\PY{l+s+s1}{\PYZsq{}}\PY{p}{)}

    \PY{n}{ax}\PY{p}{[}\PY{n}{i}\PY{p}{]}\PY{o}{.}\PY{n}{loglog}\PY{p}{(}\PY{n}{delta}\PY{p}{,} \PY{n}{np}\PY{o}{.}\PY{n}{abs}\PY{p}{(}\PY{n}{f1}\PY{p}{(}\PY{n}{delta}\PY{p}{,} \PY{n}{x}\PY{p}{)}\PY{o}{\PYZhy{}}\PY{n}{f3}\PY{p}{(}\PY{n}{delta}\PY{p}{,}\PY{n}{x}\PY{p}{,}\PY{n}{x}\PY{p}{)}\PY{p}{)}\PY{o}{/}\PY{n}{np}\PY{o}{.}\PY{n}{abs}\PY{p}{(}\PY{n}{f3}\PY{p}{(}\PY{n}{delta}\PY{p}{,}\PY{n}{x}\PY{p}{,}\PY{n}{x}\PY{p}{)}\PY{p}{)}\PY{p}{)}
    \PY{n}{ax}\PY{p}{[}\PY{n}{i}\PY{p}{]}\PY{o}{.}\PY{n}{loglog}\PY{p}{(}\PY{n}{delta}\PY{p}{,} \PY{n}{np}\PY{o}{.}\PY{n}{abs}\PY{p}{(}\PY{n}{f1}\PY{p}{(}\PY{n}{delta}\PY{p}{,}\PY{n}{x}\PY{p}{)}\PY{o}{\PYZhy{}}\PY{n}{f3}\PY{p}{(}\PY{n}{delta}\PY{p}{,} \PY{n}{x}\PY{p}{,}\PY{n}{x}\PY{p}{)}\PY{p}{)}\PY{p}{)}

    \PY{n}{ax}\PY{p}{[}\PY{n}{i}\PY{p}{]}\PY{o}{.}\PY{n}{legend}\PY{p}{(}\PY{p}{[}\PY{l+s+s1}{\PYZsq{}}\PY{l+s+s1}{Relative Error}\PY{l+s+s1}{\PYZsq{}}\PY{p}{,} \PY{l+s+s1}{\PYZsq{}}\PY{l+s+s1}{Absolute Error}\PY{l+s+s1}{\PYZsq{}}\PY{p}{]}\PY{p}{)}\PY{p}{;}
\end{Verbatim}
\end{tcolorbox}

    \begin{center}
    \adjustimage{max size={0.9\linewidth}{0.9\paperheight}}{output_28_0.png}
    \end{center}
    { \hspace*{\fill} \\}
    
    Again we see the absolute and relative error for \(x=\pi,1000\). When
using the Taylor expansion we see that for larger delta (i.e closer to
1) the error is very large. Specifically, we see the same behavior as in
the previous plots, but with a sharp increase towards the end of our
range of \(\delta\). This makes sense as we are only using a first order
approximation, so for larger \(\delta\) we are leaving a lot of
information out. Thus we can conclude that for larger \(\delta\) the
first method (using the trig identity) is the better option. For smaller
\(\delta\) both methods seem equivalent.

    \hypertarget{problem-4}{%
\subsection*{Problem 4}\label{problem-4}}

We want to show that \((1+x)^n=1+nx+\mathcal{o}(x),\:n\in\mathbf{Z}\) as
\(x\to0\).

    We will show that \((1+x)^n-1-nx=\mathcal{o}(x) \text{ as } x\to0\).

\begin{align}
    &\lim_{x\to0}\frac{|(1+x)^n-1-nx)|}{|x|}\\
    &~\\
    &\text{Expanding $(1+x)^n$ as a Taylor series centered at 0 gives the following:}\\
    &~\\
    &\lim_{x\to0}\frac{|(1+nx+\frac{n(n-1)x^2}{2}+\cdots)-1-nx)|}{|x|}\\
    &=\lim_{x\to0}\frac{|\frac{n(n-1)x^2}{2}+\cdots|}{|x|}=0
\end{align}

Thus, because the limit is zero we can conclude that
\((1+x)^n=1+nx+\mathcal{o}(x),\:n\in\mathbf{Z}\) as \(x\to0\).

    \hypertarget{problem-5}{%
\subsection*{Problem 5}\label{problem-5}}

We want to show \(x\sin(\sqrt{x})=\mathcal{O}(x^{\frac{3}{2}})\) as
\(x\to0\).

\begin{align}
    &\lim_{x\to0}\frac{x\sin(\sqrt{x})}{x^{\frac{3}{2}}} = \lim_{x\to0}\frac{\sin(\sqrt{x})}{\sqrt{x}}\\
    &\text{Taylor}\implies = \lim_{x\to0}\frac{\sqrt{x}-x^{\frac{3}{2}} + \cdots}{\sqrt{x}}\\
    &=\lim_{x\to0} 1 - x + \cdots = 1
\end{align}

Thus, because the limit is constant we can conclude that
\(x\sin(\sqrt{x})=\mathcal{O}(x^{\frac{3}{2}})\).

\newpage

    \hypertarget{problem-6}{%
\subsection*{Problem 6}\label{problem-6}}

We will use the Bisection method on the factored and expanded form of
the polynomial \(f(x)=(x-5)^9\) which has a root with multiplicity 9 at
\(x=5\).

    \begin{tcolorbox}[breakable, size=fbox, boxrule=1pt, pad at break*=1mm,colback=cellbackground, colframe=cellborder]
\prompt{In}{incolor}{23}{\boxspacing}
\begin{Verbatim}[commandchars=\\\{\}]
\PY{l+s+sd}{\PYZsq{}\PYZsq{}\PYZsq{}}
\PY{l+s+sd}{Function: Bisection}
\PY{l+s+sd}{Input:}
\PY{l+s+sd}{    a: left end point guess}
\PY{l+s+sd}{    b: righ end point guess}
\PY{l+s+sd}{    f: a callable function to find the root}
\PY{l+s+sd}{    tol: tolerance for the root}
\PY{l+s+sd}{    imax: max iterations}
\PY{l+s+sd}{    info: whether to return number of iterations}
\PY{l+s+sd}{Output:}
\PY{l+s+sd}{    \PYZhy{}the root within the tolerance}
\PY{l+s+sd}{    \PYZhy{}number of iterations to identify root (optional)}
\PY{l+s+sd}{Errors:}
\PY{l+s+sd}{    \PYZhy{}if given function is not callable}
\PY{l+s+sd}{    \PYZhy{}if no identifiable root bounded by initial guesses}
\PY{l+s+sd}{\PYZsq{}\PYZsq{}\PYZsq{}}
\PY{k}{def} \PY{n+nf}{bisection}\PY{p}{(}\PY{n}{a}\PY{p}{,} \PY{n}{b}\PY{p}{,} \PY{n}{f}\PY{p}{,} \PY{n}{tol}\PY{o}{=}\PY{l+m+mf}{1E\PYZhy{}4}\PY{p}{,} \PY{n}{imax}\PY{o}{=}\PY{l+m+mi}{1000}\PY{p}{,} \PY{n}{info}\PY{o}{=}\PY{k+kc}{False}\PY{p}{)}\PY{p}{:}
    \PY{c+c1}{\PYZsh{}Check if f is callable}
    \PY{k}{if} \PY{o+ow}{not} \PY{n}{callable}\PY{p}{(}\PY{n}{f}\PY{p}{)}\PY{p}{:}
        \PY{k}{raise} \PY{n+ne}{ValueError}\PY{p}{(}\PY{l+s+s1}{\PYZsq{}}\PY{l+s+s1}{Given function not callable}\PY{l+s+s1}{\PYZsq{}}\PY{p}{)}
    
    \PY{c+c1}{\PYZsh{}Check if there is a zero}
    \PY{k}{if} \PY{n}{f}\PY{p}{(}\PY{n}{a}\PY{p}{)}\PY{o}{*}\PY{n}{f}\PY{p}{(}\PY{n}{b}\PY{p}{)}\PY{o}{\PYZgt{}}\PY{l+m+mi}{0}\PY{p}{:}
        \PY{k}{raise} \PY{n+ne}{ValueError}\PY{p}{(}\PY{l+s+s1}{\PYZsq{}}\PY{l+s+s1}{Initial guesses do not bound a zero, or guesses are poor.}\PY{l+s+s1}{\PYZsq{}}\PY{p}{)}
        
    \PY{c+c1}{\PYZsh{}Bisect}
    \PY{k}{for} \PY{n}{i} \PY{o+ow}{in} \PY{n+nb}{range}\PY{p}{(}\PY{n}{imax}\PY{p}{)}\PY{p}{:} \PY{c+c1}{\PYZsh{}Stay under max iterations}
        \PY{n}{c} \PY{o}{=} \PY{p}{(}\PY{n}{a}\PY{o}{+}\PY{n}{b}\PY{p}{)}\PY{o}{/}\PY{l+m+mi}{2} \PY{c+c1}{\PYZsh{}Take the midpoint}
        
        \PY{k}{if} \PY{n}{f}\PY{p}{(}\PY{n}{c}\PY{p}{)}\PY{o}{==}\PY{l+m+mi}{0} \PY{o+ow}{or} \PY{p}{(}\PY{n}{b}\PY{o}{\PYZhy{}}\PY{n}{a}\PY{p}{)}\PY{o}{/}\PY{l+m+mi}{2}\PY{o}{\PYZlt{}}\PY{n}{tol}\PY{p}{:} \PY{c+c1}{\PYZsh{}Check to see if root is found}
            \PY{k}{if} \PY{n}{info}\PY{p}{:}
                \PY{k}{return} \PY{n}{c}\PY{p}{,} \PY{n}{i}\PY{o}{+}\PY{l+m+mi}{1}
            \PY{k}{return} \PY{n}{c}

        \PY{k}{if} \PY{n}{f}\PY{p}{(}\PY{n}{a}\PY{p}{)}\PY{o}{*}\PY{n}{f}\PY{p}{(}\PY{n}{c}\PY{p}{)}\PY{o}{\PYZlt{}}\PY{l+m+mi}{0}\PY{p}{:} \PY{c+c1}{\PYZsh{}Reset the interval}
            \PY{n}{b} \PY{o}{=} \PY{n}{c}
        \PY{k}{else}\PY{p}{:}
            \PY{n}{a} \PY{o}{=} \PY{n}{c}
        
    \PY{k}{return} \PY{k+kc}{None} \PY{c+c1}{\PYZsh{}Something failed along the way}

\PY{n}{a}\PY{o}{=}\PY{l+m+mf}{4.8}
\PY{n}{b}\PY{o}{=}\PY{l+m+mf}{5.3}
\end{Verbatim}
\end{tcolorbox}

    \hypertarget{a.}{%
\subsubsection*{a).}\label{a.}}

Using the factored form.

    \begin{tcolorbox}[breakable, size=fbox, boxrule=1pt, pad at break*=1mm,colback=cellbackground, colframe=cellborder]
\prompt{In}{incolor}{34}{\boxspacing}
\begin{Verbatim}[commandchars=\\\{\}]
\PY{k}{def} \PY{n+nf}{f\PYZus{}factored}\PY{p}{(}\PY{n}{x}\PY{p}{)}\PY{p}{:}
    \PY{k}{return} \PY{p}{(}\PY{n}{x}\PY{o}{\PYZhy{}}\PY{l+m+mi}{5}\PY{p}{)}\PY{o}{*}\PY{o}{*}\PY{l+m+mi}{9}

\PY{n}{root}\PY{p}{,} \PY{n}{i} \PY{o}{=} \PY{n}{bisection}\PY{p}{(}\PY{n}{a}\PY{p}{,} \PY{n}{b}\PY{p}{,} \PY{n}{f\PYZus{}factored}\PY{p}{,} \PY{n}{info}\PY{o}{=}\PY{k+kc}{True}\PY{p}{)}
\PY{n+nb}{print}\PY{p}{(}\PY{l+s+s1}{\PYZsq{}}\PY{l+s+s1}{Root @ x=}\PY{l+s+si}{\PYZpc{}f}\PY{l+s+s1}{, Took }\PY{l+s+si}{\PYZpc{}i}\PY{l+s+s1}{ iterations}\PY{l+s+s1}{\PYZsq{}} \PY{o}{\PYZpc{}} \PY{p}{(}\PY{n}{root}\PY{p}{,} \PY{n}{i}\PY{p}{)}\PY{p}{)}
\end{Verbatim}
\end{tcolorbox}

    \begin{Verbatim}[commandchars=\\\{\}]
Root @ x=5.000012, Took 13 iterations
    \end{Verbatim}

    \hypertarget{b.}{%
\subsubsection*{b).}\label{b.}}

Using the expanded form.

    \begin{tcolorbox}[breakable, size=fbox, boxrule=1pt, pad at break*=1mm,colback=cellbackground, colframe=cellborder]
\prompt{In}{incolor}{35}{\boxspacing}
\begin{Verbatim}[commandchars=\\\{\}]
\PY{k}{def} \PY{n+nf}{f\PYZus{}expanded}\PY{p}{(}\PY{n}{x}\PY{p}{)}\PY{p}{:}
    \PY{k}{return} \PY{n}{x}\PY{o}{*}\PY{o}{*}\PY{l+m+mi}{9} \PY{o}{\PYZhy{}} \PY{l+m+mi}{45}\PY{o}{*}\PY{n}{x}\PY{o}{*}\PY{o}{*}\PY{l+m+mi}{8} \PY{o}{+} \PY{l+m+mi}{900}\PY{o}{*}\PY{n}{x}\PY{o}{*}\PY{o}{*}\PY{l+m+mi}{7} \PY{o}{\PYZhy{}} \PY{l+m+mi}{10500}\PY{o}{*}\PY{n}{x}\PY{o}{*}\PY{o}{*}\PY{l+m+mi}{6} \PY{o}{+} \PY{l+m+mi}{78750}\PY{o}{*}\PY{n}{x}\PY{o}{*}\PY{o}{*}\PY{l+m+mi}{5} \PY{o}{+} \PYZbs{}
            \PY{o}{\PYZhy{}}\PY{l+m+mi}{393750}\PY{o}{*}\PY{n}{x}\PY{o}{*}\PY{o}{*}\PY{l+m+mi}{4} \PY{o}{+} \PY{l+m+mi}{1312500}\PY{o}{*}\PY{n}{x}\PY{o}{*}\PY{o}{*}\PY{l+m+mi}{3} \PY{o}{\PYZhy{}} \PY{l+m+mi}{2812500}\PY{o}{*}\PY{n}{x}\PY{o}{*}\PY{o}{*}\PY{l+m+mi}{2} \PY{o}{+} \PY{l+m+mi}{3515625}\PY{o}{*}\PY{n}{x} \PY{o}{+} \PYZbs{}
            \PY{o}{\PYZhy{}} \PY{l+m+mi}{1953125}

\PY{n}{root}\PY{p}{,} \PY{n}{i} \PY{o}{=} \PY{n}{bisection}\PY{p}{(}\PY{n}{a}\PY{p}{,} \PY{n}{b}\PY{p}{,} \PY{n}{f\PYZus{}expanded}\PY{p}{,} \PY{n}{info}\PY{o}{=}\PY{k+kc}{True}\PY{p}{)}
\PY{n+nb}{print}\PY{p}{(}\PY{l+s+s1}{\PYZsq{}}\PY{l+s+s1}{Root @ x=}\PY{l+s+si}{\PYZpc{}f}\PY{l+s+s1}{, Took }\PY{l+s+si}{\PYZpc{}i}\PY{l+s+s1}{ iterations}\PY{l+s+s1}{\PYZsq{}} \PY{o}{\PYZpc{}} \PY{p}{(}\PY{n}{root}\PY{p}{,} \PY{n}{i}\PY{p}{)}\PY{p}{)}
\end{Verbatim}
\end{tcolorbox}

    \begin{Verbatim}[commandchars=\\\{\}]
Root @ x=5.050000, Took 1 iterations
    \end{Verbatim}

    \hypertarget{c.}{%
\subsubsection*{c).}\label{c.}}

Using the factored form of the polynomial our Bisection method found the
root to within \(10^-5\) in 13 iterations -- outperforming our required
tolerance. However, using the expanded form the algorithm only ran for 1
iteration, and was not able to achieve the required tolerance. We can
note that the root estimate it gave us was the first midpoint
\(\frac{5.3+4.8}{2}=5.05\). To investigate what may have gone wrong we
can evaluate the expanded polynomial at this location.

    \begin{tcolorbox}[breakable, size=fbox, boxrule=1pt, pad at break*=1mm,colback=cellbackground, colframe=cellborder]
\prompt{In}{incolor}{44}{\boxspacing}
\begin{Verbatim}[commandchars=\\\{\}]
\PY{n}{fe} \PY{o}{=} \PY{n}{f\PYZus{}expanded}\PY{p}{(}\PY{l+m+mf}{5.05}\PY{p}{)}
\PY{n}{ff} \PY{o}{=} \PY{n}{f\PYZus{}factored}\PY{p}{(}\PY{l+m+mf}{5.05}\PY{p}{)}

\PY{n+nb}{print}\PY{p}{(}\PY{l+s+s1}{\PYZsq{}}\PY{l+s+s1}{Expanded p(5.05)=}\PY{l+s+si}{\PYZpc{}.15f}\PY{l+s+s1}{, p(5.05)==0? }\PY{l+s+si}{\PYZpc{}s}\PY{l+s+s1}{\PYZsq{}} \PY{o}{\PYZpc{}} \PY{p}{(}\PY{n}{fe}\PY{p}{,} \PY{n}{fe}\PY{o}{==}\PY{l+m+mi}{0}\PY{p}{)}\PY{p}{)}
\PY{n+nb}{print}\PY{p}{(}\PY{l+s+s1}{\PYZsq{}}\PY{l+s+s1}{Factored p(5.05)=}\PY{l+s+si}{\PYZpc{}.15f}\PY{l+s+s1}{, p(5.05)==0? }\PY{l+s+si}{\PYZpc{}s}\PY{l+s+s1}{\PYZsq{}} \PY{o}{\PYZpc{}} \PY{p}{(}\PY{n}{ff}\PY{p}{,} \PY{n}{ff}\PY{o}{==}\PY{l+m+mi}{0}\PY{p}{)}\PY{p}{)}
\end{Verbatim}
\end{tcolorbox}

    \begin{Verbatim}[commandchars=\\\{\}]
Expanded p(5.05)=0.000000000000000, p(5.05)==0? True
Factored p(5.05)=0.000000000001953, p(5.05)==0? False
    \end{Verbatim}

    Ah! So the Bisection method computed the first midpoint, checked the
function value at that midpoint, and then concluded it was the root. It
only took one iteration because it thought that was all it needed. As we
can see above, both methods evaluate to something close to zero, but it
is only the expanded form that is identically zero.

Clearly the expanded form of the polynomial has enough numerical
imprecision in it that it evaluated to zero when it should not have. All
of the error in each multiply and add compounded to result in
\(p(5.05)=0\) when that should not be the case. The function is indeed
close to zero at around \(x=5\), but it was only the factored form the
was stable enough to find the real root.


    % Add a bibliography block to the postdoc
    
    
    
\end{document}
