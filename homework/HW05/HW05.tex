\documentclass[11pt]{article}
	\usepackage{amsthm}
	\renewcommand\qedsymbol{$\blacksquare$}

    \usepackage[breakable]{tcolorbox}
    \usepackage{parskip} % Stop auto-indenting (to mimic markdown behaviour)
    
    \usepackage{iftex}
    \ifPDFTeX
    	\usepackage[T1]{fontenc}
    	\usepackage{mathpazo}
    \else
    	\usepackage{fontspec}
    \fi

    % Basic figure setup, for now with no caption control since it's done
    % automatically by Pandoc (which extracts ![](path) syntax from Markdown).
    \usepackage{graphicx}
    % Maintain compatibility with old templates. Remove in nbconvert 6.0
    \let\Oldincludegraphics\includegraphics
    % Ensure that by default, figures have no caption (until we provide a
    % proper Figure object with a Caption API and a way to capture that
    % in the conversion process - todo).
    \usepackage{caption}
    \DeclareCaptionFormat{nocaption}{}
    \captionsetup{format=nocaption,aboveskip=0pt,belowskip=0pt}

    \usepackage{float}
    \floatplacement{figure}{H} % forces figures to be placed at the correct location
    \usepackage{xcolor} % Allow colors to be defined
    \usepackage{enumerate} % Needed for markdown enumerations to work
    \usepackage{geometry} % Used to adjust the document margins
    \usepackage{amsmath} % Equations
    \usepackage{amssymb} % Equations
    \usepackage{textcomp} % defines textquotesingle
    % Hack from http://tex.stackexchange.com/a/47451/13684:
    \AtBeginDocument{%
        \def\PYZsq{\textquotesingle}% Upright quotes in Pygmentized code
    }
    \usepackage{upquote} % Upright quotes for verbatim code
    \usepackage{eurosym} % defines \euro
    \usepackage[mathletters]{ucs} % Extended unicode (utf-8) support
    \usepackage{fancyvrb} % verbatim replacement that allows latex
    \usepackage{grffile} % extends the file name processing of package graphics 
                         % to support a larger range
    \makeatletter % fix for old versions of grffile with XeLaTeX
    \@ifpackagelater{grffile}{2019/11/01}
    {
      % Do nothing on new versions
    }
    {
      \def\Gread@@xetex#1{%
        \IfFileExists{"\Gin@base".bb}%
        {\Gread@eps{\Gin@base.bb}}%
        {\Gread@@xetex@aux#1}%
      }
    }
    \makeatother
    \usepackage[Export]{adjustbox} % Used to constrain images to a maximum size
    \adjustboxset{max size={0.9\linewidth}{0.9\paperheight}}

    % The hyperref package gives us a pdf with properly built
    % internal navigation ('pdf bookmarks' for the table of contents,
    % internal cross-reference links, web links for URLs, etc.)
    \usepackage{hyperref}
    % The default LaTeX title has an obnoxious amount of whitespace. By default,
    % titling removes some of it. It also provides customization options.
    \usepackage{titling}
    \usepackage{longtable} % longtable support required by pandoc >1.10
    \usepackage{booktabs}  % table support for pandoc > 1.12.2
    \usepackage[inline]{enumitem} % IRkernel/repr support (it uses the enumerate* environment)
    \usepackage[normalem]{ulem} % ulem is needed to support strikethroughs (\sout)
                                % normalem makes italics be italics, not underlines
    \usepackage{mathrsfs}
    

    
    % Colors for the hyperref package
    \definecolor{urlcolor}{rgb}{0,.145,.698}
    \definecolor{linkcolor}{rgb}{.71,0.21,0.01}
    \definecolor{citecolor}{rgb}{.12,.54,.11}

    % ANSI colors
    \definecolor{ansi-black}{HTML}{3E424D}
    \definecolor{ansi-black-intense}{HTML}{282C36}
    \definecolor{ansi-red}{HTML}{E75C58}
    \definecolor{ansi-red-intense}{HTML}{B22B31}
    \definecolor{ansi-green}{HTML}{00A250}
    \definecolor{ansi-green-intense}{HTML}{007427}
    \definecolor{ansi-yellow}{HTML}{DDB62B}
    \definecolor{ansi-yellow-intense}{HTML}{B27D12}
    \definecolor{ansi-blue}{HTML}{208FFB}
    \definecolor{ansi-blue-intense}{HTML}{0065CA}
    \definecolor{ansi-magenta}{HTML}{D160C4}
    \definecolor{ansi-magenta-intense}{HTML}{A03196}
    \definecolor{ansi-cyan}{HTML}{60C6C8}
    \definecolor{ansi-cyan-intense}{HTML}{258F8F}
    \definecolor{ansi-white}{HTML}{C5C1B4}
    \definecolor{ansi-white-intense}{HTML}{A1A6B2}
    \definecolor{ansi-default-inverse-fg}{HTML}{FFFFFF}
    \definecolor{ansi-default-inverse-bg}{HTML}{000000}

    % common color for the border for error outputs.
    \definecolor{outerrorbackground}{HTML}{FFDFDF}

    % commands and environments needed by pandoc snippets
    % extracted from the output of `pandoc -s`
    \providecommand{\tightlist}{%
      \setlength{\itemsep}{0pt}\setlength{\parskip}{0pt}}
    \DefineVerbatimEnvironment{Highlighting}{Verbatim}{commandchars=\\\{\}}
    % Add ',fontsize=\small' for more characters per line
    \newenvironment{Shaded}{}{}
    \newcommand{\KeywordTok}[1]{\textcolor[rgb]{0.00,0.44,0.13}{\textbf{{#1}}}}
    \newcommand{\DataTypeTok}[1]{\textcolor[rgb]{0.56,0.13,0.00}{{#1}}}
    \newcommand{\DecValTok}[1]{\textcolor[rgb]{0.25,0.63,0.44}{{#1}}}
    \newcommand{\BaseNTok}[1]{\textcolor[rgb]{0.25,0.63,0.44}{{#1}}}
    \newcommand{\FloatTok}[1]{\textcolor[rgb]{0.25,0.63,0.44}{{#1}}}
    \newcommand{\CharTok}[1]{\textcolor[rgb]{0.25,0.44,0.63}{{#1}}}
    \newcommand{\StringTok}[1]{\textcolor[rgb]{0.25,0.44,0.63}{{#1}}}
    \newcommand{\CommentTok}[1]{\textcolor[rgb]{0.38,0.63,0.69}{\textit{{#1}}}}
    \newcommand{\OtherTok}[1]{\textcolor[rgb]{0.00,0.44,0.13}{{#1}}}
    \newcommand{\AlertTok}[1]{\textcolor[rgb]{1.00,0.00,0.00}{\textbf{{#1}}}}
    \newcommand{\FunctionTok}[1]{\textcolor[rgb]{0.02,0.16,0.49}{{#1}}}
    \newcommand{\RegionMarkerTok}[1]{{#1}}
    \newcommand{\ErrorTok}[1]{\textcolor[rgb]{1.00,0.00,0.00}{\textbf{{#1}}}}
    \newcommand{\NormalTok}[1]{{#1}}
    
    % Additional commands for more recent versions of Pandoc
    \newcommand{\ConstantTok}[1]{\textcolor[rgb]{0.53,0.00,0.00}{{#1}}}
    \newcommand{\SpecialCharTok}[1]{\textcolor[rgb]{0.25,0.44,0.63}{{#1}}}
    \newcommand{\VerbatimStringTok}[1]{\textcolor[rgb]{0.25,0.44,0.63}{{#1}}}
    \newcommand{\SpecialStringTok}[1]{\textcolor[rgb]{0.73,0.40,0.53}{{#1}}}
    \newcommand{\ImportTok}[1]{{#1}}
    \newcommand{\DocumentationTok}[1]{\textcolor[rgb]{0.73,0.13,0.13}{\textit{{#1}}}}
    \newcommand{\AnnotationTok}[1]{\textcolor[rgb]{0.38,0.63,0.69}{\textbf{\textit{{#1}}}}}
    \newcommand{\CommentVarTok}[1]{\textcolor[rgb]{0.38,0.63,0.69}{\textbf{\textit{{#1}}}}}
    \newcommand{\VariableTok}[1]{\textcolor[rgb]{0.10,0.09,0.49}{{#1}}}
    \newcommand{\ControlFlowTok}[1]{\textcolor[rgb]{0.00,0.44,0.13}{\textbf{{#1}}}}
    \newcommand{\OperatorTok}[1]{\textcolor[rgb]{0.40,0.40,0.40}{{#1}}}
    \newcommand{\BuiltInTok}[1]{{#1}}
    \newcommand{\ExtensionTok}[1]{{#1}}
    \newcommand{\PreprocessorTok}[1]{\textcolor[rgb]{0.74,0.48,0.00}{{#1}}}
    \newcommand{\AttributeTok}[1]{\textcolor[rgb]{0.49,0.56,0.16}{{#1}}}
    \newcommand{\InformationTok}[1]{\textcolor[rgb]{0.38,0.63,0.69}{\textbf{\textit{{#1}}}}}
    \newcommand{\WarningTok}[1]{\textcolor[rgb]{0.38,0.63,0.69}{\textbf{\textit{{#1}}}}}
    
    
    % Define a nice break command that doesn't care if a line doesn't already
    % exist.
    \def\br{\hspace*{\fill} \\* }
    % Math Jax compatibility definitions
    \def\gt{>}
    \def\lt{<}
    \let\Oldtex\TeX
    \let\Oldlatex\LaTeX
    \renewcommand{\TeX}{\textrm{\Oldtex}}
    \renewcommand{\LaTeX}{\textrm{\Oldlatex}}
    % Document parameters
    % Document title
    \title{Numerics 1: HW 5}
    \author{Cooper Simpson}
    
    
    
    
    
% Pygments definitions
\makeatletter
\def\PY@reset{\let\PY@it=\relax \let\PY@bf=\relax%
    \let\PY@ul=\relax \let\PY@tc=\relax%
    \let\PY@bc=\relax \let\PY@ff=\relax}
\def\PY@tok#1{\csname PY@tok@#1\endcsname}
\def\PY@toks#1+{\ifx\relax#1\empty\else%
    \PY@tok{#1}\expandafter\PY@toks\fi}
\def\PY@do#1{\PY@bc{\PY@tc{\PY@ul{%
    \PY@it{\PY@bf{\PY@ff{#1}}}}}}}
\def\PY#1#2{\PY@reset\PY@toks#1+\relax+\PY@do{#2}}

\expandafter\def\csname PY@tok@w\endcsname{\def\PY@tc##1{\textcolor[rgb]{0.73,0.73,0.73}{##1}}}
\expandafter\def\csname PY@tok@c\endcsname{\let\PY@it=\textit\def\PY@tc##1{\textcolor[rgb]{0.25,0.50,0.50}{##1}}}
\expandafter\def\csname PY@tok@cp\endcsname{\def\PY@tc##1{\textcolor[rgb]{0.74,0.48,0.00}{##1}}}
\expandafter\def\csname PY@tok@k\endcsname{\let\PY@bf=\textbf\def\PY@tc##1{\textcolor[rgb]{0.00,0.50,0.00}{##1}}}
\expandafter\def\csname PY@tok@kp\endcsname{\def\PY@tc##1{\textcolor[rgb]{0.00,0.50,0.00}{##1}}}
\expandafter\def\csname PY@tok@kt\endcsname{\def\PY@tc##1{\textcolor[rgb]{0.69,0.00,0.25}{##1}}}
\expandafter\def\csname PY@tok@o\endcsname{\def\PY@tc##1{\textcolor[rgb]{0.40,0.40,0.40}{##1}}}
\expandafter\def\csname PY@tok@ow\endcsname{\let\PY@bf=\textbf\def\PY@tc##1{\textcolor[rgb]{0.67,0.13,1.00}{##1}}}
\expandafter\def\csname PY@tok@nb\endcsname{\def\PY@tc##1{\textcolor[rgb]{0.00,0.50,0.00}{##1}}}
\expandafter\def\csname PY@tok@nf\endcsname{\def\PY@tc##1{\textcolor[rgb]{0.00,0.00,1.00}{##1}}}
\expandafter\def\csname PY@tok@nc\endcsname{\let\PY@bf=\textbf\def\PY@tc##1{\textcolor[rgb]{0.00,0.00,1.00}{##1}}}
\expandafter\def\csname PY@tok@nn\endcsname{\let\PY@bf=\textbf\def\PY@tc##1{\textcolor[rgb]{0.00,0.00,1.00}{##1}}}
\expandafter\def\csname PY@tok@ne\endcsname{\let\PY@bf=\textbf\def\PY@tc##1{\textcolor[rgb]{0.82,0.25,0.23}{##1}}}
\expandafter\def\csname PY@tok@nv\endcsname{\def\PY@tc##1{\textcolor[rgb]{0.10,0.09,0.49}{##1}}}
\expandafter\def\csname PY@tok@no\endcsname{\def\PY@tc##1{\textcolor[rgb]{0.53,0.00,0.00}{##1}}}
\expandafter\def\csname PY@tok@nl\endcsname{\def\PY@tc##1{\textcolor[rgb]{0.63,0.63,0.00}{##1}}}
\expandafter\def\csname PY@tok@ni\endcsname{\let\PY@bf=\textbf\def\PY@tc##1{\textcolor[rgb]{0.60,0.60,0.60}{##1}}}
\expandafter\def\csname PY@tok@na\endcsname{\def\PY@tc##1{\textcolor[rgb]{0.49,0.56,0.16}{##1}}}
\expandafter\def\csname PY@tok@nt\endcsname{\let\PY@bf=\textbf\def\PY@tc##1{\textcolor[rgb]{0.00,0.50,0.00}{##1}}}
\expandafter\def\csname PY@tok@nd\endcsname{\def\PY@tc##1{\textcolor[rgb]{0.67,0.13,1.00}{##1}}}
\expandafter\def\csname PY@tok@s\endcsname{\def\PY@tc##1{\textcolor[rgb]{0.73,0.13,0.13}{##1}}}
\expandafter\def\csname PY@tok@sd\endcsname{\let\PY@it=\textit\def\PY@tc##1{\textcolor[rgb]{0.73,0.13,0.13}{##1}}}
\expandafter\def\csname PY@tok@si\endcsname{\let\PY@bf=\textbf\def\PY@tc##1{\textcolor[rgb]{0.73,0.40,0.53}{##1}}}
\expandafter\def\csname PY@tok@se\endcsname{\let\PY@bf=\textbf\def\PY@tc##1{\textcolor[rgb]{0.73,0.40,0.13}{##1}}}
\expandafter\def\csname PY@tok@sr\endcsname{\def\PY@tc##1{\textcolor[rgb]{0.73,0.40,0.53}{##1}}}
\expandafter\def\csname PY@tok@ss\endcsname{\def\PY@tc##1{\textcolor[rgb]{0.10,0.09,0.49}{##1}}}
\expandafter\def\csname PY@tok@sx\endcsname{\def\PY@tc##1{\textcolor[rgb]{0.00,0.50,0.00}{##1}}}
\expandafter\def\csname PY@tok@m\endcsname{\def\PY@tc##1{\textcolor[rgb]{0.40,0.40,0.40}{##1}}}
\expandafter\def\csname PY@tok@gh\endcsname{\let\PY@bf=\textbf\def\PY@tc##1{\textcolor[rgb]{0.00,0.00,0.50}{##1}}}
\expandafter\def\csname PY@tok@gu\endcsname{\let\PY@bf=\textbf\def\PY@tc##1{\textcolor[rgb]{0.50,0.00,0.50}{##1}}}
\expandafter\def\csname PY@tok@gd\endcsname{\def\PY@tc##1{\textcolor[rgb]{0.63,0.00,0.00}{##1}}}
\expandafter\def\csname PY@tok@gi\endcsname{\def\PY@tc##1{\textcolor[rgb]{0.00,0.63,0.00}{##1}}}
\expandafter\def\csname PY@tok@gr\endcsname{\def\PY@tc##1{\textcolor[rgb]{1.00,0.00,0.00}{##1}}}
\expandafter\def\csname PY@tok@ge\endcsname{\let\PY@it=\textit}
\expandafter\def\csname PY@tok@gs\endcsname{\let\PY@bf=\textbf}
\expandafter\def\csname PY@tok@gp\endcsname{\let\PY@bf=\textbf\def\PY@tc##1{\textcolor[rgb]{0.00,0.00,0.50}{##1}}}
\expandafter\def\csname PY@tok@go\endcsname{\def\PY@tc##1{\textcolor[rgb]{0.53,0.53,0.53}{##1}}}
\expandafter\def\csname PY@tok@gt\endcsname{\def\PY@tc##1{\textcolor[rgb]{0.00,0.27,0.87}{##1}}}
\expandafter\def\csname PY@tok@err\endcsname{\def\PY@bc##1{\setlength{\fboxsep}{0pt}\fcolorbox[rgb]{1.00,0.00,0.00}{1,1,1}{\strut ##1}}}
\expandafter\def\csname PY@tok@kc\endcsname{\let\PY@bf=\textbf\def\PY@tc##1{\textcolor[rgb]{0.00,0.50,0.00}{##1}}}
\expandafter\def\csname PY@tok@kd\endcsname{\let\PY@bf=\textbf\def\PY@tc##1{\textcolor[rgb]{0.00,0.50,0.00}{##1}}}
\expandafter\def\csname PY@tok@kn\endcsname{\let\PY@bf=\textbf\def\PY@tc##1{\textcolor[rgb]{0.00,0.50,0.00}{##1}}}
\expandafter\def\csname PY@tok@kr\endcsname{\let\PY@bf=\textbf\def\PY@tc##1{\textcolor[rgb]{0.00,0.50,0.00}{##1}}}
\expandafter\def\csname PY@tok@bp\endcsname{\def\PY@tc##1{\textcolor[rgb]{0.00,0.50,0.00}{##1}}}
\expandafter\def\csname PY@tok@fm\endcsname{\def\PY@tc##1{\textcolor[rgb]{0.00,0.00,1.00}{##1}}}
\expandafter\def\csname PY@tok@vc\endcsname{\def\PY@tc##1{\textcolor[rgb]{0.10,0.09,0.49}{##1}}}
\expandafter\def\csname PY@tok@vg\endcsname{\def\PY@tc##1{\textcolor[rgb]{0.10,0.09,0.49}{##1}}}
\expandafter\def\csname PY@tok@vi\endcsname{\def\PY@tc##1{\textcolor[rgb]{0.10,0.09,0.49}{##1}}}
\expandafter\def\csname PY@tok@vm\endcsname{\def\PY@tc##1{\textcolor[rgb]{0.10,0.09,0.49}{##1}}}
\expandafter\def\csname PY@tok@sa\endcsname{\def\PY@tc##1{\textcolor[rgb]{0.73,0.13,0.13}{##1}}}
\expandafter\def\csname PY@tok@sb\endcsname{\def\PY@tc##1{\textcolor[rgb]{0.73,0.13,0.13}{##1}}}
\expandafter\def\csname PY@tok@sc\endcsname{\def\PY@tc##1{\textcolor[rgb]{0.73,0.13,0.13}{##1}}}
\expandafter\def\csname PY@tok@dl\endcsname{\def\PY@tc##1{\textcolor[rgb]{0.73,0.13,0.13}{##1}}}
\expandafter\def\csname PY@tok@s2\endcsname{\def\PY@tc##1{\textcolor[rgb]{0.73,0.13,0.13}{##1}}}
\expandafter\def\csname PY@tok@sh\endcsname{\def\PY@tc##1{\textcolor[rgb]{0.73,0.13,0.13}{##1}}}
\expandafter\def\csname PY@tok@s1\endcsname{\def\PY@tc##1{\textcolor[rgb]{0.73,0.13,0.13}{##1}}}
\expandafter\def\csname PY@tok@mb\endcsname{\def\PY@tc##1{\textcolor[rgb]{0.40,0.40,0.40}{##1}}}
\expandafter\def\csname PY@tok@mf\endcsname{\def\PY@tc##1{\textcolor[rgb]{0.40,0.40,0.40}{##1}}}
\expandafter\def\csname PY@tok@mh\endcsname{\def\PY@tc##1{\textcolor[rgb]{0.40,0.40,0.40}{##1}}}
\expandafter\def\csname PY@tok@mi\endcsname{\def\PY@tc##1{\textcolor[rgb]{0.40,0.40,0.40}{##1}}}
\expandafter\def\csname PY@tok@il\endcsname{\def\PY@tc##1{\textcolor[rgb]{0.40,0.40,0.40}{##1}}}
\expandafter\def\csname PY@tok@mo\endcsname{\def\PY@tc##1{\textcolor[rgb]{0.40,0.40,0.40}{##1}}}
\expandafter\def\csname PY@tok@ch\endcsname{\let\PY@it=\textit\def\PY@tc##1{\textcolor[rgb]{0.25,0.50,0.50}{##1}}}
\expandafter\def\csname PY@tok@cm\endcsname{\let\PY@it=\textit\def\PY@tc##1{\textcolor[rgb]{0.25,0.50,0.50}{##1}}}
\expandafter\def\csname PY@tok@cpf\endcsname{\let\PY@it=\textit\def\PY@tc##1{\textcolor[rgb]{0.25,0.50,0.50}{##1}}}
\expandafter\def\csname PY@tok@c1\endcsname{\let\PY@it=\textit\def\PY@tc##1{\textcolor[rgb]{0.25,0.50,0.50}{##1}}}
\expandafter\def\csname PY@tok@cs\endcsname{\let\PY@it=\textit\def\PY@tc##1{\textcolor[rgb]{0.25,0.50,0.50}{##1}}}

\def\PYZbs{\char`\\}
\def\PYZus{\char`\_}
\def\PYZob{\char`\{}
\def\PYZcb{\char`\}}
\def\PYZca{\char`\^}
\def\PYZam{\char`\&}
\def\PYZlt{\char`\<}
\def\PYZgt{\char`\>}
\def\PYZsh{\char`\#}
\def\PYZpc{\char`\%}
\def\PYZdl{\char`\$}
\def\PYZhy{\char`\-}
\def\PYZsq{\char`\'}
\def\PYZdq{\char`\"}
\def\PYZti{\char`\~}
% for compatibility with earlier versions
\def\PYZat{@}
\def\PYZlb{[}
\def\PYZrb{]}
\makeatother


    % For linebreaks inside Verbatim environment from package fancyvrb. 
    \makeatletter
        \newbox\Wrappedcontinuationbox 
        \newbox\Wrappedvisiblespacebox 
        \newcommand*\Wrappedvisiblespace {\textcolor{red}{\textvisiblespace}} 
        \newcommand*\Wrappedcontinuationsymbol {\textcolor{red}{\llap{\tiny$\m@th\hookrightarrow$}}} 
        \newcommand*\Wrappedcontinuationindent {3ex } 
        \newcommand*\Wrappedafterbreak {\kern\Wrappedcontinuationindent\copy\Wrappedcontinuationbox} 
        % Take advantage of the already applied Pygments mark-up to insert 
        % potential linebreaks for TeX processing. 
        %        {, <, #, %, $, ' and ": go to next line. 
        %        _, }, ^, &, >, - and ~: stay at end of broken line. 
        % Use of \textquotesingle for straight quote. 
        \newcommand*\Wrappedbreaksatspecials {% 
            \def\PYGZus{\discretionary{\char`\_}{\Wrappedafterbreak}{\char`\_}}% 
            \def\PYGZob{\discretionary{}{\Wrappedafterbreak\char`\{}{\char`\{}}% 
            \def\PYGZcb{\discretionary{\char`\}}{\Wrappedafterbreak}{\char`\}}}% 
            \def\PYGZca{\discretionary{\char`\^}{\Wrappedafterbreak}{\char`\^}}% 
            \def\PYGZam{\discretionary{\char`\&}{\Wrappedafterbreak}{\char`\&}}% 
            \def\PYGZlt{\discretionary{}{\Wrappedafterbreak\char`\<}{\char`\<}}% 
            \def\PYGZgt{\discretionary{\char`\>}{\Wrappedafterbreak}{\char`\>}}% 
            \def\PYGZsh{\discretionary{}{\Wrappedafterbreak\char`\#}{\char`\#}}% 
            \def\PYGZpc{\discretionary{}{\Wrappedafterbreak\char`\%}{\char`\%}}% 
            \def\PYGZdl{\discretionary{}{\Wrappedafterbreak\char`\$}{\char`\$}}% 
            \def\PYGZhy{\discretionary{\char`\-}{\Wrappedafterbreak}{\char`\-}}% 
            \def\PYGZsq{\discretionary{}{\Wrappedafterbreak\textquotesingle}{\textquotesingle}}% 
            \def\PYGZdq{\discretionary{}{\Wrappedafterbreak\char`\"}{\char`\"}}% 
            \def\PYGZti{\discretionary{\char`\~}{\Wrappedafterbreak}{\char`\~}}% 
        } 
        % Some characters . , ; ? ! / are not pygmentized. 
        % This macro makes them "active" and they will insert potential linebreaks 
        \newcommand*\Wrappedbreaksatpunct {% 
            \lccode`\~`\.\lowercase{\def~}{\discretionary{\hbox{\char`\.}}{\Wrappedafterbreak}{\hbox{\char`\.}}}% 
            \lccode`\~`\,\lowercase{\def~}{\discretionary{\hbox{\char`\,}}{\Wrappedafterbreak}{\hbox{\char`\,}}}% 
            \lccode`\~`\;\lowercase{\def~}{\discretionary{\hbox{\char`\;}}{\Wrappedafterbreak}{\hbox{\char`\;}}}% 
            \lccode`\~`\:\lowercase{\def~}{\discretionary{\hbox{\char`\:}}{\Wrappedafterbreak}{\hbox{\char`\:}}}% 
            \lccode`\~`\?\lowercase{\def~}{\discretionary{\hbox{\char`\?}}{\Wrappedafterbreak}{\hbox{\char`\?}}}% 
            \lccode`\~`\!\lowercase{\def~}{\discretionary{\hbox{\char`\!}}{\Wrappedafterbreak}{\hbox{\char`\!}}}% 
            \lccode`\~`\/\lowercase{\def~}{\discretionary{\hbox{\char`\/}}{\Wrappedafterbreak}{\hbox{\char`\/}}}% 
            \catcode`\.\active
            \catcode`\,\active 
            \catcode`\;\active
            \catcode`\:\active
            \catcode`\?\active
            \catcode`\!\active
            \catcode`\/\active 
            \lccode`\~`\~ 	
        }
    \makeatother

    \let\OriginalVerbatim=\Verbatim
    \makeatletter
    \renewcommand{\Verbatim}[1][1]{%
        %\parskip\z@skip
        \sbox\Wrappedcontinuationbox {\Wrappedcontinuationsymbol}%
        \sbox\Wrappedvisiblespacebox {\FV@SetupFont\Wrappedvisiblespace}%
        \def\FancyVerbFormatLine ##1{\hsize\linewidth
            \vtop{\raggedright\hyphenpenalty\z@\exhyphenpenalty\z@
                \doublehyphendemerits\z@\finalhyphendemerits\z@
                \strut ##1\strut}%
        }%
        % If the linebreak is at a space, the latter will be displayed as visible
        % space at end of first line, and a continuation symbol starts next line.
        % Stretch/shrink are however usually zero for typewriter font.
        \def\FV@Space {%
            \nobreak\hskip\z@ plus\fontdimen3\font minus\fontdimen4\font
            \discretionary{\copy\Wrappedvisiblespacebox}{\Wrappedafterbreak}
            {\kern\fontdimen2\font}%
        }%
        
        % Allow breaks at special characters using \PYG... macros.
        \Wrappedbreaksatspecials
        % Breaks at punctuation characters . , ; ? ! and / need catcode=\active 	
        \OriginalVerbatim[#1,codes*=\Wrappedbreaksatpunct]%
    }
    \makeatother

    % Exact colors from NB
    \definecolor{incolor}{HTML}{303F9F}
    \definecolor{outcolor}{HTML}{D84315}
    \definecolor{cellborder}{HTML}{CFCFCF}
    \definecolor{cellbackground}{HTML}{F7F7F7}
    
    % prompt
    \makeatletter
    \newcommand{\boxspacing}{\kern\kvtcb@left@rule\kern\kvtcb@boxsep}
    \makeatother
    \newcommand{\prompt}[4]{
        {\ttfamily\llap{{\color{#2}[#3]:\hspace{3pt}#4}}\vspace{-\baselineskip}}
    }
    

    
    % Prevent overflowing lines due to hard-to-break entities
    \sloppy 
    % Setup hyperref package
    \hypersetup{
      breaklinks=true,  % so long urls are correctly broken across lines
      colorlinks=true,
      urlcolor=urlcolor,
      linkcolor=linkcolor,
      citecolor=citecolor,
      }
    % Slightly bigger margins than the latex defaults
    
    \geometry{verbose,tmargin=0.5in,bmargin=1in,lmargin=1in,rmargin=1in}
    
    

\begin{document}
    
    \maketitle
    
    

    
    \hypertarget{numerics-1-homework-05}{%
\section*{Numerics 1: Homework 05}\label{numerics-1-homework-05}}

    \hypertarget{setup}{%
\subsection*{Setup}\label{setup}}

    \begin{tcolorbox}[breakable, size=fbox, boxrule=1pt, pad at break*=1mm,colback=cellbackground, colframe=cellborder]
\prompt{In}{incolor}{59}{\boxspacing}
\begin{Verbatim}[commandchars=\\\{\}]
\PY{k+kn}{import} \PY{n+nn}{numpy} \PY{k}{as} \PY{n+nn}{np}
\PY{k+kn}{import} \PY{n+nn}{pandas} \PY{k}{as} \PY{n+nn}{pd}
\PY{k+kn}{import} \PY{n+nn}{matplotlib}\PY{n+nn}{.}\PY{n+nn}{pyplot} \PY{k}{as} \PY{n+nn}{plt}
\PY{k+kn}{from} \PY{n+nn}{IPython}\PY{n+nn}{.}\PY{n+nn}{display} \PY{k+kn}{import} \PY{n}{display}
\PY{k+kn}{import} \PY{n+nn}{seaborn} \PY{k}{as} \PY{n+nn}{sns}
\PY{n}{sns}\PY{o}{.}\PY{n}{set}\PY{p}{(}\PY{p}{)}
\end{Verbatim}
\end{tcolorbox}

    \hypertarget{problem-1}{%
\subsection*{Problem 1}\label{problem-1}}

    We will consider the system \(\mathbf{Ax}=\mathbf{b}\), with
\(\mathbf{A}\) and \(\mathbf{b}\) defined in the code below, and we will
examine various stationary iteration schemes in relation to this
problem.

    \begin{tcolorbox}[breakable, size=fbox, boxrule=1pt, pad at break*=1mm,colback=cellbackground, colframe=cellborder]
\prompt{In}{incolor}{60}{\boxspacing}
\begin{Verbatim}[commandchars=\\\{\}]
\PY{n}{A} \PY{o}{=} \PY{n}{np}\PY{o}{.}\PY{n}{array}\PY{p}{(}\PY{p}{[}\PY{p}{[}\PY{l+m+mi}{4}\PY{p}{,}\PY{o}{\PYZhy{}}\PY{l+m+mi}{1}\PY{p}{,}\PY{l+m+mi}{0}\PY{p}{,}\PY{o}{\PYZhy{}}\PY{l+m+mi}{1}\PY{p}{,}\PY{l+m+mi}{0}\PY{p}{,}\PY{l+m+mi}{0}\PY{p}{]}\PY{p}{,}
            \PY{p}{[}\PY{o}{\PYZhy{}}\PY{l+m+mi}{1}\PY{p}{,}\PY{l+m+mi}{4}\PY{p}{,}\PY{o}{\PYZhy{}}\PY{l+m+mi}{1}\PY{p}{,}\PY{l+m+mi}{0}\PY{p}{,}\PY{o}{\PYZhy{}}\PY{l+m+mi}{1}\PY{p}{,}\PY{l+m+mi}{0}\PY{p}{]}\PY{p}{,}
            \PY{p}{[}\PY{l+m+mi}{0}\PY{p}{,}\PY{o}{\PYZhy{}}\PY{l+m+mi}{1}\PY{p}{,}\PY{l+m+mi}{4}\PY{p}{,}\PY{o}{\PYZhy{}}\PY{l+m+mi}{1}\PY{p}{,}\PY{l+m+mi}{0}\PY{p}{,}\PY{o}{\PYZhy{}}\PY{l+m+mi}{1}\PY{p}{]}\PY{p}{,}
            \PY{p}{[}\PY{o}{\PYZhy{}}\PY{l+m+mi}{1}\PY{p}{,}\PY{l+m+mi}{0}\PY{p}{,}\PY{o}{\PYZhy{}}\PY{l+m+mi}{1}\PY{p}{,}\PY{l+m+mi}{4}\PY{p}{,}\PY{o}{\PYZhy{}}\PY{l+m+mi}{1}\PY{p}{,}\PY{l+m+mi}{0}\PY{p}{]}\PY{p}{,}
            \PY{p}{[}\PY{l+m+mi}{0}\PY{p}{,}\PY{o}{\PYZhy{}}\PY{l+m+mi}{1}\PY{p}{,}\PY{l+m+mi}{0}\PY{p}{,}\PY{o}{\PYZhy{}}\PY{l+m+mi}{1}\PY{p}{,}\PY{l+m+mi}{4}\PY{p}{,}\PY{o}{\PYZhy{}}\PY{l+m+mi}{1}\PY{p}{]}\PY{p}{,}
            \PY{p}{[}\PY{l+m+mi}{0}\PY{p}{,}\PY{l+m+mi}{0}\PY{p}{,}\PY{o}{\PYZhy{}}\PY{l+m+mi}{1}\PY{p}{,}\PY{l+m+mi}{0}\PY{p}{,}\PY{o}{\PYZhy{}}\PY{l+m+mi}{1}\PY{p}{,}\PY{l+m+mi}{4}\PY{p}{]}\PY{p}{]}\PY{p}{)}

\PY{n}{b} \PY{o}{=} \PY{n}{np}\PY{o}{.}\PY{n}{array}\PY{p}{(}\PY{p}{[}\PY{l+m+mi}{2}\PY{p}{,}\PY{l+m+mi}{1}\PY{p}{,}\PY{l+m+mi}{2}\PY{p}{,}\PY{l+m+mi}{2}\PY{p}{,}\PY{l+m+mi}{1}\PY{p}{,}\PY{l+m+mi}{2}\PY{p}{]}\PY{p}{)}\PY{o}{.}\PY{n}{reshape}\PY{p}{(}\PY{p}{(}\PY{l+m+mi}{6}\PY{p}{,}\PY{l+m+mi}{1}\PY{p}{)}\PY{p}{)}

\PY{n+nb}{print}\PY{p}{(}\PY{n}{A}\PY{o}{.}\PY{n}{shape}\PY{p}{)}
\PY{n+nb}{print}\PY{p}{(}\PY{n}{b}\PY{o}{.}\PY{n}{shape}\PY{p}{)}
\end{Verbatim}
\end{tcolorbox}

    \begin{Verbatim}[commandchars=\\\{\}]
(6, 6)
(6, 1)
    \end{Verbatim}

    All of the following stationary iterations will be using
\(\epsilon=1E-7\), and some combination of the standard splitting which
we compute below.

    \begin{tcolorbox}[breakable, size=fbox, boxrule=1pt, pad at break*=1mm,colback=cellbackground, colframe=cellborder]
\prompt{In}{incolor}{61}{\boxspacing}
\begin{Verbatim}[commandchars=\\\{\}]
\PY{n}{eps} \PY{o}{=} \PY{l+m+mf}{1E\PYZhy{}7} \PY{c+c1}{\PYZsh{}Tolerance}

\PY{n}{L} \PY{o}{=} \PY{n}{np}\PY{o}{.}\PY{n}{tril}\PY{p}{(}\PY{n}{A}\PY{p}{,} \PY{o}{\PYZhy{}}\PY{l+m+mi}{1}\PY{p}{)} \PY{c+c1}{\PYZsh{}Lower triangular}
\PY{n}{U} \PY{o}{=} \PY{n}{np}\PY{o}{.}\PY{n}{triu}\PY{p}{(}\PY{n}{A}\PY{p}{,} \PY{l+m+mi}{1}\PY{p}{)} \PY{c+c1}{\PYZsh{}Upper triangular}

\PY{n}{D} \PY{o}{=} \PY{n}{np}\PY{o}{.}\PY{n}{tril}\PY{p}{(}\PY{n}{np}\PY{o}{.}\PY{n}{triu}\PY{p}{(}\PY{n}{A}\PY{p}{)}\PY{p}{)} \PY{c+c1}{\PYZsh{}Diagonal}
\end{Verbatim}
\end{tcolorbox}

\newpage
    \hypertarget{a.}{%
\subsubsection*{a).}\label{a.}}

We use Gauss-Jacobi to approximate the solution. Gauss-Jacobi is given
by the following iteration:

\[ \mathbf{x}_{k+1} = -\mathbf{D}^{-1}(\mathbf{L}+\mathbf{U})\mathbf{x}_k + \mathbf{D}^{-1}\mathbf{b} \]

    \begin{tcolorbox}[breakable, size=fbox, boxrule=1pt, pad at break*=1mm,colback=cellbackground, colframe=cellborder]
\prompt{In}{incolor}{62}{\boxspacing}
\begin{Verbatim}[commandchars=\\\{\}]
\PY{n}{err} \PY{o}{=} \PY{n}{np}\PY{o}{.}\PY{n}{inf} \PY{c+c1}{\PYZsh{}Initial error}
\PY{n}{x\PYZus{}0} \PY{o}{=} \PY{n}{np}\PY{o}{.}\PY{n}{zeros}\PY{p}{(}\PY{n}{b}\PY{o}{.}\PY{n}{shape}\PY{p}{)} \PY{c+c1}{\PYZsh{}Initial solution}

\PY{c+c1}{\PYZsh{}Pre\PYZhy{}compute static matrices}
\PY{n}{D\PYZus{}inv} \PY{o}{=} \PY{n}{np}\PY{o}{.}\PY{n}{linalg}\PY{o}{.}\PY{n}{inv}\PY{p}{(}\PY{n}{D}\PY{p}{)}
\PY{n}{D\PYZus{}invL\PYZus{}U} \PY{o}{=} \PY{n}{D\PYZus{}inv}\PY{o}{@}\PY{p}{(}\PY{n}{L}\PY{o}{+}\PY{n}{U}\PY{p}{)}
\PY{n}{D\PYZus{}invb} \PY{o}{=} \PY{n}{D\PYZus{}inv}\PY{n+nd}{@b}

\PY{n}{numI} \PY{o}{=} \PY{l+m+mi}{0} \PY{c+c1}{\PYZsh{}Number of iterations}

\PY{k}{while}\PY{p}{(}\PY{n}{err} \PY{o}{\PYZgt{}}\PY{o}{=} \PY{n}{eps}\PY{p}{)}\PY{p}{:}
    \PY{n}{x\PYZus{}k} \PY{o}{=} \PY{o}{\PYZhy{}}\PY{n}{D\PYZus{}invL\PYZus{}U}\PY{n+nd}{@x\PYZus{}0} \PY{o}{+} \PY{n}{D\PYZus{}invb}
    
    \PY{c+c1}{\PYZsh{}Compute error}
    \PY{n}{err} \PY{o}{=} \PY{n}{np}\PY{o}{.}\PY{n}{linalg}\PY{o}{.}\PY{n}{norm}\PY{p}{(}\PY{n}{x\PYZus{}k}\PY{o}{\PYZhy{}}\PY{n}{x\PYZus{}0}\PY{p}{,} \PY{n+nb}{ord}\PY{o}{=}\PY{n}{np}\PY{o}{.}\PY{n}{inf}\PY{p}{)}\PY{o}{/}\PY{n}{np}\PY{o}{.}\PY{n}{linalg}\PY{o}{.}\PY{n}{norm}\PY{p}{(}\PY{n}{x\PYZus{}k}\PY{p}{,} \PY{n+nb}{ord}\PY{o}{=}\PY{n}{np}\PY{o}{.}\PY{n}{inf}\PY{p}{)} \PY{c+c1}{\PYZsh{}Using 2\PYZhy{}norm}
    
    \PY{c+c1}{\PYZsh{}Count iterations}
    \PY{n}{numI} \PY{o}{+}\PY{o}{=} \PY{l+m+mi}{1}
    
    \PY{n}{x\PYZus{}0} \PY{o}{=} \PY{n}{x\PYZus{}k}
\end{Verbatim}
\end{tcolorbox}

    \begin{tcolorbox}[breakable, size=fbox, boxrule=1pt, pad at break*=1mm,colback=cellbackground, colframe=cellborder]
\prompt{In}{incolor}{63}{\boxspacing}
\begin{Verbatim}[commandchars=\\\{\}]
\PY{n+nb}{print}\PY{p}{(}\PY{l+s+s1}{\PYZsq{}}\PY{l+s+s1}{Solution in }\PY{l+s+si}{\PYZpc{}s}\PY{l+s+s1}{ iterations:}\PY{l+s+se}{\PYZbs{}n}\PY{l+s+se}{\PYZbs{}n}\PY{l+s+s1}{ x =}\PY{l+s+s1}{\PYZsq{}} \PY{o}{\PYZpc{}} \PY{n}{numI}\PY{p}{,} \PY{n}{end}\PY{o}{=}\PY{l+s+s1}{\PYZsq{}}\PY{l+s+s1}{ }\PY{l+s+s1}{\PYZsq{}}\PY{p}{)}
\PY{n+nb}{print}\PY{p}{(}\PY{n}{x\PYZus{}0}\PY{p}{,} \PY{n}{end}\PY{o}{=}\PY{l+s+s1}{\PYZsq{}}\PY{l+s+se}{\PYZbs{}n}\PY{l+s+se}{\PYZbs{}n}\PY{l+s+s1}{\PYZsq{}}\PY{p}{)}
\PY{n+nb}{print}\PY{p}{(}\PY{l+s+s1}{\PYZsq{}}\PY{l+s+s1}{Final Error: }\PY{l+s+si}{\PYZpc{}.7f}\PY{l+s+s1}{\PYZsq{}} \PY{o}{\PYZpc{}} \PY{n}{np}\PY{o}{.}\PY{n}{linalg}\PY{o}{.}\PY{n}{norm}\PY{p}{(}\PY{n}{A}\PY{n+nd}{@x\PYZus{}0}\PY{o}{\PYZhy{}}\PY{n}{b}\PY{p}{)}\PY{p}{)}
\end{Verbatim}
\end{tcolorbox}

    \begin{Verbatim}[commandchars=\\\{\}]
Solution in 41 iterations:

 x = [[1.1666665 ]
 [1.20833311]
 [1.45833311]
 [1.45833311]
 [1.20833311]
 [1.1666665 ]]

Final Error: 0.0000006
    \end{Verbatim}

\newpage
    \hypertarget{b.}{%
\subsubsection*{b).}\label{b.}}

We use Gauss-Siedel to approximate the solution. Gauss-Seidel is given
by the following iteration:

\[ \mathbf{x}_{k+1} = -(\mathbf{D}+\mathbf{L})^{-1}\mathbf{U}\mathbf{x}_k + (\mathbf{D}+\mathbf{L})^{-1}\mathbf{b} \]

    \begin{tcolorbox}[breakable, size=fbox, boxrule=1pt, pad at break*=1mm,colback=cellbackground, colframe=cellborder]
\prompt{In}{incolor}{64}{\boxspacing}
\begin{Verbatim}[commandchars=\\\{\}]
\PY{n}{err} \PY{o}{=} \PY{n}{np}\PY{o}{.}\PY{n}{inf} \PY{c+c1}{\PYZsh{}Initial error}
\PY{n}{x\PYZus{}0} \PY{o}{=} \PY{n}{np}\PY{o}{.}\PY{n}{zeros}\PY{p}{(}\PY{n}{b}\PY{o}{.}\PY{n}{shape}\PY{p}{)} \PY{c+c1}{\PYZsh{}Initial solution}

\PY{c+c1}{\PYZsh{}Pre\PYZhy{}compute static matrices}
\PY{n}{D\PYZus{}L\PYZus{}inv} \PY{o}{=} \PY{n}{np}\PY{o}{.}\PY{n}{linalg}\PY{o}{.}\PY{n}{inv}\PY{p}{(}\PY{n}{D}\PY{o}{+}\PY{n}{L}\PY{p}{)}
\PY{n}{D\PYZus{}L\PYZus{}invU} \PY{o}{=} \PY{n}{D\PYZus{}L\PYZus{}inv}\PY{n+nd}{@U}
\PY{n}{D\PYZus{}L\PYZus{}invb} \PY{o}{=} \PY{n}{D\PYZus{}L\PYZus{}inv}\PY{n+nd}{@b}

\PY{n}{numI} \PY{o}{=} \PY{l+m+mi}{0}

\PY{k}{while}\PY{p}{(}\PY{n}{err} \PY{o}{\PYZgt{}}\PY{o}{=} \PY{n}{eps}\PY{p}{)}\PY{p}{:}
    \PY{n}{x\PYZus{}k} \PY{o}{=} \PY{o}{\PYZhy{}}\PY{n}{D\PYZus{}L\PYZus{}invU}\PY{n+nd}{@x\PYZus{}0} \PY{o}{+} \PY{n}{D\PYZus{}L\PYZus{}invb}
    
    \PY{c+c1}{\PYZsh{}Compute error}
    \PY{n}{err} \PY{o}{=} \PY{n}{np}\PY{o}{.}\PY{n}{linalg}\PY{o}{.}\PY{n}{norm}\PY{p}{(}\PY{n}{x\PYZus{}k}\PY{o}{\PYZhy{}}\PY{n}{x\PYZus{}0}\PY{p}{,} \PY{n+nb}{ord}\PY{o}{=}\PY{n}{np}\PY{o}{.}\PY{n}{inf}\PY{p}{)}\PY{o}{/}\PY{n}{np}\PY{o}{.}\PY{n}{linalg}\PY{o}{.}\PY{n}{norm}\PY{p}{(}\PY{n}{x\PYZus{}k}\PY{p}{,} \PY{n+nb}{ord}\PY{o}{=}\PY{n}{np}\PY{o}{.}\PY{n}{inf}\PY{p}{)} \PY{c+c1}{\PYZsh{}Using 2\PYZhy{}norm}
    
    \PY{c+c1}{\PYZsh{}Count iterations}
    \PY{n}{numI} \PY{o}{+}\PY{o}{=} \PY{l+m+mi}{1}
    
    \PY{n}{x\PYZus{}0} \PY{o}{=} \PY{n}{x\PYZus{}k}
\end{Verbatim}
\end{tcolorbox}

    \begin{tcolorbox}[breakable, size=fbox, boxrule=1pt, pad at break*=1mm,colback=cellbackground, colframe=cellborder]
\prompt{In}{incolor}{65}{\boxspacing}
\begin{Verbatim}[commandchars=\\\{\}]
\PY{n+nb}{print}\PY{p}{(}\PY{l+s+s1}{\PYZsq{}}\PY{l+s+s1}{Solution in }\PY{l+s+si}{\PYZpc{}s}\PY{l+s+s1}{ iterations:}\PY{l+s+se}{\PYZbs{}n}\PY{l+s+se}{\PYZbs{}n}\PY{l+s+s1}{ x =}\PY{l+s+s1}{\PYZsq{}} \PY{o}{\PYZpc{}} \PY{n}{numI}\PY{p}{,} \PY{n}{end}\PY{o}{=}\PY{l+s+s1}{\PYZsq{}}\PY{l+s+s1}{ }\PY{l+s+s1}{\PYZsq{}}\PY{p}{)}
\PY{n+nb}{print}\PY{p}{(}\PY{n}{x\PYZus{}0}\PY{p}{,} \PY{n}{end}\PY{o}{=}\PY{l+s+s1}{\PYZsq{}}\PY{l+s+se}{\PYZbs{}n}\PY{l+s+se}{\PYZbs{}n}\PY{l+s+s1}{\PYZsq{}}\PY{p}{)}
\PY{n+nb}{print}\PY{p}{(}\PY{l+s+s1}{\PYZsq{}}\PY{l+s+s1}{Final Error: }\PY{l+s+si}{\PYZpc{}.7f}\PY{l+s+s1}{\PYZsq{}} \PY{o}{\PYZpc{}} \PY{n}{np}\PY{o}{.}\PY{n}{linalg}\PY{o}{.}\PY{n}{norm}\PY{p}{(}\PY{n}{A}\PY{n+nd}{@x\PYZus{}0}\PY{o}{\PYZhy{}}\PY{n}{b}\PY{p}{)}\PY{p}{)}
\end{Verbatim}
\end{tcolorbox}

    \begin{Verbatim}[commandchars=\\\{\}]
Solution in 23 iterations:

 x = [[1.16666658]
 [1.20833324]
 [1.45833325]
 [1.45833326]
 [1.20833327]
 [1.16666663]]

Final Error: 0.0000003
    \end{Verbatim}

\newpage
    \hypertarget{c.}{%
\subsubsection*{c).}\label{c.}}

We use SOR with \(\omega=1.6735\) to approximate the solution. SOR is
given by the following iteration.

\[ \mathbf{x}_{k+1} = (\mathbf{D}-\omega\mathbf{L})^{-1}[(1-\omega)\mathbf{D}+\omega\mathbf{U}]\mathbf{x}_k + \omega(\mathbf{D}-\omega\mathbf{L})^{-1}\mathbf{b} \]

    \begin{tcolorbox}[breakable, size=fbox, boxrule=1pt, pad at break*=1mm,colback=cellbackground, colframe=cellborder]
\prompt{In}{incolor}{66}{\boxspacing}
\begin{Verbatim}[commandchars=\\\{\}]
\PY{c+c1}{\PYZsh{}Only defining this one as a function so we can look at changing omega.}
\PY{k}{def} \PY{n+nf}{SOR}\PY{p}{(}\PY{n}{A}\PY{p}{,} \PY{n}{b}\PY{p}{,} \PY{n}{w}\PY{p}{)}\PY{p}{:}
    \PY{n}{err} \PY{o}{=} \PY{n}{np}\PY{o}{.}\PY{n}{inf} \PY{c+c1}{\PYZsh{}Initial error}
    \PY{n}{x\PYZus{}0} \PY{o}{=} \PY{n}{np}\PY{o}{.}\PY{n}{zeros}\PY{p}{(}\PY{n}{b}\PY{o}{.}\PY{n}{shape}\PY{p}{)} \PY{c+c1}{\PYZsh{}Initial solution}

    \PY{c+c1}{\PYZsh{}Pre\PYZhy{}compute static matrices}
    \PY{n}{T} \PY{o}{=} \PY{n}{np}\PY{o}{.}\PY{n}{linalg}\PY{o}{.}\PY{n}{inv}\PY{p}{(}\PY{n}{D}\PY{o}{+}\PY{n}{w}\PY{o}{*}\PY{n}{L}\PY{p}{)}
    \PY{n}{Z} \PY{o}{=} \PY{n}{T}\PY{o}{@}\PY{p}{(}\PY{p}{(}\PY{l+m+mi}{1}\PY{o}{\PYZhy{}}\PY{n}{w}\PY{p}{)}\PY{o}{*}\PY{n}{D} \PY{o}{\PYZhy{}} \PY{n}{w}\PY{o}{*}\PY{n}{U}\PY{p}{)}
    \PY{n}{Tb} \PY{o}{=} \PY{n}{w}\PY{o}{*}\PY{p}{(}\PY{n}{T}\PY{n+nd}{@b}\PY{p}{)}

    \PY{n}{numI} \PY{o}{=} \PY{l+m+mi}{0} \PY{c+c1}{\PYZsh{}Number of iterations}

    \PY{k}{while}\PY{p}{(}\PY{n}{err} \PY{o}{\PYZgt{}}\PY{o}{=} \PY{n}{eps}\PY{p}{)}\PY{p}{:}
        \PY{n}{x\PYZus{}k} \PY{o}{=} \PY{n}{Z}\PY{n+nd}{@x\PYZus{}0} \PY{o}{+} \PY{n}{Tb}

        \PY{c+c1}{\PYZsh{}Compute error}
        \PY{n}{err} \PY{o}{=} \PY{n}{np}\PY{o}{.}\PY{n}{linalg}\PY{o}{.}\PY{n}{norm}\PY{p}{(}\PY{n}{x\PYZus{}k}\PY{o}{\PYZhy{}}\PY{n}{x\PYZus{}0}\PY{p}{,} \PY{n}{np}\PY{o}{.}\PY{n}{inf}\PY{p}{)}\PY{o}{/}\PY{n}{np}\PY{o}{.}\PY{n}{linalg}\PY{o}{.}\PY{n}{norm}\PY{p}{(}\PY{n}{x\PYZus{}k}\PY{p}{,} \PY{n}{np}\PY{o}{.}\PY{n}{inf}\PY{p}{)} \PY{c+c1}{\PYZsh{}Using 2\PYZhy{}norm}

        \PY{c+c1}{\PYZsh{}Count iterations}
        \PY{n}{numI} \PY{o}{+}\PY{o}{=} \PY{l+m+mi}{1}

        \PY{n}{x\PYZus{}0} \PY{o}{=} \PY{n}{x\PYZus{}k}
        
    \PY{k}{return} \PY{n}{x\PYZus{}0}\PY{p}{,} \PY{n}{numI}
        
\PY{n}{sol}\PY{p}{,} \PY{n}{numI} \PY{o}{=} \PY{n}{SOR}\PY{p}{(}\PY{n}{A}\PY{p}{,} \PY{n}{b}\PY{p}{,} \PY{l+m+mf}{1.6735}\PY{p}{)}
\end{Verbatim}
\end{tcolorbox}

    \begin{tcolorbox}[breakable, size=fbox, boxrule=1pt, pad at break*=1mm,colback=cellbackground, colframe=cellborder]
\prompt{In}{incolor}{67}{\boxspacing}
\begin{Verbatim}[commandchars=\\\{\}]
\PY{n+nb}{print}\PY{p}{(}\PY{l+s+s1}{\PYZsq{}}\PY{l+s+s1}{Solution in }\PY{l+s+si}{\PYZpc{}s}\PY{l+s+s1}{ iterations:}\PY{l+s+se}{\PYZbs{}n}\PY{l+s+se}{\PYZbs{}n}\PY{l+s+s1}{ x =}\PY{l+s+s1}{\PYZsq{}} \PY{o}{\PYZpc{}} \PY{n}{numI}\PY{p}{,} \PY{n}{end}\PY{o}{=}\PY{l+s+s1}{\PYZsq{}}\PY{l+s+s1}{ }\PY{l+s+s1}{\PYZsq{}}\PY{p}{)}
\PY{n+nb}{print}\PY{p}{(}\PY{n}{sol}\PY{p}{,} \PY{n}{end}\PY{o}{=}\PY{l+s+s1}{\PYZsq{}}\PY{l+s+se}{\PYZbs{}n}\PY{l+s+se}{\PYZbs{}n}\PY{l+s+s1}{\PYZsq{}}\PY{p}{)}
\PY{n+nb}{print}\PY{p}{(}\PY{l+s+s1}{\PYZsq{}}\PY{l+s+s1}{Final Error: }\PY{l+s+si}{\PYZpc{}.7f}\PY{l+s+s1}{\PYZsq{}} \PY{o}{\PYZpc{}} \PY{n}{np}\PY{o}{.}\PY{n}{linalg}\PY{o}{.}\PY{n}{norm}\PY{p}{(}\PY{n}{A}\PY{n+nd}{@x\PYZus{}0}\PY{o}{\PYZhy{}}\PY{n}{b}\PY{p}{)}\PY{p}{)}
\end{Verbatim}
\end{tcolorbox}

    \begin{Verbatim}[commandchars=\\\{\}]
Solution in 50 iterations:

 x = [[1.16666664]
 [1.20833334]
 [1.45833336]
 [1.45833327]
 [1.20833335]
 [1.16666666]]

Final Error: 0.0000003
    \end{Verbatim}

    \hypertarget{d.}{%
\subsubsection*{d).}\label{d.}}

We can see that \(\boxed{\text{Gauss-Siedel}}\) has converged the
fastest at 23 iterations -- SOR and Gauss-Jacobi taking 50 and 41
iterations respectively. One should not always expect this to be the
case. In fact, this runs counter to what one might assume as SOR is just
a sped up version of Gauss-Siedel. Although, the important thing
to note is that it is sped up for some choices of \(\omega\). Thus we
expect other choices of \(\omega\) to result in faster convergence for
SOR compared to Gauss-Siedel.

    \hypertarget{e.}{%
\subsubsection*{e).}\label{e.}}

Letting \(c=\rho(\mathbf{B})\) we look at the following error estimate.

\[ ||\mathbf{x}_{k+1}-\mathbf{x}||\leq\frac{c}{1-c}||\mathbf{x}_{k+1}-\mathbf{x}_k|| \]

With this error estimate we can then derive bounds for the last computed
approximations in all three cases above. We note that \(\mathbf{B}\) is the matrix in front of the \(\mathbf{x}_k\) term in all of the iterations. We will also note that because we have convergence we have \(0< \rho(\mathbf{B}) < 1\). Given the general iteration scheme \(\mathbf{x}_{k+1} = \mathbf{Bx}_k + \mathbf{Zb}\) where \(\mathbf{Z}\) is some matrix dependent on the problem (but will not matter here).
\begin{align*}
	&||\mathbf{x}_{k+1}-\mathbf{x}||\leq\frac{c}{1-c}||\mathbf{x}_{k+1}-\mathbf{x}_k|| < ||\mathbf{x}_{k+1}-\mathbf{x}_k||\\
	&\text{The above line following from the bounds on the spectral radius.}\\
	&\cdots=||\mathbf{Bx}_k + \mathbf{Zb} - (\mathbf{Bx}_{k-1} + \mathbf{Zb})|| = ||\mathbf{Bx}_k - \mathbf{Bx}_{k-1}||\\
	&\cdots\leq||\mathbf{B}||\cdot||\mathbf{x}_k-\mathbf{x}_{k-1}||\\
	&\text{Which follows from Cauchy-Schwarz.}\\
	&\text{Repeating this process \(k-1\) more times:}\\
	&||\mathbf{x}_{k+1}-\mathbf{x}||\leq(||\mathbf{B}||)^{k}\cdot|||\mathbf{x}_1-\mathbf{x}_0||
\end{align*}
We can also use the fact that we have chosen \(\mathbf{x}_0=\mathbf{0}\) which further simplifies our bound.
\[\boxed{||\mathbf{x}_{k+1}-\mathbf{x}||\leq(||\mathbf{B}||)^{k}\cdot||\mathbf{Zb}||}\]


\vspace{0.5in}
    \hypertarget{f.}{%
\subsubsection*{f).}\label{f.}}

We can vary the \(\omega\) parameter in SOR to see how that affects the
solution and the convergence time. We will note that the iteration will
converge for a SPD matrix with any initial guess if \(\omega\in(0,2)\)
(\emph{Numerical Analysis 10e}). So we will look at a number of values
for \(\omega\) in this range and examine the number of iterations to
converge.

    \begin{tcolorbox}[breakable, size=fbox, boxrule=1pt, pad at break*=1mm,colback=cellbackground, colframe=cellborder]
\prompt{In}{incolor}{68}{\boxspacing}
\begin{Verbatim}[commandchars=\\\{\}]
\PY{n}{w\PYZus{}vec} \PY{o}{=} \PY{n}{np}\PY{o}{.}\PY{n}{arange}\PY{p}{(}\PY{l+m+mf}{0.1}\PY{p}{,}\PY{l+m+mi}{2}\PY{p}{,}\PY{l+m+mf}{0.1}\PY{p}{)}
\PY{n}{I\PYZus{}vec} \PY{o}{=} \PY{p}{[}\PY{p}{]}
\PY{k}{for} \PY{n}{w} \PY{o+ow}{in} \PY{n}{w\PYZus{}vec}\PY{p}{:}
    \PY{n}{\PYZus{}}\PY{p}{,} \PY{n}{I} \PY{o}{=} \PY{n}{SOR}\PY{p}{(}\PY{n}{A}\PY{p}{,} \PY{n}{b}\PY{p}{,} \PY{n}{w}\PY{p}{)}
    \PY{n}{I\PYZus{}vec}\PY{o}{.}\PY{n}{append}\PY{p}{(}\PY{n}{I}\PY{p}{)}
\end{Verbatim}
\end{tcolorbox}

    \begin{tcolorbox}[breakable, size=fbox, boxrule=1pt, pad at break*=1mm,colback=cellbackground, colframe=cellborder]
\prompt{In}{incolor}{69}{\boxspacing}
\begin{Verbatim}[commandchars=\\\{\}]
\PY{n}{fig}\PY{p}{,} \PY{n}{ax} \PY{o}{=} \PY{n}{plt}\PY{o}{.}\PY{n}{subplots}\PY{p}{(}\PY{l+m+mi}{1}\PY{p}{,}\PY{l+m+mi}{1}\PY{p}{,}\PY{n}{figsize}\PY{o}{=}\PY{p}{(}\PY{l+m+mi}{10}\PY{p}{,}\PY{l+m+mi}{10}\PY{p}{)}\PY{p}{)}
\PY{n}{ax}\PY{o}{.}\PY{n}{scatter}\PY{p}{(}\PY{n}{w\PYZus{}vec}\PY{p}{,} \PY{n}{I\PYZus{}vec}\PY{p}{)}

\PY{n}{ax}\PY{o}{.}\PY{n}{set\PYZus{}title}\PY{p}{(}\PY{l+s+sa}{r}\PY{l+s+s1}{\PYZsq{}}\PY{l+s+s1}{Iterations vs \PYZdl{}}\PY{l+s+s1}{\PYZbs{}}\PY{l+s+s1}{omega\PYZdl{}}\PY{l+s+s1}{\PYZsq{}}\PY{p}{)}
\PY{n}{ax}\PY{o}{.}\PY{n}{set\PYZus{}ylabel}\PY{p}{(}\PY{l+s+s1}{\PYZsq{}}\PY{l+s+s1}{Iterations to Convergence}\PY{l+s+s1}{\PYZsq{}}\PY{p}{)}
\PY{n}{ax}\PY{o}{.}\PY{n}{set\PYZus{}xlabel}\PY{p}{(}\PY{l+s+sa}{r}\PY{l+s+s1}{\PYZsq{}}\PY{l+s+s1}{\PYZdl{}}\PY{l+s+s1}{\PYZbs{}}\PY{l+s+s1}{omega\PYZdl{}}\PY{l+s+s1}{\PYZsq{}}\PY{p}{)}\PY{p}{;}
\PY{n+nb}{print}\PY{p}{(}\PY{l+s+s1}{\PYZsq{}}\PY{l+s+s1}{Minimum: }\PY{l+s+si}{\PYZpc{}f}\PY{l+s+s1}{, Iterations: }\PY{l+s+si}{\PYZpc{}f}\PY{l+s+s1}{\PYZsq{}} \PY{o}{\PYZpc{}} \PY{p}{(}\PY{n}{w\PYZus{}vec}\PY{p}{[}\PY{n}{np}\PY{o}{.}\PY{n}{argmin}\PY{p}{(}\PY{n}{I\PYZus{}vec}\PY{p}{)}\PY{p}{]}\PY{p}{,} \PY{n+nb}{min}\PY{p}{(}\PY{n}{I\PYZus{}vec}\PY{p}{)}\PY{p}{)}\PY{p}{)}
\end{Verbatim}
\end{tcolorbox}

    \begin{Verbatim}[commandchars=\\\{\}]
Minimum: 1.200000, Iterations: 15.000000
    \end{Verbatim}

    \begin{center}
    \adjustimage{max size={0.9\linewidth}{0.9\paperheight}}{output_22_1.png}
    \end{center}
    { \hspace*{\fill} \\}
    
    In the scatter plot above we see the number of iterations to convergence
as a function of the parameter \(\omega\) in SOR. For the omega values
investigated here (and as can be seen in the figure) the minimum is
achieved with \(\omega=1.2\), and converges in 15 iterations. Thus, we
see that altering \(\omega\) can have a dramatic effect on the rate of
convergence in SOR.

    \hypertarget{problem-2}{%
\subsection*{Problem 2}\label{problem-2}}

    We let \(\mathbf{A}\) be a non-singular square matrix of order \(n\),
and \(\mathbf{X}_0\) be an arbitrary square matrix of order \(n\). Then
define the following sequence of matrices:

\[ \mathbf{X}_{k+1} = \mathbf{X}_k + \mathbf{X}_k(\mathbf{I}-\mathbf{AX}_k),\:k=1,2,... \]

    \hypertarget{a.}{%
\subsubsection*{a).}\label{a.}}

We will show that \(\lim_{k\to\infty}\mathbf{X}_k=\mathbf{A}^{-1}\) if
and only if \(\rho(\mathbf{I}-\mathbf{AX}_0)<1\).

\begin{proof}~\\
	Let \(\mathbf{A}\) be a non-singular matrix of order \(n\), and \(\mathbf{X}_0\) be an arbitrary matrix of order \(n\). We begin by considering \(\mathbf{I}-\mathbf{AX}_{k+1}\).
	\begin{align*}
		&\mathbf{I}-\mathbf{AX}_{k+1} = \mathbf{I}-\mathbf{A}(\mathbf{X}_k+\mathbf{X}_k(\mathbf{I}-\mathbf{AX}_k))\\
		&\text{The above follows from plugging in the iteration scheme.}\\
		&\cdots=\mathbf{I}-2\mathbf{AX}_k+(\mathbf{AX}_k)^2=(\mathbf{I}-\mathbf{AX}_k)^2\\
		&\text{We can then repeat the same process to yield...}\\
		&\mathbf{I}-\mathbf{AX}_{k+1} = ((\mathbf{I}-\mathbf{AX}_{k-1})^2)^2\\
		&\text{Continuing this process \(k-1\) more times...}\\
		&\implies \mathbf{I}-\mathbf{AX}_{k+1} = (\mathbf{I}-\mathbf{AX}_{k-1})^{2^{k+1}}
	\end{align*}
	We will now note the following theorem (from class) for a square matrix $\mathbf{B}$:
	\[\lim_{m\to\infty}\mathbf{B}^m = 0 \Leftrightarrow \rho(\mathbf{B})<1\]
	Taking the limit of our expression defined above:
	\[\lim_{k\to\infty}\mathbf{I}-\mathbf{AX}_{k+1} = \lim_{k\to\infty} (\mathbf{I}-\mathbf{AX}_{k-1})^{2^{k+1}}\]
	If we assume that \(\lim_{k\to\infty}\mathbf{X}_k=\mathbf{A}^{-1}\)...
	\[\implies 0 = \lim_{k\to\infty} (\mathbf{I}-\mathbf{AX}_{k-1})^{2^{k+1}}\]
	\[\therefore \rho(\mathbf{I}-\mathbf{AX}_0)<1\]
	If we assume \(\rho(\mathbf{I}-\mathbf{AX}_0)<1\)...
	\begin{align*}
		&\implies \lim_{k\to\infty} (\mathbf{I}-\mathbf{AX}_{k-1})^{2^{k+1}} = 0\\
		&\implies 0 = \mathbf{I} - \mathbf{AX}_{\infty} \implies \mathbf{A}^{-1} = \mathbf{X}_{\infty}\\
		&\therefore \mathbf{X}_{\infty}=\lim_{k\to\infty}\mathbf{X}_k=\mathbf{A}^-1
	\end{align*}
	Having proved the both directions of the statement we can conclude that \(\lim_{k\to\infty}\mathbf{X}_k=\mathbf{A}^{-1}\) if
	and only if \(\rho(\mathbf{I}-\mathbf{AX}_0)<1\).
\end{proof}

\newpage
    \hypertarget{b.}{%
\subsubsection*{b).}\label{b.}}

We will use this iteration (defined above) to compute the inverse of
\(\mathbf{A}\) -- where \(\mathbf{A}\) and \(\mathbf{X}_0\) are defined
in the code below.

    \begin{tcolorbox}[breakable, size=fbox, boxrule=1pt, pad at break*=1mm,colback=cellbackground, colframe=cellborder]
\prompt{In}{incolor}{70}{\boxspacing}
\begin{Verbatim}[commandchars=\\\{\}]
\PY{n}{A} \PY{o}{=} \PY{n}{np}\PY{o}{.}\PY{n}{array}\PY{p}{(}\PY{p}{[}\PY{p}{[}\PY{l+m+mi}{1}\PY{p}{,}\PY{l+m+mi}{1}\PY{p}{]}\PY{p}{,}
             \PY{p}{[}\PY{l+m+mi}{1}\PY{p}{,}\PY{l+m+mi}{2}\PY{p}{]}\PY{p}{]}\PY{p}{)}

\PY{n}{X\PYZus{}0} \PY{o}{=} \PY{n}{np}\PY{o}{.}\PY{n}{array}\PY{p}{(}\PY{p}{[}\PY{p}{[}\PY{l+m+mf}{1.9}\PY{p}{,}\PY{o}{\PYZhy{}}\PY{l+m+mf}{0.9}\PY{p}{]}\PY{p}{,}
                \PY{p}{[}\PY{o}{\PYZhy{}}\PY{l+m+mf}{0.9}\PY{p}{,}\PY{l+m+mf}{0.9}\PY{p}{]}\PY{p}{]}\PY{p}{)}

\PY{n}{A\PYZus{}inv} \PY{o}{=} \PY{n}{np}\PY{o}{.}\PY{n}{array}\PY{p}{(}\PY{p}{[}\PY{p}{[}\PY{l+m+mi}{2}\PY{p}{,}\PY{o}{\PYZhy{}}\PY{l+m+mi}{1}\PY{p}{]}\PY{p}{,}
                 \PY{p}{[}\PY{o}{\PYZhy{}}\PY{l+m+mi}{1}\PY{p}{,}\PY{l+m+mi}{1}\PY{p}{]}\PY{p}{]}\PY{p}{)} \PY{c+c1}{\PYZsh{}True inverse}
\end{Verbatim}
\end{tcolorbox}

    \begin{tcolorbox}[breakable, size=fbox, boxrule=1pt, pad at break*=1mm,colback=cellbackground, colframe=cellborder]
\prompt{In}{incolor}{71}{\boxspacing}
\begin{Verbatim}[commandchars=\\\{\}]
\PY{n}{maxI} \PY{o}{=} \PY{l+m+mi}{100} \PY{c+c1}{\PYZsh{}Max iterations}
\PY{n}{tol} \PY{o}{=} \PY{l+m+mf}{1E\PYZhy{}9} \PY{c+c1}{\PYZsh{}Tolerance}

\PY{n}{X} \PY{o}{=} \PY{n}{X\PYZus{}0} \PY{c+c1}{\PYZsh{}Just so we can have a version of X\PYZus{}0 unchanged}
\PY{n}{I} \PY{o}{=} \PY{n}{np}\PY{o}{.}\PY{n}{eye}\PY{p}{(}\PY{p}{(}\PY{l+m+mi}{2}\PY{p}{)}\PY{p}{)} \PY{c+c1}{\PYZsh{}Identity}

\PY{k}{for} \PY{n}{i} \PY{o+ow}{in} \PY{n+nb}{range}\PY{p}{(}\PY{n}{maxI}\PY{p}{)}\PY{p}{:}
    \PY{n}{X\PYZus{}k} \PY{o}{=} \PY{n}{X} \PY{o}{+} \PY{n}{X}\PY{o}{@}\PY{p}{(}\PY{n}{I} \PY{o}{\PYZhy{}} \PY{n}{A}\PY{n+nd}{@X}\PY{p}{)} \PY{c+c1}{\PYZsh{}Iteration scheme}

    \PY{k}{if} \PY{n}{np}\PY{o}{.}\PY{n}{linalg}\PY{o}{.}\PY{n}{norm}\PY{p}{(}\PY{n}{X\PYZus{}k} \PY{o}{\PYZhy{}} \PY{n}{X}\PY{p}{,} \PY{n+nb}{ord}\PY{o}{=}\PY{n}{np}\PY{o}{.}\PY{n}{inf}\PY{p}{)}\PY{o}{/}\PY{n}{np}\PY{o}{.}\PY{n}{linalg}\PY{o}{.}\PY{n}{norm}\PY{p}{(}\PY{n}{X\PYZus{}k}\PY{p}{,} \PY{n+nb}{ord}\PY{o}{=}\PY{n}{np}\PY{o}{.}\PY{n}{inf}\PY{p}{)} \PY{o}{\PYZlt{}} \PY{n}{tol}\PY{p}{:}
        \PY{n}{X} \PY{o}{=} \PY{n}{X\PYZus{}k}
        \PY{k}{break}

    \PY{n}{X} \PY{o}{=} \PY{n}{X\PYZus{}k}
\end{Verbatim}
\end{tcolorbox}

    \begin{tcolorbox}[breakable, size=fbox, boxrule=1pt, pad at break*=1mm,colback=cellbackground, colframe=cellborder]
\prompt{In}{incolor}{72}{\boxspacing}
\begin{Verbatim}[commandchars=\\\{\}]
\PY{n+nb}{print}\PY{p}{(}\PY{l+s+s1}{\PYZsq{}}\PY{l+s+s1}{Calculated inverse in }\PY{l+s+si}{\PYZpc{}s}\PY{l+s+s1}{ iterations:}\PY{l+s+se}{\PYZbs{}n}\PY{l+s+s1}{\PYZsq{}} \PY{o}{\PYZpc{}} \PY{n}{i}\PY{p}{)}
\PY{n+nb}{print}\PY{p}{(}\PY{n}{X}\PY{p}{)}
\end{Verbatim}
\end{tcolorbox}

    \begin{Verbatim}[commandchars=\\\{\}]
Calculated inverse in 4 iterations:

[[ 2. -1.]
 [-1.  1.]]
    \end{Verbatim}

 We are also interested in the cost for the iteration scheme examined above as compared to the cost of Gaussian Elimination. Gaussian Elimination is asymptotically \(\mathcal{O}(n^3)\), and has a specific cost of \(\frac{2}{3}n^3+\mathcal{O}(n^2)\). We will calculate the cost of our iteration below:
 \begin{align*}
	&\mathbf{AX}_k &\text{\(n^3\) multiplies and \(n^2(n-1)\) adds}\\
	&\mathbf{I}-\mathbf{AX}_k &\text{\(n^2\) adds}\\
	&\mathbf{X}_k(\mathbf{I}-\mathbf{AX}_k) &\text{\(n^3\) multiplies and \(n^2(n-1)\) adds}\\
	&\mathbf{X}_k+\mathbf{X}_k(\mathbf{I}-\mathbf{AX}_k) &\text{\(n^2\) adds}
 \end{align*}
 Adding this all up we can see we have an asymptotic cost of \(\mathcal{O}(n^3)\), but a specific cost of \(4n^3+\mathcal{O}(n^2)\). Thus we can see that the iteration scheme we have considered is asymptotically the same cost as Gaussian Elimination, but it has a larger constant. In reality they are the same cost.

    \hypertarget{problem-3}{%
\subsection*{Problem 3}\label{problem-3}}

    We consider the following linear system where \(a\in\mathcal{R}\):

\[
\begin{bmatrix}
    1 &-a\\
    -a &1
\end{bmatrix}
\mathbf{x} = \mathbf{b}
\]

    Under certain conditions the system above can be solved with the
following iterative method.

\[
\begin{bmatrix}
    1 &0\\
    -\omega a &1
\end{bmatrix}
\mathbf{x}_{k+1}
=
\begin{bmatrix}
    1-\omega &\omega a\\
    0 &1-\omega
\end{bmatrix}
\mathbf{x}_k + \omega\mathbf{b}
\]

    \hypertarget{a.}{%
\subsubsection*{a).}\label{a.}}

We want to know for which values of \(a\) will the iteration converge,
assuming \(\omega=1\). Plugging in this value for \(\omega\) we get the following:

\[
\begin{bmatrix}
1 &0\\
-a &1
\end{bmatrix}
\mathbf{x}_{k+1}
=
\begin{bmatrix}
0 &a\\
0 &0
\end{bmatrix}
\mathbf{x}_k + \mathbf{b}
\]

If we let 
\[\mathbf{A}=\begin{bmatrix}1 &-a\\-a&1\end{bmatrix}\]
Then our iteration equation is of the the following form:
\[(\mathbf{L}+\mathbf{D})\mathbf{x_{k+1}}=-\mathbf{U}\mathbf{x}_k+b\]
Which is clearly just Gauss-Siedel, so we will need \(\rho(-(\mathbf{L}+\mathbf{D})^{-1}\mathbf{U})<1\).
\begin{align*}
	&-(\mathbf{L}+\mathbf{D})^{-1}\mathbf{U} = 
	\begin{bmatrix}
		1 &0\\
		a & 1
	\end{bmatrix}
	\begin{bmatrix}
		0 &a\\0 &0
	\end{bmatrix}
	=
	\begin{bmatrix}
		0 &a\\
		0 &a^2
	\end{bmatrix}
\end{align*}
Taking the determinant of the matrix above minus \(\lambda\mathbf{I}\) we have the following characteristic polynomial.
\[(-\lambda)(a^2-\lambda)=0\]
This tells us our eigenvalues are 0 and \(a^2\). This means we need \(\boxed{|a|<1}\) to ensure \(\rho(-(\mathbf{L}+\mathbf{D})^{-1}\mathbf{U})<1\), giving us convergence of the iteration.

\newpage
    \hypertarget{b.}{%
\subsubsection*{b).}\label{b.}}

For \(a=0.5\) we want to determine which \(\omega\in\{0.8:0.1:1.3\}\)
(that may not be proper set notation, but I think it makes sense), which
minimizes the spectral radius of the following matrix.

\[
\begin{bmatrix}
    1 &0\\
    -\omega a &1
\end{bmatrix}^{-1}
\begin{bmatrix}
    1-\omega &\omega a\\
    0 &1-\omega
\end{bmatrix}
\]

    \begin{tcolorbox}[breakable, size=fbox, boxrule=1pt, pad at break*=1mm,colback=cellbackground, colframe=cellborder]
\prompt{In}{incolor}{73}{\boxspacing}
\begin{Verbatim}[commandchars=\\\{\}]
\PY{n}{a} \PY{o}{=} \PY{l+m+mf}{0.5}
\PY{n}{w\PYZus{}vec} \PY{o}{=} \PY{n}{np}\PY{o}{.}\PY{n}{arange}\PY{p}{(}\PY{l+m+mf}{0.8}\PY{p}{,} \PY{l+m+mf}{1.4}\PY{p}{,} \PY{l+m+mf}{0.1}\PY{p}{)}

\PY{n}{r\PYZus{}min} \PY{o}{=} \PY{n}{np}\PY{o}{.}\PY{n}{inf}
\PY{n}{w\PYZus{}min} \PY{o}{=} \PY{k+kc}{None}

\PY{k}{for} \PY{n}{w} \PY{o+ow}{in} \PY{n}{w\PYZus{}vec}\PY{p}{:}
    \PY{c+c1}{\PYZsh{}Calculate the matrix}
    \PY{n}{A} \PY{o}{=} \PY{n}{np}\PY{o}{.}\PY{n}{array}\PY{p}{(}\PY{p}{[}\PY{p}{[}\PY{l+m+mi}{1}\PY{p}{,}\PY{l+m+mi}{0}\PY{p}{]}\PY{p}{,}\PY{p}{[}\PY{o}{\PYZhy{}}\PY{n}{w}\PY{o}{*}\PY{n}{a}\PY{p}{,} \PY{l+m+mi}{1}\PY{p}{]}\PY{p}{]}\PY{p}{)}
    \PY{n}{B} \PY{o}{=} \PY{n}{np}\PY{o}{.}\PY{n}{array}\PY{p}{(}\PY{p}{[}\PY{p}{[}\PY{l+m+mi}{1}\PY{o}{\PYZhy{}}\PY{n}{w}\PY{p}{,} \PY{n}{w}\PY{o}{*}\PY{n}{a}\PY{p}{]}\PY{p}{,}\PY{p}{[}\PY{l+m+mi}{0}\PY{p}{,}\PY{l+m+mi}{1}\PY{o}{\PYZhy{}}\PY{n}{w}\PY{p}{]}\PY{p}{]}\PY{p}{)}
    
    \PY{n}{C} \PY{o}{=} \PY{n}{np}\PY{o}{.}\PY{n}{linalg}\PY{o}{.}\PY{n}{inv}\PY{p}{(}\PY{n}{A}\PY{p}{)}\PY{n+nd}{@B}
    
    \PY{n}{e\PYZus{}max} \PY{o}{=} \PY{n+nb}{max}\PY{p}{(}\PY{n}{np}\PY{o}{.}\PY{n}{abs}\PY{p}{(}\PY{n}{np}\PY{o}{.}\PY{n}{linalg}\PY{o}{.}\PY{n}{eigvals}\PY{p}{(}\PY{n}{C}\PY{p}{)}\PY{p}{)}\PY{p}{)} \PY{c+c1}{\PYZsh{}Extract max magnitude eigenvalue}
    
    \PY{k}{if} \PY{n}{e\PYZus{}max}\PY{o}{\PYZlt{}}\PY{n}{r\PYZus{}min}\PY{p}{:}
        \PY{n}{r\PYZus{}min} \PY{o}{=} \PY{n}{e\PYZus{}max}
        \PY{n}{w\PYZus{}min} \PY{o}{=} \PY{n}{w}
\end{Verbatim}
\end{tcolorbox}

    \begin{tcolorbox}[breakable, size=fbox, boxrule=1pt, pad at break*=1mm,colback=cellbackground, colframe=cellborder]
\prompt{In}{incolor}{74}{\boxspacing}
\begin{Verbatim}[commandchars=\\\{\}]
\PY{n+nb}{print}\PY{p}{(}\PY{l+s+s1}{\PYZsq{}}\PY{l+s+s1}{Minimum Spectral Radius: }\PY{l+s+si}{\PYZpc{}f}\PY{l+s+s1}{, Minimizing w: }\PY{l+s+si}{\PYZpc{}f}\PY{l+s+s1}{\PYZsq{}} \PY{o}{\PYZpc{}} \PY{p}{(}\PY{n}{r\PYZus{}min}\PY{p}{,} \PY{n}{w\PYZus{}min}\PY{p}{)}\PY{p}{)}
\end{Verbatim}
\end{tcolorbox}

    \begin{Verbatim}[commandchars=\\\{\}]
Minimum Spectral Radius: 0.100000, Minimizing w: 1.100000
    \end{Verbatim}


    % Add a bibliography block to the postdoc
    
    
    
\end{document}
