\documentclass[11pt]{article}

    \usepackage[breakable]{tcolorbox}
    \usepackage{parskip} % Stop auto-indenting (to mimic markdown behaviour)
    
    \usepackage{iftex}
    \ifPDFTeX
    	\usepackage[T1]{fontenc}
    	\usepackage{mathpazo}
    \else
    	\usepackage{fontspec}
    \fi

    % Basic figure setup, for now with no caption control since it's done
    % automatically by Pandoc (which extracts ![](path) syntax from Markdown).
    \usepackage{graphicx}
    % Maintain compatibility with old templates. Remove in nbconvert 6.0
    \let\Oldincludegraphics\includegraphics
    % Ensure that by default, figures have no caption (until we provide a
    % proper Figure object with a Caption API and a way to capture that
    % in the conversion process - todo).
    \usepackage{caption}
    \DeclareCaptionFormat{nocaption}{}
    \captionsetup{format=nocaption,aboveskip=0pt,belowskip=0pt}

    \usepackage{float}
    \floatplacement{figure}{H} % forces figures to be placed at the correct location
    \usepackage{xcolor} % Allow colors to be defined
    \usepackage{enumerate} % Needed for markdown enumerations to work
    \usepackage{geometry} % Used to adjust the document margins
    \usepackage{amsmath} % Equations
    \usepackage{amssymb} % Equations
    \usepackage{textcomp} % defines textquotesingle
    % Hack from http://tex.stackexchange.com/a/47451/13684:
    \AtBeginDocument{%
        \def\PYZsq{\textquotesingle}% Upright quotes in Pygmentized code
    }
    \usepackage{upquote} % Upright quotes for verbatim code
    \usepackage{eurosym} % defines \euro
    \usepackage[mathletters]{ucs} % Extended unicode (utf-8) support
    \usepackage{fancyvrb} % verbatim replacement that allows latex
    \usepackage{grffile} % extends the file name processing of package graphics 
                         % to support a larger range
    \makeatletter % fix for old versions of grffile with XeLaTeX
    \@ifpackagelater{grffile}{2019/11/01}
    {
      % Do nothing on new versions
    }
    {
      \def\Gread@@xetex#1{%
        \IfFileExists{"\Gin@base".bb}%
        {\Gread@eps{\Gin@base.bb}}%
        {\Gread@@xetex@aux#1}%
      }
    }
    \makeatother
    \usepackage[Export]{adjustbox} % Used to constrain images to a maximum size
    \adjustboxset{max size={0.9\linewidth}{0.9\paperheight}}

    % The hyperref package gives us a pdf with properly built
    % internal navigation ('pdf bookmarks' for the table of contents,
    % internal cross-reference links, web links for URLs, etc.)
    \usepackage{hyperref}
    % The default LaTeX title has an obnoxious amount of whitespace. By default,
    % titling removes some of it. It also provides customization options.
    \usepackage{titling}
    \usepackage{longtable} % longtable support required by pandoc >1.10
    \usepackage{booktabs}  % table support for pandoc > 1.12.2
    \usepackage[inline]{enumitem} % IRkernel/repr support (it uses the enumerate* environment)
    \usepackage[normalem]{ulem} % ulem is needed to support strikethroughs (\sout)
                                % normalem makes italics be italics, not underlines
    \usepackage{mathrsfs}
    

    
    % Colors for the hyperref package
    \definecolor{urlcolor}{rgb}{0,.145,.698}
    \definecolor{linkcolor}{rgb}{.71,0.21,0.01}
    \definecolor{citecolor}{rgb}{.12,.54,.11}

    % ANSI colors
    \definecolor{ansi-black}{HTML}{3E424D}
    \definecolor{ansi-black-intense}{HTML}{282C36}
    \definecolor{ansi-red}{HTML}{E75C58}
    \definecolor{ansi-red-intense}{HTML}{B22B31}
    \definecolor{ansi-green}{HTML}{00A250}
    \definecolor{ansi-green-intense}{HTML}{007427}
    \definecolor{ansi-yellow}{HTML}{DDB62B}
    \definecolor{ansi-yellow-intense}{HTML}{B27D12}
    \definecolor{ansi-blue}{HTML}{208FFB}
    \definecolor{ansi-blue-intense}{HTML}{0065CA}
    \definecolor{ansi-magenta}{HTML}{D160C4}
    \definecolor{ansi-magenta-intense}{HTML}{A03196}
    \definecolor{ansi-cyan}{HTML}{60C6C8}
    \definecolor{ansi-cyan-intense}{HTML}{258F8F}
    \definecolor{ansi-white}{HTML}{C5C1B4}
    \definecolor{ansi-white-intense}{HTML}{A1A6B2}
    \definecolor{ansi-default-inverse-fg}{HTML}{FFFFFF}
    \definecolor{ansi-default-inverse-bg}{HTML}{000000}

    % common color for the border for error outputs.
    \definecolor{outerrorbackground}{HTML}{FFDFDF}

    % commands and environments needed by pandoc snippets
    % extracted from the output of `pandoc -s`
    \providecommand{\tightlist}{%
      \setlength{\itemsep}{0pt}\setlength{\parskip}{0pt}}
    \DefineVerbatimEnvironment{Highlighting}{Verbatim}{commandchars=\\\{\}}
    % Add ',fontsize=\small' for more characters per line
    \newenvironment{Shaded}{}{}
    \newcommand{\KeywordTok}[1]{\textcolor[rgb]{0.00,0.44,0.13}{\textbf{{#1}}}}
    \newcommand{\DataTypeTok}[1]{\textcolor[rgb]{0.56,0.13,0.00}{{#1}}}
    \newcommand{\DecValTok}[1]{\textcolor[rgb]{0.25,0.63,0.44}{{#1}}}
    \newcommand{\BaseNTok}[1]{\textcolor[rgb]{0.25,0.63,0.44}{{#1}}}
    \newcommand{\FloatTok}[1]{\textcolor[rgb]{0.25,0.63,0.44}{{#1}}}
    \newcommand{\CharTok}[1]{\textcolor[rgb]{0.25,0.44,0.63}{{#1}}}
    \newcommand{\StringTok}[1]{\textcolor[rgb]{0.25,0.44,0.63}{{#1}}}
    \newcommand{\CommentTok}[1]{\textcolor[rgb]{0.38,0.63,0.69}{\textit{{#1}}}}
    \newcommand{\OtherTok}[1]{\textcolor[rgb]{0.00,0.44,0.13}{{#1}}}
    \newcommand{\AlertTok}[1]{\textcolor[rgb]{1.00,0.00,0.00}{\textbf{{#1}}}}
    \newcommand{\FunctionTok}[1]{\textcolor[rgb]{0.02,0.16,0.49}{{#1}}}
    \newcommand{\RegionMarkerTok}[1]{{#1}}
    \newcommand{\ErrorTok}[1]{\textcolor[rgb]{1.00,0.00,0.00}{\textbf{{#1}}}}
    \newcommand{\NormalTok}[1]{{#1}}
    
    % Additional commands for more recent versions of Pandoc
    \newcommand{\ConstantTok}[1]{\textcolor[rgb]{0.53,0.00,0.00}{{#1}}}
    \newcommand{\SpecialCharTok}[1]{\textcolor[rgb]{0.25,0.44,0.63}{{#1}}}
    \newcommand{\VerbatimStringTok}[1]{\textcolor[rgb]{0.25,0.44,0.63}{{#1}}}
    \newcommand{\SpecialStringTok}[1]{\textcolor[rgb]{0.73,0.40,0.53}{{#1}}}
    \newcommand{\ImportTok}[1]{{#1}}
    \newcommand{\DocumentationTok}[1]{\textcolor[rgb]{0.73,0.13,0.13}{\textit{{#1}}}}
    \newcommand{\AnnotationTok}[1]{\textcolor[rgb]{0.38,0.63,0.69}{\textbf{\textit{{#1}}}}}
    \newcommand{\CommentVarTok}[1]{\textcolor[rgb]{0.38,0.63,0.69}{\textbf{\textit{{#1}}}}}
    \newcommand{\VariableTok}[1]{\textcolor[rgb]{0.10,0.09,0.49}{{#1}}}
    \newcommand{\ControlFlowTok}[1]{\textcolor[rgb]{0.00,0.44,0.13}{\textbf{{#1}}}}
    \newcommand{\OperatorTok}[1]{\textcolor[rgb]{0.40,0.40,0.40}{{#1}}}
    \newcommand{\BuiltInTok}[1]{{#1}}
    \newcommand{\ExtensionTok}[1]{{#1}}
    \newcommand{\PreprocessorTok}[1]{\textcolor[rgb]{0.74,0.48,0.00}{{#1}}}
    \newcommand{\AttributeTok}[1]{\textcolor[rgb]{0.49,0.56,0.16}{{#1}}}
    \newcommand{\InformationTok}[1]{\textcolor[rgb]{0.38,0.63,0.69}{\textbf{\textit{{#1}}}}}
    \newcommand{\WarningTok}[1]{\textcolor[rgb]{0.38,0.63,0.69}{\textbf{\textit{{#1}}}}}
    
    
    % Define a nice break command that doesn't care if a line doesn't already
    % exist.
    \def\br{\hspace*{\fill} \\* }
    % Math Jax compatibility definitions
    \def\gt{>}
    \def\lt{<}
    \let\Oldtex\TeX
    \let\Oldlatex\LaTeX
    \renewcommand{\TeX}{\textrm{\Oldtex}}
    \renewcommand{\LaTeX}{\textrm{\Oldlatex}}
    % Document parameters
    % Document title
    \title{HW06}
    
    
    
    
    
% Pygments definitions
\makeatletter
\def\PY@reset{\let\PY@it=\relax \let\PY@bf=\relax%
    \let\PY@ul=\relax \let\PY@tc=\relax%
    \let\PY@bc=\relax \let\PY@ff=\relax}
\def\PY@tok#1{\csname PY@tok@#1\endcsname}
\def\PY@toks#1+{\ifx\relax#1\empty\else%
    \PY@tok{#1}\expandafter\PY@toks\fi}
\def\PY@do#1{\PY@bc{\PY@tc{\PY@ul{%
    \PY@it{\PY@bf{\PY@ff{#1}}}}}}}
\def\PY#1#2{\PY@reset\PY@toks#1+\relax+\PY@do{#2}}

\expandafter\def\csname PY@tok@w\endcsname{\def\PY@tc##1{\textcolor[rgb]{0.73,0.73,0.73}{##1}}}
\expandafter\def\csname PY@tok@c\endcsname{\let\PY@it=\textit\def\PY@tc##1{\textcolor[rgb]{0.25,0.50,0.50}{##1}}}
\expandafter\def\csname PY@tok@cp\endcsname{\def\PY@tc##1{\textcolor[rgb]{0.74,0.48,0.00}{##1}}}
\expandafter\def\csname PY@tok@k\endcsname{\let\PY@bf=\textbf\def\PY@tc##1{\textcolor[rgb]{0.00,0.50,0.00}{##1}}}
\expandafter\def\csname PY@tok@kp\endcsname{\def\PY@tc##1{\textcolor[rgb]{0.00,0.50,0.00}{##1}}}
\expandafter\def\csname PY@tok@kt\endcsname{\def\PY@tc##1{\textcolor[rgb]{0.69,0.00,0.25}{##1}}}
\expandafter\def\csname PY@tok@o\endcsname{\def\PY@tc##1{\textcolor[rgb]{0.40,0.40,0.40}{##1}}}
\expandafter\def\csname PY@tok@ow\endcsname{\let\PY@bf=\textbf\def\PY@tc##1{\textcolor[rgb]{0.67,0.13,1.00}{##1}}}
\expandafter\def\csname PY@tok@nb\endcsname{\def\PY@tc##1{\textcolor[rgb]{0.00,0.50,0.00}{##1}}}
\expandafter\def\csname PY@tok@nf\endcsname{\def\PY@tc##1{\textcolor[rgb]{0.00,0.00,1.00}{##1}}}
\expandafter\def\csname PY@tok@nc\endcsname{\let\PY@bf=\textbf\def\PY@tc##1{\textcolor[rgb]{0.00,0.00,1.00}{##1}}}
\expandafter\def\csname PY@tok@nn\endcsname{\let\PY@bf=\textbf\def\PY@tc##1{\textcolor[rgb]{0.00,0.00,1.00}{##1}}}
\expandafter\def\csname PY@tok@ne\endcsname{\let\PY@bf=\textbf\def\PY@tc##1{\textcolor[rgb]{0.82,0.25,0.23}{##1}}}
\expandafter\def\csname PY@tok@nv\endcsname{\def\PY@tc##1{\textcolor[rgb]{0.10,0.09,0.49}{##1}}}
\expandafter\def\csname PY@tok@no\endcsname{\def\PY@tc##1{\textcolor[rgb]{0.53,0.00,0.00}{##1}}}
\expandafter\def\csname PY@tok@nl\endcsname{\def\PY@tc##1{\textcolor[rgb]{0.63,0.63,0.00}{##1}}}
\expandafter\def\csname PY@tok@ni\endcsname{\let\PY@bf=\textbf\def\PY@tc##1{\textcolor[rgb]{0.60,0.60,0.60}{##1}}}
\expandafter\def\csname PY@tok@na\endcsname{\def\PY@tc##1{\textcolor[rgb]{0.49,0.56,0.16}{##1}}}
\expandafter\def\csname PY@tok@nt\endcsname{\let\PY@bf=\textbf\def\PY@tc##1{\textcolor[rgb]{0.00,0.50,0.00}{##1}}}
\expandafter\def\csname PY@tok@nd\endcsname{\def\PY@tc##1{\textcolor[rgb]{0.67,0.13,1.00}{##1}}}
\expandafter\def\csname PY@tok@s\endcsname{\def\PY@tc##1{\textcolor[rgb]{0.73,0.13,0.13}{##1}}}
\expandafter\def\csname PY@tok@sd\endcsname{\let\PY@it=\textit\def\PY@tc##1{\textcolor[rgb]{0.73,0.13,0.13}{##1}}}
\expandafter\def\csname PY@tok@si\endcsname{\let\PY@bf=\textbf\def\PY@tc##1{\textcolor[rgb]{0.73,0.40,0.53}{##1}}}
\expandafter\def\csname PY@tok@se\endcsname{\let\PY@bf=\textbf\def\PY@tc##1{\textcolor[rgb]{0.73,0.40,0.13}{##1}}}
\expandafter\def\csname PY@tok@sr\endcsname{\def\PY@tc##1{\textcolor[rgb]{0.73,0.40,0.53}{##1}}}
\expandafter\def\csname PY@tok@ss\endcsname{\def\PY@tc##1{\textcolor[rgb]{0.10,0.09,0.49}{##1}}}
\expandafter\def\csname PY@tok@sx\endcsname{\def\PY@tc##1{\textcolor[rgb]{0.00,0.50,0.00}{##1}}}
\expandafter\def\csname PY@tok@m\endcsname{\def\PY@tc##1{\textcolor[rgb]{0.40,0.40,0.40}{##1}}}
\expandafter\def\csname PY@tok@gh\endcsname{\let\PY@bf=\textbf\def\PY@tc##1{\textcolor[rgb]{0.00,0.00,0.50}{##1}}}
\expandafter\def\csname PY@tok@gu\endcsname{\let\PY@bf=\textbf\def\PY@tc##1{\textcolor[rgb]{0.50,0.00,0.50}{##1}}}
\expandafter\def\csname PY@tok@gd\endcsname{\def\PY@tc##1{\textcolor[rgb]{0.63,0.00,0.00}{##1}}}
\expandafter\def\csname PY@tok@gi\endcsname{\def\PY@tc##1{\textcolor[rgb]{0.00,0.63,0.00}{##1}}}
\expandafter\def\csname PY@tok@gr\endcsname{\def\PY@tc##1{\textcolor[rgb]{1.00,0.00,0.00}{##1}}}
\expandafter\def\csname PY@tok@ge\endcsname{\let\PY@it=\textit}
\expandafter\def\csname PY@tok@gs\endcsname{\let\PY@bf=\textbf}
\expandafter\def\csname PY@tok@gp\endcsname{\let\PY@bf=\textbf\def\PY@tc##1{\textcolor[rgb]{0.00,0.00,0.50}{##1}}}
\expandafter\def\csname PY@tok@go\endcsname{\def\PY@tc##1{\textcolor[rgb]{0.53,0.53,0.53}{##1}}}
\expandafter\def\csname PY@tok@gt\endcsname{\def\PY@tc##1{\textcolor[rgb]{0.00,0.27,0.87}{##1}}}
\expandafter\def\csname PY@tok@err\endcsname{\def\PY@bc##1{\setlength{\fboxsep}{0pt}\fcolorbox[rgb]{1.00,0.00,0.00}{1,1,1}{\strut ##1}}}
\expandafter\def\csname PY@tok@kc\endcsname{\let\PY@bf=\textbf\def\PY@tc##1{\textcolor[rgb]{0.00,0.50,0.00}{##1}}}
\expandafter\def\csname PY@tok@kd\endcsname{\let\PY@bf=\textbf\def\PY@tc##1{\textcolor[rgb]{0.00,0.50,0.00}{##1}}}
\expandafter\def\csname PY@tok@kn\endcsname{\let\PY@bf=\textbf\def\PY@tc##1{\textcolor[rgb]{0.00,0.50,0.00}{##1}}}
\expandafter\def\csname PY@tok@kr\endcsname{\let\PY@bf=\textbf\def\PY@tc##1{\textcolor[rgb]{0.00,0.50,0.00}{##1}}}
\expandafter\def\csname PY@tok@bp\endcsname{\def\PY@tc##1{\textcolor[rgb]{0.00,0.50,0.00}{##1}}}
\expandafter\def\csname PY@tok@fm\endcsname{\def\PY@tc##1{\textcolor[rgb]{0.00,0.00,1.00}{##1}}}
\expandafter\def\csname PY@tok@vc\endcsname{\def\PY@tc##1{\textcolor[rgb]{0.10,0.09,0.49}{##1}}}
\expandafter\def\csname PY@tok@vg\endcsname{\def\PY@tc##1{\textcolor[rgb]{0.10,0.09,0.49}{##1}}}
\expandafter\def\csname PY@tok@vi\endcsname{\def\PY@tc##1{\textcolor[rgb]{0.10,0.09,0.49}{##1}}}
\expandafter\def\csname PY@tok@vm\endcsname{\def\PY@tc##1{\textcolor[rgb]{0.10,0.09,0.49}{##1}}}
\expandafter\def\csname PY@tok@sa\endcsname{\def\PY@tc##1{\textcolor[rgb]{0.73,0.13,0.13}{##1}}}
\expandafter\def\csname PY@tok@sb\endcsname{\def\PY@tc##1{\textcolor[rgb]{0.73,0.13,0.13}{##1}}}
\expandafter\def\csname PY@tok@sc\endcsname{\def\PY@tc##1{\textcolor[rgb]{0.73,0.13,0.13}{##1}}}
\expandafter\def\csname PY@tok@dl\endcsname{\def\PY@tc##1{\textcolor[rgb]{0.73,0.13,0.13}{##1}}}
\expandafter\def\csname PY@tok@s2\endcsname{\def\PY@tc##1{\textcolor[rgb]{0.73,0.13,0.13}{##1}}}
\expandafter\def\csname PY@tok@sh\endcsname{\def\PY@tc##1{\textcolor[rgb]{0.73,0.13,0.13}{##1}}}
\expandafter\def\csname PY@tok@s1\endcsname{\def\PY@tc##1{\textcolor[rgb]{0.73,0.13,0.13}{##1}}}
\expandafter\def\csname PY@tok@mb\endcsname{\def\PY@tc##1{\textcolor[rgb]{0.40,0.40,0.40}{##1}}}
\expandafter\def\csname PY@tok@mf\endcsname{\def\PY@tc##1{\textcolor[rgb]{0.40,0.40,0.40}{##1}}}
\expandafter\def\csname PY@tok@mh\endcsname{\def\PY@tc##1{\textcolor[rgb]{0.40,0.40,0.40}{##1}}}
\expandafter\def\csname PY@tok@mi\endcsname{\def\PY@tc##1{\textcolor[rgb]{0.40,0.40,0.40}{##1}}}
\expandafter\def\csname PY@tok@il\endcsname{\def\PY@tc##1{\textcolor[rgb]{0.40,0.40,0.40}{##1}}}
\expandafter\def\csname PY@tok@mo\endcsname{\def\PY@tc##1{\textcolor[rgb]{0.40,0.40,0.40}{##1}}}
\expandafter\def\csname PY@tok@ch\endcsname{\let\PY@it=\textit\def\PY@tc##1{\textcolor[rgb]{0.25,0.50,0.50}{##1}}}
\expandafter\def\csname PY@tok@cm\endcsname{\let\PY@it=\textit\def\PY@tc##1{\textcolor[rgb]{0.25,0.50,0.50}{##1}}}
\expandafter\def\csname PY@tok@cpf\endcsname{\let\PY@it=\textit\def\PY@tc##1{\textcolor[rgb]{0.25,0.50,0.50}{##1}}}
\expandafter\def\csname PY@tok@c1\endcsname{\let\PY@it=\textit\def\PY@tc##1{\textcolor[rgb]{0.25,0.50,0.50}{##1}}}
\expandafter\def\csname PY@tok@cs\endcsname{\let\PY@it=\textit\def\PY@tc##1{\textcolor[rgb]{0.25,0.50,0.50}{##1}}}

\def\PYZbs{\char`\\}
\def\PYZus{\char`\_}
\def\PYZob{\char`\{}
\def\PYZcb{\char`\}}
\def\PYZca{\char`\^}
\def\PYZam{\char`\&}
\def\PYZlt{\char`\<}
\def\PYZgt{\char`\>}
\def\PYZsh{\char`\#}
\def\PYZpc{\char`\%}
\def\PYZdl{\char`\$}
\def\PYZhy{\char`\-}
\def\PYZsq{\char`\'}
\def\PYZdq{\char`\"}
\def\PYZti{\char`\~}
% for compatibility with earlier versions
\def\PYZat{@}
\def\PYZlb{[}
\def\PYZrb{]}
\makeatother


    % For linebreaks inside Verbatim environment from package fancyvrb. 
    \makeatletter
        \newbox\Wrappedcontinuationbox 
        \newbox\Wrappedvisiblespacebox 
        \newcommand*\Wrappedvisiblespace {\textcolor{red}{\textvisiblespace}} 
        \newcommand*\Wrappedcontinuationsymbol {\textcolor{red}{\llap{\tiny$\m@th\hookrightarrow$}}} 
        \newcommand*\Wrappedcontinuationindent {3ex } 
        \newcommand*\Wrappedafterbreak {\kern\Wrappedcontinuationindent\copy\Wrappedcontinuationbox} 
        % Take advantage of the already applied Pygments mark-up to insert 
        % potential linebreaks for TeX processing. 
        %        {, <, #, %, $, ' and ": go to next line. 
        %        _, }, ^, &, >, - and ~: stay at end of broken line. 
        % Use of \textquotesingle for straight quote. 
        \newcommand*\Wrappedbreaksatspecials {% 
            \def\PYGZus{\discretionary{\char`\_}{\Wrappedafterbreak}{\char`\_}}% 
            \def\PYGZob{\discretionary{}{\Wrappedafterbreak\char`\{}{\char`\{}}% 
            \def\PYGZcb{\discretionary{\char`\}}{\Wrappedafterbreak}{\char`\}}}% 
            \def\PYGZca{\discretionary{\char`\^}{\Wrappedafterbreak}{\char`\^}}% 
            \def\PYGZam{\discretionary{\char`\&}{\Wrappedafterbreak}{\char`\&}}% 
            \def\PYGZlt{\discretionary{}{\Wrappedafterbreak\char`\<}{\char`\<}}% 
            \def\PYGZgt{\discretionary{\char`\>}{\Wrappedafterbreak}{\char`\>}}% 
            \def\PYGZsh{\discretionary{}{\Wrappedafterbreak\char`\#}{\char`\#}}% 
            \def\PYGZpc{\discretionary{}{\Wrappedafterbreak\char`\%}{\char`\%}}% 
            \def\PYGZdl{\discretionary{}{\Wrappedafterbreak\char`\$}{\char`\$}}% 
            \def\PYGZhy{\discretionary{\char`\-}{\Wrappedafterbreak}{\char`\-}}% 
            \def\PYGZsq{\discretionary{}{\Wrappedafterbreak\textquotesingle}{\textquotesingle}}% 
            \def\PYGZdq{\discretionary{}{\Wrappedafterbreak\char`\"}{\char`\"}}% 
            \def\PYGZti{\discretionary{\char`\~}{\Wrappedafterbreak}{\char`\~}}% 
        } 
        % Some characters . , ; ? ! / are not pygmentized. 
        % This macro makes them "active" and they will insert potential linebreaks 
        \newcommand*\Wrappedbreaksatpunct {% 
            \lccode`\~`\.\lowercase{\def~}{\discretionary{\hbox{\char`\.}}{\Wrappedafterbreak}{\hbox{\char`\.}}}% 
            \lccode`\~`\,\lowercase{\def~}{\discretionary{\hbox{\char`\,}}{\Wrappedafterbreak}{\hbox{\char`\,}}}% 
            \lccode`\~`\;\lowercase{\def~}{\discretionary{\hbox{\char`\;}}{\Wrappedafterbreak}{\hbox{\char`\;}}}% 
            \lccode`\~`\:\lowercase{\def~}{\discretionary{\hbox{\char`\:}}{\Wrappedafterbreak}{\hbox{\char`\:}}}% 
            \lccode`\~`\?\lowercase{\def~}{\discretionary{\hbox{\char`\?}}{\Wrappedafterbreak}{\hbox{\char`\?}}}% 
            \lccode`\~`\!\lowercase{\def~}{\discretionary{\hbox{\char`\!}}{\Wrappedafterbreak}{\hbox{\char`\!}}}% 
            \lccode`\~`\/\lowercase{\def~}{\discretionary{\hbox{\char`\/}}{\Wrappedafterbreak}{\hbox{\char`\/}}}% 
            \catcode`\.\active
            \catcode`\,\active 
            \catcode`\;\active
            \catcode`\:\active
            \catcode`\?\active
            \catcode`\!\active
            \catcode`\/\active 
            \lccode`\~`\~ 	
        }
    \makeatother

    \let\OriginalVerbatim=\Verbatim
    \makeatletter
    \renewcommand{\Verbatim}[1][1]{%
        %\parskip\z@skip
        \sbox\Wrappedcontinuationbox {\Wrappedcontinuationsymbol}%
        \sbox\Wrappedvisiblespacebox {\FV@SetupFont\Wrappedvisiblespace}%
        \def\FancyVerbFormatLine ##1{\hsize\linewidth
            \vtop{\raggedright\hyphenpenalty\z@\exhyphenpenalty\z@
                \doublehyphendemerits\z@\finalhyphendemerits\z@
                \strut ##1\strut}%
        }%
        % If the linebreak is at a space, the latter will be displayed as visible
        % space at end of first line, and a continuation symbol starts next line.
        % Stretch/shrink are however usually zero for typewriter font.
        \def\FV@Space {%
            \nobreak\hskip\z@ plus\fontdimen3\font minus\fontdimen4\font
            \discretionary{\copy\Wrappedvisiblespacebox}{\Wrappedafterbreak}
            {\kern\fontdimen2\font}%
        }%
        
        % Allow breaks at special characters using \PYG... macros.
        \Wrappedbreaksatspecials
        % Breaks at punctuation characters . , ; ? ! and / need catcode=\active 	
        \OriginalVerbatim[#1,codes*=\Wrappedbreaksatpunct]%
    }
    \makeatother

    % Exact colors from NB
    \definecolor{incolor}{HTML}{303F9F}
    \definecolor{outcolor}{HTML}{D84315}
    \definecolor{cellborder}{HTML}{CFCFCF}
    \definecolor{cellbackground}{HTML}{F7F7F7}
    
    % prompt
    \makeatletter
    \newcommand{\boxspacing}{\kern\kvtcb@left@rule\kern\kvtcb@boxsep}
    \makeatother
    \newcommand{\prompt}[4]{
        {\ttfamily\llap{{\color{#2}[#3]:\hspace{3pt}#4}}\vspace{-\baselineskip}}
    }
    

    
    % Prevent overflowing lines due to hard-to-break entities
    \sloppy 
    % Setup hyperref package
    \hypersetup{
      breaklinks=true,  % so long urls are correctly broken across lines
      colorlinks=true,
      urlcolor=urlcolor,
      linkcolor=linkcolor,
      citecolor=citecolor,
      }
    % Slightly bigger margins than the latex defaults
    
    \geometry{verbose,tmargin=1in,bmargin=1in,lmargin=1in,rmargin=1in}
    
    

\begin{document}
    
    %\maketitle
    
    

    
    \hypertarget{setup}{%
\subsection{Setup}\label{setup}}

    \begin{tcolorbox}[breakable, size=fbox, boxrule=1pt, pad at break*=1mm,colback=cellbackground, colframe=cellborder]
\prompt{In}{incolor}{89}{\boxspacing}
\begin{Verbatim}[commandchars=\\\{\}]
\PY{k+kn}{import} \PY{n+nn}{numpy} \PY{k}{as} \PY{n+nn}{np}
\PY{k+kn}{import} \PY{n+nn}{pandas} \PY{k}{as} \PY{n+nn}{pd}
\PY{k+kn}{import} \PY{n+nn}{matplotlib}\PY{n+nn}{.}\PY{n+nn}{pyplot} \PY{k}{as} \PY{n+nn}{plt}
\PY{k+kn}{from} \PY{n+nn}{IPython}\PY{n+nn}{.}\PY{n+nn}{display} \PY{k+kn}{import} \PY{n}{display}
\PY{k+kn}{import} \PY{n+nn}{seaborn} \PY{k}{as} \PY{n+nn}{sns}
\PY{k+kn}{from} \PY{n+nn}{scipy}\PY{n+nn}{.}\PY{n+nn}{sparse}\PY{n+nn}{.}\PY{n+nn}{linalg} \PY{k+kn}{import} \PY{n}{cg}

\PY{n}{sns}\PY{o}{.}\PY{n}{set}\PY{p}{(}\PY{p}{)}
\end{Verbatim}
\end{tcolorbox}

    \hypertarget{problem-2}{%
\subsection{Problem 2}\label{problem-2}}

    We consider a sparse \(500\times 500\) matrix \(\mathbf{A}\) built with
the following process:

\begin{itemize}
\tightlist
\item
  1 in each diagonal entry
\item
  Each off diagonal entry \(\sim U(-1,1)\)
\item
  Replace each off diagonal entry with \(|a_{ij}| > \tau\) by 0
\item
  $\tau = $ 0.01, 0.05, 0.1, 0.2
\end{itemize}

This results in four matrices \(\mathbf{A}_k\), one for each value of
\(\tau\). We then consider the right hand side to be
\(\mathbf{b}\sim U(-1,1)\).

    \begin{tcolorbox}[breakable, size=fbox, boxrule=1pt, pad at break*=1mm,colback=cellbackground, colframe=cellborder]
\prompt{In}{incolor}{90}{\boxspacing}
\begin{Verbatim}[commandchars=\\\{\}]
\PY{n}{A} \PY{o}{=} \PY{p}{[}\PY{p}{]} \PY{c+c1}{\PYZsh{}List of our constructed matrices}
\PY{n}{tau} \PY{o}{=} \PY{p}{[}\PY{l+m+mf}{0.01}\PY{p}{,} \PY{l+m+mf}{0.05}\PY{p}{,} \PY{l+m+mf}{0.1}\PY{p}{,} \PY{l+m+mf}{0.2}\PY{p}{]}

\PY{n}{np}\PY{o}{.}\PY{n}{random}\PY{o}{.}\PY{n}{seed}\PY{p}{(}\PY{l+m+mi}{0}\PY{p}{)}

\PY{k}{for} \PY{n}{k} \PY{o+ow}{in} \PY{n+nb}{range}\PY{p}{(}\PY{n+nb}{len}\PY{p}{(}\PY{n}{tau}\PY{p}{)}\PY{p}{)}\PY{p}{:}
    \PY{c+c1}{\PYZsh{}Create a diagonal matrix first}
    \PY{n}{mat} \PY{o}{=} \PY{n}{np}\PY{o}{.}\PY{n}{diag}\PY{p}{(}\PY{n}{np}\PY{o}{.}\PY{n}{ones}\PY{p}{(}\PY{l+m+mi}{500}\PY{p}{)}\PY{p}{)}
    
    \PY{k}{for} \PY{n}{i} \PY{o+ow}{in} \PY{n+nb}{range}\PY{p}{(}\PY{n}{mat}\PY{o}{.}\PY{n}{shape}\PY{p}{[}\PY{l+m+mi}{0}\PY{p}{]}\PY{p}{)}\PY{p}{:}
        \PY{c+c1}{\PYZsh{}Only look at upper diagonal and then mirror}
        \PY{k}{for} \PY{n}{j} \PY{o+ow}{in} \PY{n+nb}{range}\PY{p}{(}\PY{n}{i}\PY{o}{+}\PY{l+m+mi}{1}\PY{p}{,} \PY{n}{mat}\PY{o}{.}\PY{n}{shape}\PY{p}{[}\PY{l+m+mi}{1}\PY{p}{]}\PY{p}{)}\PY{p}{:}
            \PY{c+c1}{\PYZsh{}Draw a unifrom}
            \PY{n}{u} \PY{o}{=} \PY{n}{np}\PY{o}{.}\PY{n}{random}\PY{o}{.}\PY{n}{uniform}\PY{p}{(}\PY{o}{\PYZhy{}}\PY{l+m+mi}{1}\PY{p}{,}\PY{l+m+mi}{1}\PY{p}{)}
            
            \PY{c+c1}{\PYZsh{}Check size}
            \PY{k}{if} \PY{n}{np}\PY{o}{.}\PY{n}{abs}\PY{p}{(}\PY{n}{u}\PY{p}{)} \PY{o}{\PYZgt{}} \PY{n}{tau}\PY{p}{[}\PY{n}{k}\PY{p}{]}\PY{p}{:}
                \PY{n}{u} \PY{o}{=} \PY{l+m+mi}{0}
            
            \PY{c+c1}{\PYZsh{}Maintain symmetry}
            \PY{n}{mat}\PY{p}{[}\PY{n}{i}\PY{p}{,}\PY{n}{j}\PY{p}{]} \PY{o}{=} \PY{n}{u}
            \PY{n}{mat}\PY{p}{[}\PY{n}{j}\PY{p}{,}\PY{n}{i}\PY{p}{]} \PY{o}{=} \PY{n}{u}
                
    \PY{n}{A}\PY{o}{.}\PY{n}{append}\PY{p}{(}\PY{n}{mat}\PY{p}{)}

\PY{c+c1}{\PYZsh{}Make the RHS a standard normal}
\PY{n}{b} \PY{o}{=} \PY{n}{np}\PY{o}{.}\PY{n}{random}\PY{o}{.}\PY{n}{uniform}\PY{p}{(}\PY{o}{\PYZhy{}}\PY{l+m+mi}{1}\PY{p}{,} \PY{l+m+mi}{1}\PY{p}{,} \PY{n}{size}\PY{o}{=}\PY{l+m+mi}{500}\PY{p}{)}\PY{o}{.}\PY{n}{reshape}\PY{p}{(}\PY{p}{(}\PY{l+m+mi}{500}\PY{p}{,}\PY{l+m+mi}{1}\PY{p}{)}\PY{p}{)}

\PY{c+c1}{\PYZsh{}Make sure all the A\PYZsq{}s are symmetric}
\PY{k}{for} \PY{n}{i}\PY{p}{,} \PY{n}{mat} \PY{o+ow}{in} \PY{n+nb}{enumerate}\PY{p}{(}\PY{n}{A}\PY{p}{)}\PY{p}{:}
    \PY{k}{if} \PY{o+ow}{not} \PY{p}{(}\PY{n}{mat}\PY{o}{.}\PY{n}{T} \PY{o}{==} \PY{n}{mat}\PY{p}{)}\PY{o}{.}\PY{n}{all}\PY{p}{(}\PY{p}{)}\PY{p}{:}
        \PY{n+nb}{print}\PY{p}{(}\PY{l+s+s1}{\PYZsq{}}\PY{l+s+s1}{Error, matrix }\PY{l+s+si}{\PYZpc{}s}\PY{l+s+s1}{ not symmetric}\PY{l+s+s1}{\PYZsq{}} \PY{o}{\PYZpc{}} \PY{n}{i}\PY{p}{)}

\PY{c+c1}{\PYZsh{}Just check the dimensions to be safe}
\PY{n+nb}{print}\PY{p}{(}\PY{n+nb}{len}\PY{p}{(}\PY{n}{A}\PY{p}{)}\PY{p}{)}
\PY{n+nb}{print}\PY{p}{(}\PY{n}{A}\PY{p}{[}\PY{l+m+mi}{0}\PY{p}{]}\PY{o}{.}\PY{n}{shape}\PY{p}{)}
\PY{n+nb}{print}\PY{p}{(}\PY{n}{b}\PY{o}{.}\PY{n}{shape}\PY{p}{)}
\end{Verbatim}
\end{tcolorbox}

    \begin{Verbatim}[commandchars=\\\{\}]
4
(500, 500)
(500, 1)
    \end{Verbatim}

    \hypertarget{a.}{%
\subsubsection{a).}\label{a.}}

Here we can see the algorithms for Steepest Descent and Conjugate
Gradient.

    \begin{tcolorbox}[breakable, size=fbox, boxrule=1pt, pad at break*=1mm,colback=cellbackground, colframe=cellborder]
\prompt{In}{incolor}{91}{\boxspacing}
\begin{Verbatim}[commandchars=\\\{\}]
\PY{l+s+sd}{\PYZsq{}\PYZsq{}\PYZsq{}}
\PY{l+s+sd}{Steepest Descent Algorithm:}
\PY{l+s+sd}{Input:}
\PY{l+s+sd}{    A: Coefficient matrix}
\PY{l+s+sd}{    b: Result vector}
\PY{l+s+sd}{    x: Initial solution guess}
\PY{l+s+sd}{    tol: Residual error tolerance}
\PY{l+s+sd}{    maxI: Maximum allowed iterations}
\PY{l+s+sd}{    norm: Norm type for tolerance}
\PY{l+s+sd}{    }
\PY{l+s+sd}{Output:}
\PY{l+s+sd}{    Success: Solution (x) and sequence of residuals (r\PYZus{}seq)}
\PY{l+s+sd}{    Failure: Print error message, but return same data}
\PY{l+s+sd}{\PYZsq{}\PYZsq{}\PYZsq{}}
\PY{k}{def} \PY{n+nf}{steep}\PY{p}{(}\PY{n}{A}\PY{p}{,} \PY{n}{b}\PY{p}{,} \PY{n}{x}\PY{p}{,} \PY{n}{tol}\PY{o}{=}\PY{l+m+mf}{1E\PYZhy{}6}\PY{p}{,} \PY{n}{maxI}\PY{o}{=}\PY{l+m+mi}{500}\PY{p}{,} \PY{n}{norm}\PY{o}{=}\PY{n}{np}\PY{o}{.}\PY{n}{inf}\PY{p}{,} \PY{n}{resnorm}\PY{o}{=}\PY{l+m+mi}{2}\PY{p}{,} \PY{n}{iterates}\PY{o}{=}\PY{k+kc}{False}\PY{p}{)}\PY{p}{:}
    
    \PY{n}{r\PYZus{}seq} \PY{o}{=} \PY{p}{[}\PY{p}{]} \PY{c+c1}{\PYZsh{}Sequence of residual norms}
    
    \PY{k}{if} \PY{n}{iterates}\PY{p}{:}
        \PY{n}{iter\PYZus{}seq} \PY{o}{=} \PY{p}{[}\PY{n}{x}\PY{p}{]}
    
    \PY{c+c1}{\PYZsh{}Calculate initial residual}
    \PY{n}{r} \PY{o}{=} \PY{n}{b} \PY{o}{\PYZhy{}} \PY{n}{A}\PY{n+nd}{@x}
    \PY{n}{r\PYZus{}norm} \PY{o}{=} \PY{n}{np}\PY{o}{.}\PY{n}{linalg}\PY{o}{.}\PY{n}{norm}\PY{p}{(}\PY{n}{r}\PY{p}{,} \PY{n+nb}{ord}\PY{o}{=}\PY{n}{resnorm}\PY{p}{)}
    \PY{n}{r\PYZus{}seq}\PY{o}{.}\PY{n}{append}\PY{p}{(}\PY{n}{r\PYZus{}norm}\PY{p}{)}
    
    \PY{c+c1}{\PYZsh{}Iterate}
    \PY{k}{for} \PY{n}{i} \PY{o+ow}{in} \PY{n+nb}{range}\PY{p}{(}\PY{n}{maxI}\PY{p}{)}\PY{p}{:}
        \PY{k}{if} \PY{n}{r\PYZus{}norm}\PY{o}{\PYZlt{}}\PY{n}{tol}\PY{p}{:}
            \PY{k}{if} \PY{n}{iterates}\PY{p}{:}
                \PY{k}{return} \PY{n}{x}\PY{p}{,} \PY{n}{r\PYZus{}seq}\PY{p}{,} \PY{n}{i}\PY{p}{,} \PY{n}{iter\PYZus{}seq}
            \PY{k}{return} \PY{n}{x}\PY{p}{,} \PY{n}{r\PYZus{}seq}\PY{p}{,} \PY{n}{i}
        
        \PY{n}{alpha} \PY{o}{=} \PY{p}{(}\PY{n}{r}\PY{o}{.}\PY{n}{T}\PY{n+nd}{@r}\PY{p}{)}\PY{o}{/}\PY{p}{(}\PY{n}{r}\PY{o}{.}\PY{n}{T}\PY{n+nd}{@A}\PY{n+nd}{@r}\PY{p}{)} \PY{c+c1}{\PYZsh{}Calculate alpha for this iteration}

        \PY{n}{x} \PY{o}{=} \PY{n}{x} \PY{o}{+} \PY{n}{alpha}\PY{o}{*}\PY{n}{r} \PY{c+c1}{\PYZsh{}Update solution}
        
        \PY{k}{if} \PY{n}{iterates}\PY{p}{:}
            \PY{n}{iter\PYZus{}seq}\PY{o}{.}\PY{n}{append}\PY{p}{(}\PY{n}{x}\PY{p}{)}

        \PY{n}{r} \PY{o}{=} \PY{n}{b} \PY{o}{\PYZhy{}} \PY{n}{A}\PY{n+nd}{@x} \PY{c+c1}{\PYZsh{}Update residual}
        \PY{n}{r\PYZus{}norm} \PY{o}{=} \PY{n}{np}\PY{o}{.}\PY{n}{linalg}\PY{o}{.}\PY{n}{norm}\PY{p}{(}\PY{n}{r}\PY{p}{,} \PY{n+nb}{ord}\PY{o}{=}\PY{n}{resnorm}\PY{p}{)}
        \PY{n}{r\PYZus{}seq}\PY{o}{.}\PY{n}{append}\PY{p}{(}\PY{n}{r\PYZus{}norm}\PY{p}{)}
        
    \PY{n+nb}{print}\PY{p}{(}\PY{l+s+s1}{\PYZsq{}}\PY{l+s+s1}{Maximum iterations exceeded without achieving tolerance.}\PY{l+s+s1}{\PYZsq{}}\PY{p}{)}
    \PY{k}{if} \PY{n}{iterates}\PY{p}{:}
        \PY{k}{return} \PY{n}{x}\PY{p}{,} \PY{n}{r\PYZus{}seq}\PY{p}{,} \PY{n}{i}\PY{p}{,} \PY{n}{iter\PYZus{}seq}
    \PY{k}{return} \PY{n}{x}\PY{p}{,} \PY{n}{r\PYZus{}seq}\PY{p}{,} \PY{n}{i}

\PY{l+s+sd}{\PYZsq{}\PYZsq{}\PYZsq{}}
\PY{l+s+sd}{Conjugate Gradient Algorithm:}
\PY{l+s+sd}{Input:}
\PY{l+s+sd}{    A: Coefficient matrix}
\PY{l+s+sd}{    b: Result vector}
\PY{l+s+sd}{    x: Initial solution guess}
\PY{l+s+sd}{    tol: Residual error tolerance}
\PY{l+s+sd}{    maxI: Maximum allowed iterations}
\PY{l+s+sd}{    norm: Norm type for tolerance}
\PY{l+s+sd}{    }
\PY{l+s+sd}{Output:}
\PY{l+s+sd}{    Success: Solution (x) and sequence of residuals (r\PYZus{}seq)}
\PY{l+s+sd}{    Failure: Print error message, but return same data}
\PY{l+s+sd}{\PYZsq{}\PYZsq{}\PYZsq{}}
\PY{k}{def} \PY{n+nf}{conjGrad}\PY{p}{(}\PY{n}{A}\PY{p}{,} \PY{n}{b}\PY{p}{,} \PY{n}{x}\PY{p}{,} \PY{n}{tol}\PY{o}{=}\PY{l+m+mf}{1E\PYZhy{}6}\PY{p}{,} \PY{n}{maxI}\PY{o}{=}\PY{l+m+mi}{500}\PY{p}{,} \PY{n}{norm}\PY{o}{=}\PY{n}{np}\PY{o}{.}\PY{n}{inf}\PY{p}{,} \PY{n}{resnorm}\PY{o}{=}\PY{l+m+mi}{2}\PY{p}{)}\PY{p}{:}
    
    \PY{n}{r\PYZus{}seq} \PY{o}{=} \PY{p}{[}\PY{p}{]} \PY{c+c1}{\PYZsh{}Sequence of residuals}
    
    \PY{n}{r0} \PY{o}{=} \PY{n}{b} \PY{o}{\PYZhy{}} \PY{n}{A}\PY{n+nd}{@x} \PY{c+c1}{\PYZsh{}Initial residual}
    \PY{n}{r\PYZus{}norm} \PY{o}{=} \PY{n}{np}\PY{o}{.}\PY{n}{linalg}\PY{o}{.}\PY{n}{norm}\PY{p}{(}\PY{n}{r0}\PY{p}{,} \PY{n+nb}{ord}\PY{o}{=}\PY{n}{resnorm}\PY{p}{)}
    \PY{n}{r\PYZus{}seq}\PY{o}{.}\PY{n}{append}\PY{p}{(}\PY{n}{r\PYZus{}norm}\PY{p}{)}
    
    \PY{n}{p} \PY{o}{=} \PY{n}{r0} \PY{c+c1}{\PYZsh{}Initial conjugate vector}
    
    \PY{k}{for} \PY{n}{i} \PY{o+ow}{in} \PY{n+nb}{range}\PY{p}{(}\PY{n}{maxI}\PY{p}{)}\PY{p}{:}   
        \PY{k}{if} \PY{n}{r\PYZus{}norm}\PY{o}{\PYZlt{}}\PY{n}{tol}\PY{p}{:}
            \PY{k}{return} \PY{n}{x}\PY{p}{,} \PY{n}{r\PYZus{}seq}\PY{p}{,} \PY{n}{i}
        
        \PY{n}{alpha} \PY{o}{=} \PY{p}{(}\PY{n}{r0}\PY{o}{.}\PY{n}{T}\PY{n+nd}{@r0}\PY{p}{)}\PY{o}{/}\PY{p}{(}\PY{n}{p}\PY{o}{.}\PY{n}{T}\PY{n+nd}{@A}\PY{n+nd}{@p}\PY{p}{)}
        
        \PY{n}{x} \PY{o}{=} \PY{n}{x} \PY{o}{+} \PY{n}{alpha}\PY{o}{*}\PY{n}{p} \PY{c+c1}{\PYZsh{}Update solution}
        
        \PY{n}{r} \PY{o}{=} \PY{n}{r0} \PY{o}{\PYZhy{}} \PY{n}{alpha}\PY{o}{*}\PY{n}{A}\PY{n+nd}{@p} \PY{c+c1}{\PYZsh{}Update residual}
        
        \PY{n}{beta} \PY{o}{=} \PY{p}{(}\PY{n}{r}\PY{o}{.}\PY{n}{T}\PY{n+nd}{@r}\PY{p}{)}\PY{o}{/}\PY{p}{(}\PY{n}{r0}\PY{o}{.}\PY{n}{T}\PY{n+nd}{@r0}\PY{p}{)}
        
        \PY{n}{p} \PY{o}{=} \PY{n}{r} \PY{o}{+} \PY{n}{beta}\PY{o}{*}\PY{n}{p} \PY{c+c1}{\PYZsh{}Update conjugate vector}
        
        \PY{c+c1}{\PYZsh{}No longer need old residual}
        \PY{n}{r0} \PY{o}{=} \PY{n}{r}
        \PY{n}{r\PYZus{}norm} \PY{o}{=} \PY{n}{np}\PY{o}{.}\PY{n}{linalg}\PY{o}{.}\PY{n}{norm}\PY{p}{(}\PY{n}{r0}\PY{p}{,} \PY{n+nb}{ord}\PY{o}{=}\PY{n}{resnorm}\PY{p}{)}
        \PY{n}{r\PYZus{}seq}\PY{o}{.}\PY{n}{append}\PY{p}{(}\PY{n}{r\PYZus{}norm}\PY{p}{)}
        
    \PY{n+nb}{print}\PY{p}{(}\PY{l+s+s1}{\PYZsq{}}\PY{l+s+s1}{Maximum iterations exceeded without achieving tolerance.}\PY{l+s+s1}{\PYZsq{}}\PY{p}{)}
    \PY{k}{return} \PY{n}{x}\PY{p}{,} \PY{n}{r\PYZus{}seq}\PY{p}{,} \PY{n}{i}
\end{Verbatim}
\end{tcolorbox}

    \hypertarget{b.}{%
\subsubsection{b).}\label{b.}}

We apply Steepest Descent to solve each of the linear systems
\(\mathbf{A}_k\mathbf{x}=\mathbf{B}\)

    \begin{tcolorbox}[breakable, size=fbox, boxrule=1pt, pad at break*=1mm,colback=cellbackground, colframe=cellborder]
\prompt{In}{incolor}{92}{\boxspacing}
\begin{Verbatim}[commandchars=\\\{\}]
\PY{n}{R} \PY{o}{=} \PY{p}{[}\PY{p}{]}
\PY{n}{steep\PYZus{}i} \PY{o}{=} \PY{p}{[}\PY{p}{]}
\PY{n}{I} \PY{o}{=} \PY{p}{[}\PY{p}{]}

\PY{c+c1}{\PYZsh{}Solve each linear system}
\PY{k}{for} \PY{n}{i}\PY{p}{,} \PY{n}{mat} \PY{o+ow}{in} \PY{n+nb}{enumerate}\PY{p}{(}\PY{n}{A}\PY{p}{)}\PY{p}{:}
    \PY{n+nb}{print}\PY{p}{(}\PY{l+s+s1}{\PYZsq{}}\PY{l+s+s1}{Matrix }\PY{l+s+si}{\PYZpc{}s}\PY{l+s+s1}{\PYZsq{}} \PY{o}{\PYZpc{}} \PY{n}{i}\PY{p}{)}
    \PY{n}{\PYZus{}}\PY{p}{,} \PY{n}{r\PYZus{}seq}\PY{p}{,} \PY{n}{iters}\PY{p}{,} \PY{n}{iter\PYZus{}seq} \PY{o}{=} \PY{n}{steep}\PY{p}{(}\PY{n}{mat}\PY{p}{,} \PY{n}{b}\PY{p}{,} \PY{n}{np}\PY{o}{.}\PY{n}{zeros}\PY{p}{(}\PY{p}{(}\PY{l+m+mi}{500}\PY{p}{,}\PY{l+m+mi}{1}\PY{p}{)}\PY{p}{)}\PY{p}{,} \PY{n}{iterates}\PY{o}{=}\PY{k+kc}{True}\PY{p}{)}
    
    \PY{n}{R}\PY{o}{.}\PY{n}{append}\PY{p}{(}\PY{n}{r\PYZus{}seq}\PY{p}{)} \PY{c+c1}{\PYZsh{}List of sequences of residuals}
    \PY{n}{steep\PYZus{}i}\PY{o}{.}\PY{n}{append}\PY{p}{(}\PY{n}{iters}\PY{p}{)}
    \PY{n}{I}\PY{o}{.}\PY{n}{append}\PY{p}{(}\PY{n}{iter\PYZus{}seq}\PY{p}{)}
\end{Verbatim}
\end{tcolorbox}

    \begin{Verbatim}[commandchars=\\\{\}]
Matrix 0
Matrix 1
Matrix 2
Matrix 3
Maximum iterations exceeded without achieving tolerance.
    \end{Verbatim}

    \begin{Verbatim}[commandchars=\\\{\}]
/home/rs-coop/anaconda3/lib/python3.7/site-packages/ipykernel\_launcher.py:34:
RuntimeWarning: overflow encountered in matmul
/home/rs-coop/anaconda3/lib/python3.7/site-packages/ipykernel\_launcher.py:34:
RuntimeWarning: invalid value encountered in true\_divide
    \end{Verbatim}

    We can see that the fourth linear system with \(\tau=0.2\) is having
problems. Specifically, it seems to exceed the maximum number of
iterations and encounter some numerical issues.

    \begin{tcolorbox}[breakable, size=fbox, boxrule=1pt, pad at break*=1mm,colback=cellbackground, colframe=cellborder]
\prompt{In}{incolor}{93}{\boxspacing}
\begin{Verbatim}[commandchars=\\\{\}]
\PY{n}{fig1}\PY{p}{,} \PY{n}{ax1} \PY{o}{=} \PY{n}{plt}\PY{o}{.}\PY{n}{subplots}\PY{p}{(}\PY{l+m+mi}{1}\PY{p}{,}\PY{l+m+mi}{1}\PY{p}{,}\PY{n}{figsize}\PY{o}{=}\PY{p}{(}\PY{l+m+mi}{10}\PY{p}{,}\PY{l+m+mi}{10}\PY{p}{)}\PY{p}{)}

\PY{k}{for} \PY{n}{i}\PY{p}{,} \PY{n}{norm\PYZus{}seq} \PY{o+ow}{in} \PY{n+nb}{enumerate}\PY{p}{(}\PY{n}{R}\PY{p}{)}\PY{p}{:}
    \PY{k}{if} \PY{n}{i} \PY{o}{\PYZlt{}} \PY{l+m+mi}{3}\PY{p}{:}
        \PY{n}{ax1}\PY{o}{.}\PY{n}{semilogy}\PY{p}{(}\PY{n}{norm\PYZus{}seq}\PY{p}{)}
    
\PY{n}{ax1}\PY{o}{.}\PY{n}{set\PYZus{}title}\PY{p}{(}\PY{l+s+s1}{\PYZsq{}}\PY{l+s+s1}{Log of Residual 2\PYZhy{}norm Per Iteration}\PY{l+s+s1}{\PYZsq{}}\PY{p}{)}
\PY{n}{ax1}\PY{o}{.}\PY{n}{set\PYZus{}xlabel}\PY{p}{(}\PY{l+s+s1}{\PYZsq{}}\PY{l+s+s1}{Iteration}\PY{l+s+s1}{\PYZsq{}}\PY{p}{)}
\PY{n}{ax1}\PY{o}{.}\PY{n}{set\PYZus{}ylabel}\PY{p}{(}\PY{l+s+s1}{\PYZsq{}}\PY{l+s+s1}{Log of residual 2\PYZhy{}norm}\PY{l+s+s1}{\PYZsq{}}\PY{p}{)}
\PY{n}{ax1}\PY{o}{.}\PY{n}{legend}\PY{p}{(}\PY{n}{tau}\PY{p}{)}\PY{p}{;}
\end{Verbatim}
\end{tcolorbox}

    \begin{center}
    \adjustimage{max size={0.9\linewidth}{0.9\paperheight}}{output_10_0.png}
    \end{center}
    { \hspace*{\fill} \\}
    
    \hypertarget{c.}{%
\subsubsection{c).}\label{c.}}

We apply Conjugate Gradient to solve each of the linear systems
\(\mathbf{A}_k\mathbf{x}=\mathbf{B}\)

    \begin{tcolorbox}[breakable, size=fbox, boxrule=1pt, pad at break*=1mm,colback=cellbackground, colframe=cellborder]
\prompt{In}{incolor}{94}{\boxspacing}
\begin{Verbatim}[commandchars=\\\{\}]
\PY{n}{R} \PY{o}{=} \PY{p}{[}\PY{p}{]}
\PY{n}{conj\PYZus{}i} \PY{o}{=} \PY{p}{[}\PY{p}{]}

\PY{c+c1}{\PYZsh{}Solve each linear system}
\PY{k}{for} \PY{n}{i}\PY{p}{,} \PY{n}{mat} \PY{o+ow}{in} \PY{n+nb}{enumerate}\PY{p}{(}\PY{n}{A}\PY{p}{)}\PY{p}{:}
    \PY{n+nb}{print}\PY{p}{(}\PY{l+s+s1}{\PYZsq{}}\PY{l+s+s1}{Matrix }\PY{l+s+si}{\PYZpc{}s}\PY{l+s+s1}{\PYZsq{}} \PY{o}{\PYZpc{}} \PY{n}{i}\PY{p}{)}
    \PY{n}{\PYZus{}}\PY{p}{,} \PY{n}{r\PYZus{}seq}\PY{p}{,} \PY{n}{iters} \PY{o}{=} \PY{n}{conjGrad}\PY{p}{(}\PY{n}{mat}\PY{p}{,} \PY{n}{b}\PY{p}{,} \PY{n}{np}\PY{o}{.}\PY{n}{zeros}\PY{p}{(}\PY{p}{(}\PY{l+m+mi}{500}\PY{p}{,}\PY{l+m+mi}{1}\PY{p}{)}\PY{p}{)}\PY{p}{)}

    \PY{n}{R}\PY{o}{.}\PY{n}{append}\PY{p}{(}\PY{n}{r\PYZus{}seq}\PY{p}{)} \PY{c+c1}{\PYZsh{}List of sequences of residuals}
    \PY{n}{conj\PYZus{}i}\PY{o}{.}\PY{n}{append}\PY{p}{(}\PY{n}{iters}\PY{p}{)}
\end{Verbatim}
\end{tcolorbox}

    \begin{Verbatim}[commandchars=\\\{\}]
Matrix 0
Matrix 1
Matrix 2
Matrix 3
Maximum iterations exceeded without achieving tolerance.
    \end{Verbatim}

    Again we can see that our fourth linear system where \(\tau=0.2\) has
failed to converge.

    \begin{tcolorbox}[breakable, size=fbox, boxrule=1pt, pad at break*=1mm,colback=cellbackground, colframe=cellborder]
\prompt{In}{incolor}{95}{\boxspacing}
\begin{Verbatim}[commandchars=\\\{\}]
\PY{n}{fig2}\PY{p}{,} \PY{n}{ax2} \PY{o}{=} \PY{n}{plt}\PY{o}{.}\PY{n}{subplots}\PY{p}{(}\PY{l+m+mi}{1}\PY{p}{,}\PY{l+m+mi}{1}\PY{p}{,}\PY{n}{figsize}\PY{o}{=}\PY{p}{(}\PY{l+m+mi}{10}\PY{p}{,}\PY{l+m+mi}{10}\PY{p}{)}\PY{p}{)}

\PY{k}{for} \PY{n}{i}\PY{p}{,}\PY{n}{norm\PYZus{}seq} \PY{o+ow}{in} \PY{n+nb}{enumerate}\PY{p}{(}\PY{n}{R}\PY{p}{)}\PY{p}{:}
    \PY{k}{if} \PY{n}{i} \PY{o}{\PYZlt{}} \PY{l+m+mi}{3}\PY{p}{:}
        \PY{n}{ax2}\PY{o}{.}\PY{n}{semilogy}\PY{p}{(}\PY{n}{norm\PYZus{}seq}\PY{p}{)}
    
\PY{n}{ax2}\PY{o}{.}\PY{n}{set\PYZus{}title}\PY{p}{(}\PY{l+s+s1}{\PYZsq{}}\PY{l+s+s1}{Log of Residual 2\PYZhy{}norm Per Iteration}\PY{l+s+s1}{\PYZsq{}}\PY{p}{)}
\PY{n}{ax2}\PY{o}{.}\PY{n}{set\PYZus{}xlabel}\PY{p}{(}\PY{l+s+s1}{\PYZsq{}}\PY{l+s+s1}{Iteration}\PY{l+s+s1}{\PYZsq{}}\PY{p}{)}
\PY{n}{ax2}\PY{o}{.}\PY{n}{set\PYZus{}ylabel}\PY{p}{(}\PY{l+s+s1}{\PYZsq{}}\PY{l+s+s1}{Log of residual 2\PYZhy{}norm}\PY{l+s+s1}{\PYZsq{}}\PY{p}{)}
\PY{n}{ax2}\PY{o}{.}\PY{n}{legend}\PY{p}{(}\PY{n}{tau}\PY{p}{)}\PY{p}{;}
\end{Verbatim}
\end{tcolorbox}

    \begin{center}
    \adjustimage{max size={0.9\linewidth}{0.9\paperheight}}{output_14_0.png}
    \end{center}
    { \hspace*{\fill} \\}
    
    \hypertarget{d.}{%
\subsubsection{d).}\label{d.}}

In the preceding two figures we see that a solution has been converged
to for the linear systems corresponding to \(\tau=0.01, 0.05, 0.1\).
However, both steepest descent and conjugate gradient fail to converge
on the linear system where \(\tau=0.2\). Ignoring this for a moment we
see that in all of the other linear systems, the conjugate gradient
algorithm is systematically faster -- which is expected. Furthermore, we
see that with both algorithms we see that smaller \(\tau\) results in
faster convergence.

We are interested why \(\tau=0.2\) fails to converge. We know that the
algorithms will succeed when the matrix is SPD, so we will examine the
eigenvalues of our matrices \(\mathbf{A}_k\).

    \begin{tcolorbox}[breakable, size=fbox, boxrule=1pt, pad at break*=1mm,colback=cellbackground, colframe=cellborder]
\prompt{In}{incolor}{96}{\boxspacing}
\begin{Verbatim}[commandchars=\\\{\}]
\PY{c+c1}{\PYZsh{}Looking at eigenvalues of A\PYZus{}k}
\PY{k}{for} \PY{n}{i}\PY{p}{,} \PY{n}{mat} \PY{o+ow}{in} \PY{n+nb}{enumerate}\PY{p}{(}\PY{n}{A}\PY{p}{)}\PY{p}{:}
    \PY{n}{evals}\PY{p}{,} \PY{n}{\PYZus{}} \PY{o}{=} \PY{n}{np}\PY{o}{.}\PY{n}{linalg}\PY{o}{.}\PY{n}{eig}\PY{p}{(}\PY{n}{mat}\PY{p}{)}
    
    \PY{k}{if} \PY{p}{(}\PY{n}{evals} \PY{o}{\PYZlt{}}\PY{o}{=} \PY{l+m+mi}{0}\PY{p}{)}\PY{o}{.}\PY{n}{any}\PY{p}{(}\PY{p}{)}\PY{p}{:}
        \PY{n+nb}{print}\PY{p}{(}\PY{l+s+s1}{\PYZsq{}}\PY{l+s+s1}{Non\PYZhy{}positive eigenvalues in matrix }\PY{l+s+si}{\PYZpc{}s}\PY{l+s+s1}{ with tau=}\PY{l+s+si}{\PYZpc{}.1f}\PY{l+s+s1}{\PYZsq{}} \PY{o}{\PYZpc{}}\PY{p}{(}\PY{n}{i}\PY{o}{+}\PY{l+m+mi}{1}\PY{p}{,} \PY{n}{tau}\PY{p}{[}\PY{n}{i}\PY{p}{]}\PY{p}{)}\PY{p}{)} 
\end{Verbatim}
\end{tcolorbox}

    \begin{Verbatim}[commandchars=\\\{\}]
Non-positive eigenvalues in matrix 4 with tau=0.2
    \end{Verbatim}

    Ah-ha! We can see that the fourth matrix with the largest \(\tau\)
(\(\tau=2\)) has non-positive eigenvalues. This indicates that the
matrix is not positive-definite, and thus we wont have convergence. The
other matrices are SPD and that is why we get convergence.

    \hypertarget{e.}{%
\subsubsection{e).}\label{e.}}

We have the following error bound for the steepest descent algorithm:

\[ ||\mathbf{x}_{k}-\mathbf{x}_*||_A\leq\frac{\lambda_{max}-\lambda_{min}}{\lambda_{max}+\lambda_{min}}||\mathbf{x}_{k-1}-\mathbf{x}_*||_A \]

For the conjugate gradient algorithm we have the following error bound:

\[ ||\mathbf{x}_{k}-\mathbf{x}_*||_A\leq 2\Bigg(\frac{1-\sqrt{\kappa(A)^{-1}}}{1+\sqrt{\kappa(A)^{-1}}}\Bigg)^n||\mathbf{x}_*||_A \]

Where \(\mathbf{x}_*\) is the solution of the linear system and
\(\kappa(A)\) is the condition number.

    \begin{tcolorbox}[breakable, size=fbox, boxrule=1pt, pad at break*=1mm,colback=cellbackground, colframe=cellborder]
\prompt{In}{incolor}{97}{\boxspacing}
\begin{Verbatim}[commandchars=\\\{\}]
\PY{n}{solutions} \PY{o}{=} \PY{p}{[}\PY{n}{cg}\PY{p}{(}\PY{n}{mat}\PY{p}{,} \PY{n}{b}\PY{p}{)}\PY{p}{[}\PY{l+m+mi}{0}\PY{p}{]} \PY{k}{for} \PY{n}{mat} \PY{o+ow}{in} \PY{n}{A}\PY{p}{[}\PY{p}{:}\PY{o}{\PYZhy{}}\PY{l+m+mi}{1}\PY{p}{]}\PY{p}{]} \PY{c+c1}{\PYZsh{}Solutions to the first three linear systems}

\PY{k}{def} \PY{n+nf}{steepBound}\PY{p}{(}\PY{n}{A}\PY{p}{,} \PY{n}{x\PYZus{}prev}\PY{p}{,} \PY{n}{sol}\PY{p}{)}\PY{p}{:}
    \PY{n}{evals}\PY{p}{,} \PY{n}{\PYZus{}} \PY{o}{=} \PY{n}{np}\PY{o}{.}\PY{n}{linalg}\PY{o}{.}\PY{n}{eig}\PY{p}{(}\PY{n}{A}\PY{p}{)}
    \PY{n}{l\PYZus{}max} \PY{o}{=} \PY{n+nb}{max}\PY{p}{(}\PY{n}{evals}\PY{p}{)}
    \PY{n}{l\PYZus{}min} \PY{o}{=} \PY{n+nb}{min}\PY{p}{(}\PY{n}{evals}\PY{p}{)}
    
    \PY{n}{z} \PY{o}{=} \PY{n}{x\PYZus{}prev} \PY{o}{\PYZhy{}} \PY{n}{sol}
    
    \PY{k}{return} \PY{p}{(}\PY{p}{(}\PY{n}{l\PYZus{}max}\PY{o}{\PYZhy{}}\PY{n}{l\PYZus{}min}\PY{p}{)}\PY{o}{/}\PY{p}{(}\PY{n}{l\PYZus{}max}\PY{o}{+}\PY{n}{l\PYZus{}min}\PY{p}{)}\PY{p}{)}\PY{o}{*}\PY{p}{(}\PY{n}{z}\PY{o}{.}\PY{n}{T}\PY{n+nd}{@A}\PY{n+nd}{@z}\PY{p}{)}

\PY{k}{def} \PY{n+nf}{conjBound}\PY{p}{(}\PY{n}{A}\PY{p}{,} \PY{n}{sol}\PY{p}{,} \PY{n}{n}\PY{p}{)}\PY{p}{:}
    \PY{n}{k} \PY{o}{=} \PY{l+m+mi}{1}\PY{o}{/}\PY{n}{np}\PY{o}{.}\PY{n}{linalg}\PY{o}{.}\PY{n}{cond}\PY{p}{(}\PY{n}{A}\PY{p}{)}
    
    \PY{k}{return} \PY{p}{(}\PY{n}{sol}\PY{o}{.}\PY{n}{T}\PY{n+nd}{@A}\PY{n+nd}{@sol}\PY{p}{)}\PY{o}{*}\PY{l+m+mi}{2}\PY{o}{*}\PY{p}{(}\PY{p}{(}\PY{l+m+mi}{1}\PY{o}{\PYZhy{}}\PY{n}{np}\PY{o}{.}\PY{n}{sqrt}\PY{p}{(}\PY{n}{k}\PY{p}{)}\PY{p}{)}\PY{o}{/}\PY{p}{(}\PY{l+m+mi}{1}\PY{o}{+}\PY{n}{np}\PY{o}{.}\PY{n}{sqrt}\PY{p}{(}\PY{n}{k}\PY{p}{)}\PY{p}{)}\PY{p}{)}\PY{o}{*}\PY{o}{*}\PY{n}{n}
\end{Verbatim}
\end{tcolorbox}

    \begin{tcolorbox}[breakable, size=fbox, boxrule=1pt, pad at break*=1mm,colback=cellbackground, colframe=cellborder]
\prompt{In}{incolor}{98}{\boxspacing}
\begin{Verbatim}[commandchars=\\\{\}]
\PY{k}{for} \PY{n}{i}\PY{p}{,} \PY{n}{mat} \PY{o+ow}{in} \PY{n+nb}{enumerate}\PY{p}{(}\PY{n}{A}\PY{p}{[}\PY{p}{:}\PY{o}{\PYZhy{}}\PY{l+m+mi}{1}\PY{p}{]}\PY{p}{)}\PY{p}{:}
    \PY{n}{sol} \PY{o}{=} \PY{n}{solutions}\PY{p}{[}\PY{n}{i}\PY{p}{]}\PY{o}{.}\PY{n}{reshape}\PY{p}{(}\PY{p}{(}\PY{l+m+mi}{500}\PY{p}{,}\PY{l+m+mi}{1}\PY{p}{)}\PY{p}{)}
    \PY{n}{er} \PY{o}{=} \PY{n}{steepBound}\PY{p}{(}\PY{n}{mat}\PY{p}{,} \PY{n}{I}\PY{p}{[}\PY{n}{i}\PY{p}{]}\PY{p}{[}\PY{o}{\PYZhy{}}\PY{l+m+mi}{2}\PY{p}{]}\PY{p}{,} \PY{n}{sol}\PY{p}{)}
    \PY{n}{v} \PY{o}{=} \PY{n}{I}\PY{p}{[}\PY{n}{i}\PY{p}{]}\PY{p}{[}\PY{o}{\PYZhy{}}\PY{l+m+mi}{1}\PY{p}{]} \PY{o}{\PYZhy{}} \PY{n}{sol}
    \PY{n}{res} \PY{o}{=} \PY{n}{v}\PY{o}{.}\PY{n}{T}\PY{n+nd}{@mat}\PY{n+nd}{@v}
    
    \PY{n+nb}{print}\PY{p}{(}\PY{l+s+s1}{\PYZsq{}}\PY{l+s+s1}{Bound: }\PY{l+s+si}{\PYZpc{}.10f}\PY{l+s+s1}{, Actual: }\PY{l+s+si}{\PYZpc{}.10f}\PY{l+s+s1}{\PYZsq{}} \PY{o}{\PYZpc{}} \PY{p}{(}\PY{n}{er}\PY{p}{,} \PY{n}{res}\PY{p}{)}\PY{p}{)}
\end{Verbatim}
\end{tcolorbox}

    \begin{Verbatim}[commandchars=\\\{\}]
Bound: 0.0000000000, Actual: 0.0000000013
Bound: 0.0000000048, Actual: 0.0000000166
Bound: 0.0000000203, Actual: 0.0000000249
    \end{Verbatim}

    In the above output can see for the first three matrices where we have
convergence, that the bound estimate is quite close to the actual
residual norm. We note that the bound estimate is always smaller, but
not by much. The two quantities are not the same, so we would not expect
an exact match, but a relatively close one is validating.

    \begin{tcolorbox}[breakable, size=fbox, boxrule=1pt, pad at break*=1mm,colback=cellbackground, colframe=cellborder]
\prompt{In}{incolor}{99}{\boxspacing}
\begin{Verbatim}[commandchars=\\\{\}]
\PY{n}{conj\PYZus{}bound} \PY{o}{=} \PY{p}{[}\PY{n}{conjBound}\PY{p}{(}\PY{n}{A}\PY{p}{[}\PY{n}{i}\PY{p}{]}\PY{p}{,} \PY{n}{solutions}\PY{p}{[}\PY{n}{i}\PY{p}{]}\PY{p}{,} \PY{n}{np}\PY{o}{.}\PY{n}{arange}\PY{p}{(}\PY{n}{conj\PYZus{}i}\PY{p}{[}\PY{n}{i}\PY{p}{]}\PY{o}{+}\PY{l+m+mi}{1}\PY{p}{)}\PY{p}{)} \PY{k}{for} \PY{n}{i} \PY{o+ow}{in} \PY{n+nb}{range}\PY{p}{(}\PY{n+nb}{len}\PY{p}{(}\PY{n}{A}\PY{p}{)}\PY{o}{\PYZhy{}}\PY{l+m+mi}{1}\PY{p}{)}\PY{p}{]}
\end{Verbatim}
\end{tcolorbox}

    \begin{tcolorbox}[breakable, size=fbox, boxrule=1pt, pad at break*=1mm,colback=cellbackground, colframe=cellborder]
\prompt{In}{incolor}{100}{\boxspacing}
\begin{Verbatim}[commandchars=\\\{\}]
\PY{k}{for} \PY{n}{i} \PY{o+ow}{in} \PY{n+nb}{range}\PY{p}{(}\PY{n+nb}{len}\PY{p}{(}\PY{n}{A}\PY{p}{)}\PY{o}{\PYZhy{}}\PY{l+m+mi}{1}\PY{p}{)}\PY{p}{:}
\PY{c+c1}{\PYZsh{}     ax1.semilogy(steep\PYZus{}bound[i], c=\PYZsq{}k\PYZsq{})}
    \PY{n}{ax2}\PY{o}{.}\PY{n}{semilogy}\PY{p}{(}\PY{n}{conj\PYZus{}bound}\PY{p}{[}\PY{n}{i}\PY{p}{]}\PY{p}{,} \PY{n}{c}\PY{o}{=}\PY{l+s+s1}{\PYZsq{}}\PY{l+s+s1}{k}\PY{l+s+s1}{\PYZsq{}}\PY{p}{)}
    
\PY{n}{fig2}
\end{Verbatim}
\end{tcolorbox}
 
            
\prompt{Out}{outcolor}{100}{}
    
    \begin{center}
    \adjustimage{max size={0.9\linewidth}{0.9\paperheight}}{output_23_0.png}
    \end{center}
    { \hspace*{\fill} \\}
    

    We see the same plot as for part c, but with three extra black lines.
These lines correspond to the error bound for conjugate gradient. We can
see that the associated norm of the residual is less than the error
bound and follows a similar trajectory. Again it is not the same
quantity, but it is validating to see a similar trend.

\newpage
    \hypertarget{problem-4}{%
\subsection{Problem 4}\label{problem-4}}

    We consider solving the system of non-linear equations:

\[ f_1(x,y) = 3x^2+4y^2-1 = 0 \:\:\:\: f_2(x,y) = y^3-8x^3-1 = 0 \]

Where we are looking for a solution \(\mathbf{\alpha}\) near
\((x,y)=(-0.5,0.25)\).

    \hypertarget{a.}{%
\subsubsection{a).}\label{a.}}

We apply fixed point iteration with the following matrix formulation.

\[
\mathbf{g}(\mathbf{x}) = \mathbf{x} -
\begin{bmatrix}
    0.016 & -0.17\\
    0.52 &-0.26
\end{bmatrix}
\begin{bmatrix}
    3x^2+4y^2-1\\
    y^3-8x^3-1
\end{bmatrix}
\]

    \begin{tcolorbox}[breakable, size=fbox, boxrule=1pt, pad at break*=1mm,colback=cellbackground, colframe=cellborder]
\prompt{In}{incolor}{101}{\boxspacing}
\begin{Verbatim}[commandchars=\\\{\}]
\PY{l+s+sd}{\PYZsq{}\PYZsq{}\PYZsq{}}
\PY{l+s+sd}{Fixed Point Iteration:}
\PY{l+s+sd}{Input:}
\PY{l+s+sd}{    g: Vector of iteration functions}
\PY{l+s+sd}{    x0: Initial solution guess}
\PY{l+s+sd}{    tol: Residual error tolerance}
\PY{l+s+sd}{    maxI: Maximum allowed iterations}
\PY{l+s+sd}{    norm: Norm type for tolerance}
\PY{l+s+sd}{    }
\PY{l+s+sd}{Output:}
\PY{l+s+sd}{    Success: Solution (x) and number of iterations (i)}
\PY{l+s+sd}{    Failure: ValueError for exceeding maximum iterations}
\PY{l+s+sd}{\PYZsq{}\PYZsq{}\PYZsq{}}
\PY{k}{def} \PY{n+nf}{FPI}\PY{p}{(}\PY{n}{g}\PY{p}{,} \PY{n}{x0}\PY{p}{,} \PY{n}{tol}\PY{o}{=}\PY{l+m+mf}{1E\PYZhy{}6}\PY{p}{,} \PY{n}{maxI}\PY{o}{=}\PY{l+m+mi}{1000}\PY{p}{,} \PY{n}{norm}\PY{o}{=}\PY{n}{np}\PY{o}{.}\PY{n}{inf}\PY{p}{)}\PY{p}{:}
    
    \PY{n}{x} \PY{o}{=} \PY{n}{np}\PY{o}{.}\PY{n}{zeros}\PY{p}{(}\PY{n}{x0}\PY{o}{.}\PY{n}{shape}\PY{p}{)} \PY{c+c1}{\PYZsh{}New iterate}
    
    \PY{k}{for} \PY{n}{i} \PY{o+ow}{in} \PY{n+nb}{range}\PY{p}{(}\PY{l+m+mi}{1}\PY{p}{,} \PY{n}{maxI}\PY{o}{+}\PY{l+m+mi}{1}\PY{p}{)}\PY{p}{:}
        \PY{n}{x} \PY{o}{=} \PY{n}{g}\PY{p}{(}\PY{n}{x0}\PY{p}{)} \PY{c+c1}{\PYZsh{}Evaluate new iterate}
        
        \PY{c+c1}{\PYZsh{}Check if change is below tolerance}
        \PY{k}{if} \PY{n}{np}\PY{o}{.}\PY{n}{linalg}\PY{o}{.}\PY{n}{norm}\PY{p}{(}\PY{n}{x}\PY{o}{\PYZhy{}}\PY{n}{x0}\PY{p}{,} \PY{n+nb}{ord}\PY{o}{=}\PY{n}{norm}\PY{p}{)}\PY{o}{/}\PY{n}{np}\PY{o}{.}\PY{n}{linalg}\PY{o}{.}\PY{n}{norm}\PY{p}{(}\PY{n}{x}\PY{p}{,} \PY{n+nb}{ord}\PY{o}{=}\PY{n}{norm}\PY{p}{)} \PY{o}{\PYZlt{}} \PY{n}{tol}\PY{p}{:}
            \PY{k}{return} \PY{n}{x}\PY{p}{,} \PY{n}{i}
        
        \PY{n}{x0} \PY{o}{=} \PY{n}{x}
        
    \PY{k}{raise} \PY{n+ne}{ValueError}\PY{p}{(}\PY{l+s+s1}{\PYZsq{}}\PY{l+s+s1}{Maximum number of iterations exceeded.}\PY{l+s+s1}{\PYZsq{}}\PY{p}{)}
\end{Verbatim}
\end{tcolorbox}

    \begin{tcolorbox}[breakable, size=fbox, boxrule=1pt, pad at break*=1mm,colback=cellbackground, colframe=cellborder]
\prompt{In}{incolor}{102}{\boxspacing}
\begin{Verbatim}[commandchars=\\\{\}]
\PY{c+c1}{\PYZsh{}Coefficient matrix for our iteration function g}
\PY{n}{C} \PY{o}{=} \PY{n}{np}\PY{o}{.}\PY{n}{array}\PY{p}{(}\PY{p}{[}\PY{p}{[}\PY{l+m+mf}{0.016}\PY{p}{,} \PY{o}{\PYZhy{}}\PY{l+m+mf}{0.17}\PY{p}{]}\PY{p}{,}\PY{p}{[}\PY{l+m+mf}{0.52}\PY{p}{,} \PY{o}{\PYZhy{}}\PY{l+m+mf}{0.26}\PY{p}{]}\PY{p}{]}\PY{p}{)}

\PY{c+c1}{\PYZsh{}Returns the functions evaluated at (x1,x2)=x}
\PY{k}{def} \PY{n+nf}{F}\PY{p}{(}\PY{n}{x}\PY{p}{)}\PY{p}{:}
    \PY{n}{x1}\PY{p}{,} \PY{n}{x2} \PY{o}{=} \PY{n}{x}
    \PY{k}{return} \PY{p}{[}\PY{l+m+mi}{3}\PY{o}{*}\PY{n}{x1}\PY{o}{*}\PY{o}{*}\PY{l+m+mi}{2} \PY{o}{+} \PY{l+m+mi}{4}\PY{o}{*}\PY{n}{x2}\PY{o}{*}\PY{o}{*}\PY{l+m+mi}{2} \PY{o}{\PYZhy{}} \PY{l+m+mi}{1}\PY{p}{,} \PY{n}{x2}\PY{o}{*}\PY{o}{*}\PY{l+m+mi}{3} \PY{o}{\PYZhy{}} \PY{l+m+mi}{8}\PY{o}{*}\PY{n}{x1}\PY{o}{*}\PY{o}{*}\PY{l+m+mi}{3} \PY{o}{\PYZhy{}} \PY{l+m+mi}{1}\PY{p}{]}

\PY{c+c1}{\PYZsh{}Returns the g iteration function at a point (x1,x2)=x}
\PY{k}{def} \PY{n+nf}{g}\PY{p}{(}\PY{n}{x}\PY{p}{)}\PY{p}{:}
    \PY{n}{z} \PY{o}{=} \PY{n}{np}\PY{o}{.}\PY{n}{array}\PY{p}{(}\PY{n}{F}\PY{p}{(}\PY{n}{x}\PY{p}{)}\PY{p}{)}
    \PY{k}{return} \PY{n}{x} \PY{o}{\PYZhy{}} \PY{n}{C}\PY{n+nd}{@z}
\end{Verbatim}
\end{tcolorbox}

    \begin{tcolorbox}[breakable, size=fbox, boxrule=1pt, pad at break*=1mm,colback=cellbackground, colframe=cellborder]
\prompt{In}{incolor}{103}{\boxspacing}
\begin{Verbatim}[commandchars=\\\{\}]
\PY{n}{guess} \PY{o}{=} \PY{n}{np}\PY{o}{.}\PY{n}{array}\PY{p}{(}\PY{p}{[}\PY{p}{[}\PY{o}{\PYZhy{}}\PY{l+m+mf}{0.5}\PY{p}{]}\PY{p}{,} \PY{p}{[}\PY{l+m+mf}{0.25}\PY{p}{]}\PY{p}{]}\PY{p}{)}
\PY{n}{alpha}\PY{p}{,} \PY{n}{i} \PY{o}{=} \PY{n}{FPI}\PY{p}{(}\PY{n}{g}\PY{p}{,} \PY{n}{guess}\PY{p}{,} \PY{n}{tol}\PY{o}{=}\PY{l+m+mf}{1E\PYZhy{}7}\PY{p}{)}
\PY{n}{alpha} \PY{o}{=} \PY{n}{alpha}\PY{o}{.}\PY{n}{reshape}\PY{p}{(}\PY{p}{(}\PY{l+m+mi}{2}\PY{p}{,}\PY{p}{)}\PY{p}{)}
\PY{n+nb}{print}\PY{p}{(}\PY{l+s+s1}{\PYZsq{}}\PY{l+s+s1}{Solution (x,y) in }\PY{l+s+si}{\PYZpc{}s}\PY{l+s+s1}{ iterations:}\PY{l+s+s1}{\PYZsq{}} \PY{o}{\PYZpc{}} \PY{n}{i}\PY{p}{)}
\PY{n+nb}{print}\PY{p}{(}\PY{n}{alpha}\PY{p}{,} \PY{l+s+s1}{\PYZsq{}}\PY{l+s+se}{\PYZbs{}n}\PY{l+s+s1}{\PYZsq{}}\PY{p}{)}
\PY{n+nb}{print}\PY{p}{(}\PY{l+s+s1}{\PYZsq{}}\PY{l+s+s1}{Functions evaluated at solution:}\PY{l+s+s1}{\PYZsq{}}\PY{p}{)}
\PY{n+nb}{print}\PY{p}{(}\PY{n}{F}\PY{p}{(}\PY{n}{alpha}\PY{p}{)}\PY{p}{)}
\end{Verbatim}
\end{tcolorbox}

    \begin{Verbatim}[commandchars=\\\{\}]
Solution (x,y) in 5 iterations:
[-0.4972512   0.25407859]

Functions evaluated at solution:
[5.54330359392452e-10, 1.2558354356428936e-09]
    \end{Verbatim}

    We can see that we have found the root of our equations as\ldots{}

\[ \boxed{(x,y)=(-0.4972512, 0.25407859)} \]

This is with 7-digits of accuracy and was achieved in \(\boxed{5}\)
iterations.

    \hypertarget{b.}{%
\subsubsection{b).}\label{b.}}

This is a good choice for \(\mathbf{g}(\mathbf{x})\) because (as we will
see below) the infinity norm of the Jacobian of our iteration function
\(\mathbf{g}(\mathbf{x})\) is strictly less than 1 at the root. The
Jacobian is obtained by differentiating \(\mathbf{g}\) with respect to
\(x\) and inserting that as the first column, and then differentiating
with respect to \(y\) and inserting that as the second column.

    \begin{tcolorbox}[breakable, size=fbox, boxrule=1pt, pad at break*=1mm,colback=cellbackground, colframe=cellborder]
\prompt{In}{incolor}{104}{\boxspacing}
\begin{Verbatim}[commandchars=\\\{\}]
\PY{c+c1}{\PYZsh{}Returns the Jacobian of the g function above at a point (x1,x2)=x}
\PY{k}{def} \PY{n+nf}{Jacob}\PY{p}{(}\PY{n}{x}\PY{p}{)}\PY{p}{:}
    \PY{n}{x1}\PY{p}{,} \PY{n}{x2} \PY{o}{=} \PY{n}{x}
    \PY{n}{dx1} \PY{o}{=} \PY{n}{np}\PY{o}{.}\PY{n}{array}\PY{p}{(}\PY{p}{[}\PY{l+m+mi}{1}\PY{p}{,}\PY{l+m+mi}{0}\PY{p}{]}\PY{p}{)}\PY{o}{.}\PY{n}{reshape}\PY{p}{(}\PY{p}{(}\PY{l+m+mi}{2}\PY{p}{,}\PY{l+m+mi}{1}\PY{p}{)}\PY{p}{)} \PY{o}{\PYZhy{}} \PY{n}{C}\PY{o}{@}\PY{p}{(}\PY{n}{np}\PY{o}{.}\PY{n}{array}\PY{p}{(}\PY{p}{[}\PY{l+m+mi}{6}\PY{o}{*}\PY{n}{x1}\PY{p}{,}\PY{o}{\PYZhy{}}\PY{l+m+mi}{24}\PY{o}{*}\PY{n}{x1}\PY{o}{*}\PY{o}{*}\PY{l+m+mi}{2}\PY{p}{]}\PY{p}{)}\PY{o}{.}\PY{n}{reshape}\PY{p}{(}\PY{p}{(}\PY{l+m+mi}{2}\PY{p}{,}\PY{l+m+mi}{1}\PY{p}{)}\PY{p}{)}\PY{p}{)}
    \PY{n}{dx2} \PY{o}{=} \PY{n}{np}\PY{o}{.}\PY{n}{array}\PY{p}{(}\PY{p}{[}\PY{l+m+mi}{0}\PY{p}{,}\PY{l+m+mi}{1}\PY{p}{]}\PY{p}{)}\PY{o}{.}\PY{n}{reshape}\PY{p}{(}\PY{p}{(}\PY{l+m+mi}{2}\PY{p}{,}\PY{l+m+mi}{1}\PY{p}{)}\PY{p}{)} \PY{o}{\PYZhy{}} \PY{n}{C}\PY{o}{@}\PY{p}{(}\PY{n}{np}\PY{o}{.}\PY{n}{array}\PY{p}{(}\PY{p}{[}\PY{l+m+mi}{8}\PY{o}{*}\PY{n}{x2}\PY{p}{,}\PY{l+m+mi}{3}\PY{o}{*}\PY{n}{x2}\PY{o}{*}\PY{o}{*}\PY{l+m+mi}{2}\PY{p}{]}\PY{p}{)}\PY{o}{.}\PY{n}{reshape}\PY{p}{(}\PY{p}{(}\PY{l+m+mi}{2}\PY{p}{,}\PY{l+m+mi}{1}\PY{p}{)}\PY{p}{)}\PY{p}{)}
    
    \PY{k}{return} \PY{n}{np}\PY{o}{.}\PY{n}{concatenate}\PY{p}{(}\PY{p}{(}\PY{n}{dx1}\PY{p}{,} \PY{n}{dx2}\PY{p}{)}\PY{p}{,} \PY{n}{axis}\PY{o}{=}\PY{l+m+mi}{1}\PY{p}{)}
\end{Verbatim}
\end{tcolorbox}

    \begin{tcolorbox}[breakable, size=fbox, boxrule=1pt, pad at break*=1mm,colback=cellbackground, colframe=cellborder]
\prompt{In}{incolor}{105}{\boxspacing}
\begin{Verbatim}[commandchars=\\\{\}]
\PY{n}{J} \PY{o}{=} \PY{n}{Jacob}\PY{p}{(}\PY{n}{alpha}\PY{p}{)}
\PY{n}{norm} \PY{o}{=} \PY{n}{np}\PY{o}{.}\PY{n}{linalg}\PY{o}{.}\PY{n}{norm}\PY{p}{(}\PY{n}{J}\PY{p}{,} \PY{n}{np}\PY{o}{.}\PY{n}{inf}\PY{p}{)}

\PY{n+nb}{print}\PY{p}{(}\PY{l+s+s1}{\PYZsq{}}\PY{l+s+s1}{Infinity norm of Jacobian at root: }\PY{l+s+si}{\PYZpc{}f}\PY{l+s+s1}{\PYZsq{}} \PY{o}{\PYZpc{}} \PY{n}{norm}\PY{p}{)}
\end{Verbatim}
\end{tcolorbox}

    \begin{Verbatim}[commandchars=\\\{\}]
Infinity norm of Jacobian at root: 0.039322
    \end{Verbatim}

    We can clearly see that the this value is strictly less than 1. Thus we
know that there is an open ball around the solution where any fixed
point iteration will converge. This is what makes our choice for
\(\mathbf{g}(\mathbf{x})\) a good choice.


    % Add a bibliography block to the postdoc
    
    
    
\end{document}
